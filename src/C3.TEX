%
%  Beschreibung von SPEED 2
%
\beginsection{SPEED~2}
\label{sec:speed2}
%
Der SPEED~2 ist die zweite Anlage aus der Serie der
\underline{s}chnellen
\underline{P}inch\underline{e}ntladungs\underline{e}xperimente in
\underline{D}üsseldorf. Die Anlage wurde bereits 1985 in Betrieb
genommen. Sie wird von der Fokusgruppe der Heinrich-Heine-Universität
betrieben und ist auch noch heute in seinem Entladungscharakter
einzigartig.
\par
Die hier beschriebene Variante der Anlage mit gepulstem Gaseinlaß für
Hoch-Z-Gase wurde seit 1988 eingesetzt. Überwiegend wurde seit 1990 der
MPM untersucht. Ab 1995 verschob sich der Schwerpunkt der Messungen auf
den SCM.
\par
Die wichtigsten Daten des Treibers SPEED~2 sind in der Tabelle
\vref{tab:treiberdaten} zusammengefaßt. Die Werte in der Spalte \glqq
maximal\grqq\ beziehen sich auf das erreichbare Maximum des Treibers.
Dabei werden aber die elektrischen Isolatoren des Gerätes und der
eingebauten Kondensatoren bis an die Zerstörungsgrenze belastet. Diese
Werte werden im typischen Betrieb nicht ausgenutzt und nur für
spezielle Versuche annähernd erreicht.
%
\par
\begin{table}[H]
  \center
  \begin{tabular}{|l|c|c|}
    \hline
                      & maximal          & typisch                 \\
    \hline
    Bankenergie       & \wert{187}{kJ}   & \wert{67.5}{kJ}         \\
    \hline
    Ladespannung      & \wert{300}{kV}   & \wert{180}{kV}          \\
    \hline
    Pinchstrom        & \wert{2.5}{MA}   & \wert{1.0}{MA}         \\
    \hline
    Stromanstieg      & \wert{20}{kA/ns} & \wert{10}{kA/ns}        \\
    \hline
    Stromanstiegszeit & \multicolumn{2}{c|}{ \wert{400}{ns} }      \\
    \hline
    Kapazität         & \multicolumn{2}{c|}{ \wert{4.16}{$\mu$F} }  \\
    \hline
    Impedanz          & \multicolumn{2}{c|}{ \wert{60}{m$\Omega$} } \\
    \hline
  \end{tabular}
  \caption{Elektrische Daten vom Experiment SPEED~2}
  \label{tab:treiberdaten}
\end{table}
%
\par
Auffällig an den Daten ist die große Impedanz von \wert{60}{m$\Omega$}.
Damit ist während der überwiegenden Zeit der Entladung die Impedanz des
Treibers größer als die der Last (Plasma). Es handelt sich also um eine
Stromquelle. (Bei einer stabilen Stromquelle ist immer $\rm R_{\rm
innen} \gg \rm R_{\rm Last}$, bei einer stabilen Spannungsquelle ist
immer $\rm R_{\rm innen} \ll \rm R_{\rm Last}$.)
\par
Der schnelle Stromanstieg wird durch einen sehr niederinduktiven Aufbau
der Anlage erreicht. Die Plasmafokusanlage SPEED~2 wird in den
folgenden Abschnitten dieses Kapitels \thesection\ beschrieben. Dabei
wird die Anlage relativ detailliert beschrieben. So werden die nötigen
Informationen für das Verständnis der einzelnen Betriebsparameter
geliefert. Im Kapitel \vref{sec:betriebsparameter} wird dann
vorgestellt, wie die Entladungsmodi von diesen Parametern bestimmt
werden.
%
\beginsubsection{Treiber SPEED~2}
%
Das Experiment kann in zwei Funktionseinheiten aufgeteilt werden. Die erste
Einheit ist der Treiber, der als Quelle den Strom möglichst schnell und stark
in die Last treibt. Die zweite Einheit, die Last, ist das eigentliche
Plasmaexperiment in einer z-Pinch- oder Fokus-Geometrie. In dieser Arbeit wird
nur von Entladungen in Fokus-Geometrie berichtet.
%
\par
\begin{figure}[H]
  \center
  \fbox{\importimage{C3-04}}
  \caption{Schaltung eines Marx-Moduls des Treibers SPEED~2.
      Der innere, rotationssymmetrische Aufbau der
      Funkenstrecken ist schematisch wiedergegeben.}
  \label{fig:modul}
\end{figure}
%
\par
Der Treiber SPEED~2 ist aus Marx-Modulen aufgebaut. Bei einem
Marx-Modul werden Kondensatoren parallel aufgeladen und in einer
Serienschaltung entladen. Dieses führt zu einer
Spannungsvervielfachung.
\par
Die SPEED~2 Module bestehen aus je 6
Kondensatoren\footnote{Maxwell Laboratories, San Diego, USA
(\url{http://www.maxwell.com/}). Jeder Kondensator besteht
wiederum aus 2 Packs mit je 3 Kondensatorelementen. Die Packs sind
in Serie, die Kondensatorelemente parallel geschaltet. Ein
Kondensatorelement ist ein niederinduktiv angeschlossener
Folienkondensator. Die Spannungsfestigkeit wird durch eine
Ölfüllung verbessert. Herstellergarantie: \wert{40}{kV}.} die mit
drei Funkenstrecken in Serie geschaltet werden. Die
Spannungsfestigkeit der Kondensatoren von \wert{50}{kV} führt zu
einer Grenze für die erreichbare Gesamtspannung.
\par
Die Funkenstrecken geben durch ihren Arbeitsbereich eine untere
Grenze für die Gesamtspannung. Der Druck in den Funkenstrecken
kann nicht beliebig abgesenkt werden, weil nach einer Entladung
ein Teil der Rückstände mit einem Luftstrom bei diesem Druck aus
den Funkenstrecken transportiert werden. Nur die flache Kennlinie
(Ver"-zö"-ge"-rungs"-zeit gegen angelegte Spannung) dieses
Funkenstreckentyps erlaubt die einfache Variation der Spannung des
Treibers im Bereich von \wert{150}{kV} bis \wert{300}{kV}.
\par
Die Abbildung \vref{fig:modul} zeigt die Schaltung eines
Marx-Moduls von SPEED~2.
\par
Die eingezeichnete Induktivität L (lange, hochspannungsfeste Spule
ohne Kern) ist nur in einem der 40 Module bestückt, über sie
fließt der Ladestrom der Kondensatoren. Bei der Entladung über den
Plasmafokus fließt nur ein kleiner Teil des Stromes durch die
Spule.
%
\par
\begin{figure}[H]
  \center
  \fbox{\importimage{C3-05}}
  \caption{Aufbau der Batterie von SPEED~2 (maßstabsgetreu, Details vereinfacht)}
  \label{fig:aufbau}
\end{figure}
%
\par
Die Abbildung \vref{fig:aufbau} zeigt die räumliche Anordnung der 40
Module des Treibers. Der Hauptkollektor, die Vorkollektoren und die
Leitungen zu den einzelnen Modulen sind niederinduktiv als Bandleiter
ausgeführt.
\par
Die große Anzahl der parallelgeschalteten Module führt zur der geringen
Gesamtinduktivität und zu dem großen Maximalstrom.
\par
Daneben hat die Parallelschaltung einen wichtigen Aspekt für den regelmäßigen
Experimentierbetrieb. Die Anlage kann auch dann betrieben werden, wenn einzelne
Module ausgefallen sind. In den Meßreihen macht sich dieses durch leicht
veränderte Bankenergieen bei gleicher Ladespannung bemerkbar. Die geringen
Änderungen der Bankenergie müssen hingenommen werden, wenn nicht, bei Verzicht
auf die maximal mögliche Energie, ein Teil der Module als Ersatzmodule
bereitgehalten werden.
\par
Da sich beim Abschalten, die Induktivität im gleichen Verhältnis
vergrößert, wie die Kapazität sich verringert, wird die
Stromanstiegszeit dabei nicht verändert. Dieses wurde auch
ausgenutzt, um die Bankenergie als Betriebsparameter zu variieren.
Damit die Symmetrie der Entladung nicht gestört wird, wurden die
Module möglichst symmetrisch verteilt geschaltet.
%
\beginsubsection{Plasmafokus SPEED~2}
\label{sec:plasmafokus}
%
Die Last dieses Hochleistungstreibers bildet ein Plasmafokus. Die verwendete
Elektrodenkonfiguration ist in der Abbildung \vref{fig:elektroden} dargestellt.
In der Abbildung ist die Anode nach unten abgeschnitten. Die Gesamtlänge der
Anode beträgt ca. \wert{30}{cm}.
%
\par
\begin{figure}[H]
  \center
  \fbox{\importimage{C3-02}}
  \caption{Elektrodenkonfiguration von SPEED~2 (maßstabsgetreu,
     Details vereinfacht). Elektroden: Kupfer, Isolator: \AlO beschichtetes
     Quarzglas.}
  \label{fig:elektroden}
\end{figure}
%
\par
Die Kathodenstäbe aus Kupfer dienen bei dieser Geometrie nicht als
Rück"-lei"-ter für den Strom. Sie erzeugen nur das notwendige
elektrische Feld im Bereich des Isolators für eine homogene
Ausbildung der Gleitentladung über der Isolatoroberfläche. Ohne
diese Stäbe kommt es zu keiner effizienten Pinchbildung aufgrund
der schlechten Zündung.
\par
Die Oberfläche des \wert{5}{mm} dicken Quarzglasisolators ist
\wert{0.2}{mm} dick mit \AlO\ beschichtet. Diese Beschichtung hat das
Zündverhalten bei großen Energiedichten gegenüber unbeschichteten
Isolatoren deutlich verbessert. Die Problematik der Isolatoren wird in
dem Abschnitt \vref{iso:problem} ausführlich beschrieben.
\par
Die Anode aus Kupfer ist auf der Achse durchbohrt. Durch diese
\wert{8}{mm} Bohrung wird ein Injektionsgas in den Bereich der
Pinchsäule gegeben. Den Ablauf der Entladung in dieser speziellen
Situation zeigt die Abbildung \vref{fig:pinchbildung}.
%
\par
\begin{figure}[H]
  \center
  \fbox{\importimage{C3-01}}
  \caption{Schichtentwicklung und Gasinjektion bei SPEED~2 (schematisch).
  Maßstabsgetreu ist ein Bild einer Röntgenabsorptionsmessung
  des Argongases in die Zeichnung montiert. Das Blickfeld beschneidet das
  Bild des Gasstroms links und oben.}
  \label{fig:pinchbildung}
\end{figure}
%
\par
Der zeitliche Ablauf einer Entladung mit Gasinjektion: Das
Entladungsgefäß wird mit einer Deuterium-Gasfüllung von typisch
\wert{5}{hPa} vorbereitet. Über das Ventil wird das Injektionsgas,
typisch Neon oder Argon, eingeblasen. Nach einer Zeitverzögerung im
ms-Bereich, damit das Gas durch die Bohrung austreten kann, wird die
Spannung angelegt. Es bildet sich die Gleitentladung über der
Isolatoroberfläche (1). Die Plasmaschicht hebt ab  (2) und wird durch
magnetische Kräfte komprimiert (3). Dabei wird zuerst nur Deuteriumgas,
dann auch im Bereich des Fokus Injektionsgas, ionisiert und
komprimiert.
\par
Es ist nicht möglich, mit einer statischen Fül"-lung eines
Gasgemisches zu beginnen, weil nur in reinem Wasserstoff- oder
Deuteriumgas die Schichtbildung am Isolator funktioniert.
Beimischungen von wenigen Volumenprozent eines schweren Gases
verhindern eine effektive Entladung.
\par
Die Gasinjektion wird von einem schnellen Magnetventil gesteuert. In
der Arbeit von Doll \cite{doll:diplom} ist das Ventil und die
Ansteuerungselektronik beschrieben. Im Vergleich zur ursprünglichen
Version wurde die Schaltung mittlerweile leicht modifiziert:
\par
Eine kleine Kondensatorbatterie mit einer Kapazität von
\wert{475}{$\mu$F} wird auf \wert{275}{V} aufgeladen. Geschaltet über
einen Thyristor werden die Kondensatoren über die Spule des
Magnetventils entladen, so daß ein Strompuls durch die Spule entsteht.
\par
Der Strom durch die Spule wurde mit einem Shunt zeitaufgelöst gemessen. Der
Maximalstrom von \wert{59}{A} wird nach einer Anstiegszeit von \wert{1.3}{ms}
erreicht. Danach fällt der Strom ab und erreicht nach \wert{4.6}{ms} seinen
10\%-Wert.
\par
Aufgrund der mechanischen Trägheit und der Kraft der
Rückstellfeder des Ventils, bewirkt der \wert{5.9}{ms} lange
Strompuls eine \wert{15}{ms} lange Öffnung des Ventils. Diese Zeit
wurde elektrisch gemessen. Ein Draht (\wert{\phi = 2}{mm}),
isoliert geführt durch die Auslaßseite des Ventils auf den
Ventilteller schließt über den Metallkörper des Ventils einen
Stromkreis. Beim Herunterziehen des Ventiltellers, durch das
Magnetfeld, wird der Kreis geöffnet. Bewegt sich der Teller
zurück, so schließt sich der Stromkreis, aber nur für einen kurzen
Moment (ca. \wert{0.5}{ms}), denn der Draht wird durch den Teller
weggestossen.
\par
Ein Versuch, die Öffnungszeit optisch zu messen, mißlang. Es besteht
keine direkte Sichtlinie auf den Kolben. Aufgrund der zu geringen
Intensität, war das Messen des Streulichts durch das Ventil, mit dem
vorhandenen Photomultiplier, nicht möglich.
\par
Die Gasmenge, die innerhalb der gesamten Öffnungszeit des Ventils in
den Rezipienten strömt, ist leicht zu messen. Aufgrund des großen
Volumens (\wert{193}{$\ell$}) erhöht sich der Druck nur gering (typisch
\wert{0.5}{hPa}) bei der Injektion. Bei diesen niedrigen Drücken kann
gut mit dem Gesetz für das ideale Gas gerechnet werden.
\par
Die Messungen wurden in Ab"-hän"-gig"-keit des Injektionsdrucks
p(Inj.) und des Gases durchgeführt. Gefunden wurde für die Menge
des Neons N(Ne) bzw. der Menge des Argons N(Ar) bei einer
Injektion mit dem Injektionsdruck p(Ne) bzw. p(Ar):
\begin{eqnarray*}
  \rm{N(Ne)} =
  & 1.06 \cdot 10^{-8} \,\rm \frac{mol}{Pa} \cdot \rm{p(Ne)}
  &\quad ,
  \\
  \rm{N(Ar)} =
  & 9.09 \cdot 10^{-9} \,\rm \frac{mol}{Pa} \cdot \rm{p(Ar)}
  &\quad .
\end{eqnarray*}
Die Abweichungen von der Linearität wurden im Bereich
\ewert{3.5}{5}{Pa} bis \ewert{9.5}{5}{Pa} zu maximal \wert{\pm 5}{\%}
bestimmt. Die Einstellung des Injektionsdrucks an einem handelsüblichen
Druckreduzierventil mit einer \wert{0.5}{bar} Skala, war vermutlich die
Ursache für die Abweichung von der Linearität. Der Druck wird am
Reduzierventil bei geschlossenem Magnetventil eingestellt. Beide
Ventile sind mit einem \wert{3.0}{m} langen Schlauch verbunden, der als
Vorratsgefäß für den Gaspuls dient. (Alle Schläuche sind hier aus
Polyethylen, Innendurchmesser \wert{6}{mm}, \wert{1}{mm} Wandstärke.)
\par
Die Gesamtmenge des Injektionsgases ist aber nicht entscheidend. Die
Entladung muß gezündet werden, bevor sich das Injektionsgas im
Rezipienten verteilt hat, oftmals wird sogar gezündet, bevor das Ventil
wieder geschlossen ist. Die gemessene Gasmenge dient hier nur zur
Übertragung der Messung von Argon auf Neon am Ende dieses Abschnitts.
Zusammen mit der Öffnungszeit, dienen die Werte auch zur
Plausibilitätskontrolle der folgenden Abschätzungen.
\par
Als Betriebsparameter wurde dazu in älteren Arbeiten
\cite{maelzig:phd,lucas:diplom,roewe:phd} die
Ver"-zö"-ge"-rungs"-zeit $\tau$ zwischen dem elektrischen
Monitorsignal des Magnetventils, abgeleitet aus dem Spannungspuls
zum Magneten, und dem Triggerpuls zu den HV-Trigger des Treibers
angegeben (vgl. dazu Abbildung \vref{fig:triggerung}, {\bf ZG}).
Diese Ver"-zö"-ge"-rungs"-zeit erhält aber auch die Zeit für das
Öffnen des Ventils, die Zeit für das Durchströmen des Schlauchs
vom Ventil zur Anode und die Verzögerungszeiten in den
HV-Triggern.
\par
Hier in der Arbeit wird eine effektive Injektionszeit \teff
angegeben. Sie berechnet sich für den aktuellen Aufbau der Anlage
aus der Ver"-zö"-ge"-rungs"-zeit $\tau$ über
\begin{eqnarray*}
  \tau \rm _{eff}(Ne) =
  & \tau - 4.5\,\rm ms
  &\quad ,
  \\
  \tau \rm _{eff}(Ar) =
  & \tau - 5.5\,\rm ms
  &\quad .
\end{eqnarray*}
Die effektive Injektionszeit \teff = 0 ist genau die minimale
Ver"-zö"-ge"-rungs"-zeit an, ab der die Entladungen durch ihre
Rönt"-gen"-strah"-lung auf den Diagnostiken sichtbar werden. Bei
kleineren Ver"-zö"-ge"-rungs"-zei"-ten (entsprechend negativen
effektiven Injektionszeiten) ist kaum schweres Gas in dem
Pinchplasma, das als Strahlungsquelle dienen könnte. Die Zeiten
wurden experimentell für beide Injektionsgase bestimmt. Im
wesentlichen ist es die Zeit für das Durchströmen des Schlauches,
dabei sind die leichteren Neon-Atome schneller. Weil der Offset
von der Schlauchlänge abhängig ist, muß beim Vergleich mit älteren
Arbeiten berücksichtigt werden, daß die Schlauchlänge von
\wert{0.35}{m} \cite{doll:diplom} über \wert{1.0}{m}
\cite{lucas:diplom} auf \wert{1.5}{m} verlängert wurde. Die
Verlängerungen waren notwendig geworden, da es immer wieder
Probleme mit Gleitentladungen entlang des Schlauches zum Ventil
gab.
\par
Mit der effektiven Injektionszeit \teff ist ein guter Betriebsparameter
gefunden, aber für weitere Auswertungen ist die Menge der
Injektionsgasteilchen im Pinch interessant. Diese ist aber meßtechnisch
schwer zugänglich. In der Arbeit von Mälzig \cite{maelzig:phd} ist eine
Messung beschrieben:
\par
Über eine Röntgenabsorptionsmessung konnten für einen Betriebspunkt Werte
ermittelt werden. Das in Abbildung \vref{fig:pinchbildung} montierte Photo
zeigt diese Messung. Der Gasstrom ist darauf als Schwärzung sichtbar.
\par
Die Ver"-zö"-ge"-rungs"-zeit $\tau$ beträgt bei der Aufnahme
\wert{6.5}{ms}, woraus sich die Injektionszeit \teff zu
\wert{4.5}{ms} abschätzen läßt (Schlauchlänge \wert{1.0}{m}). Der
Argondruck ist nicht angegeben, damals wurde oft mit
\ewert{4.5}{5}{Pa} gearbeitet, vermutlich auch bei diesem
Experiment.
\par
Für vier verschiedene Höhen über der Anode sind daraus die Dichten n
des Argongases und die Halbwertsbreite d des Gasstroms bestimmt worden.
Da die Dichten aus der Absorption berechnet wurden, unter der Annahme n
= const. über d, läßt sich in guter Näherung die Menge des Argongases
über der Zylindervolumen bestimmen. Es ergeben sich mit den Werte aus
\cite{maelzig:phd} die Gasmengen:
%
\par
\begin{table}[H]
  \center
  \begin{tabular}{|r|c|c|c|}
      \hline
    z-Bereich \ \  & Dichte n & Halbwertsbreite d & Gasmenge \\
      \hline
    0   - \wert{2.5}{mm}\  & \ewert{4.1}{23}{m$^{-3}$} & \wert{12.2}{mm} &  \ewert{1.2}{17}{} \\
    2.5 - \wert{5.5}{mm}\  & \ewert{2.9}{23}{m$^{-3}$} & \wert{12.6}{mm} &  \ewert{1.1}{17}{} \\
    5.5 - \wert{8.5}{mm}\  & \ewert{1.9}{23}{m$^{-3}$} & \wert{14.6}{mm} &  \ewert{9.5}{16}{} \\
    8.5 - \wert{11.5}{mm}\ & \ewert{1.4}{23}{m$^{-3}$} & \wert{15.4}{mm} &  \ewert{7.8}{16}{} \\
      \hline
  \end{tabular}
  \caption{Injektionsgasmenge (Teilchenanzahl) über der Anode}
  \label{tab:gasmengen:maelzig}
\end{table}
%
\par
In der Summe ergeben sich \ewert{6.7}{-7}{mol} in dem \wert{11.5}{mm}
hohen Gaskegel. Die Gesamtmenge des injizierten Argons, unter den
angenommenen Betriebsparametern, wurde mit \ewert{3.7}{-3}{mol}
gemessen. Die Teilmenge von 1/5500 innerhalb der ersten \wert{2}{ms}
von \wert{15}{ms} Öffnungszeit, innerhalb eines begrenzten Volumens,
erscheint möglich.
\par
Die Pinchsäule ist aber, gerade im SCM, nicht nur \wert{11}{mm}
lang. Aus Mangel an besseren Werten wird daher die mittlere
Liniendichte für Abschätzungen benutzt. Sie ergibt sich zu n =
\ewert{3.5}{19}{m$^{-1}$}. Die Liniendichte wird vermutlich
proportional zum Injektionsdruck p(Ar) sein, weil die
Gesamtgasmenge diese Proportionalität auch gut erfüllt. Die
Ab"-hän"-gig"-keit von der Injektionszeit ist sehr unsicher. Für
erste Ab"-schät"-zun"-gen wird auch dabei die Proportionalität
angenommen. Die Liniendichte für Argon läßt sich also aus den
Betriebsparametern mit
\begin{displaymath}
  \rm n(Ar) = 3.9 \cdot 10^{16} \, \rm m^{-1} \cdot \rm \frac{p(Ar)}{Pa} \cdot \frac{\tau_{\rm eff}}{s}
\end{displaymath}
abschätzen. Die Abschätzung für Neon erfolgt über das oben beschriebene
Verhältnis der Gesamtgasmengen:
\begin{displaymath}
  \rm n(Ne) = 4.5 \cdot 10^{16} \, \rm m^{-1} \cdot \rm \frac{p(Ne)}{Pa} \cdot \frac{\tau_{\rm eff}}{s}
\end{displaymath}
\par
Da dieser Zusammenhang auf einige unsichere Werte aufbaut, kann er nur
im Sinne einer ersten Abschätzung dienen. Eine bessere Kenntnis des
Gasstroms ist auf jeden Fall wünschenswert, aber experimentell
aufwendig.
\par
Dazu kommt auch noch das Problem, daß sich die Bohrung in der Anode für
die Gasinjektion bei den Entladungen verändert. Die Bohrung hat sich im
Laufe von ca. 2000 Entladungen auf mehr als den doppelten Durchmesser
erweitert und wurde auch leicht asymmetrisch. Werden auch Entladungen
ohne Gasinjektion durchgeführt, so wird auch eine deutliche Verengung
der Bohrung ca. \wert{5}{cm} innerhalb der Anode beobachtet. Das
Material, das im oberen Bereich abgetragen wird, wandert ohne
Gasinjektion in die Bohrung hinein.
\par
Zwei Schnitte durch die Anode in Abbildung \vref{fig:bohrung}
zeigen die Ver"-än"-de"-rung"-en der Bohrung. Die Anordnung der
Diagnostiken relativ zu diesen Schnitten kann der Abbildung
\vref{fig:Richtung:Diagnostiken} entnommen werden.
\par
Erfolglos wurde versucht, den Gasstrom durch ein Kupferrohr
(Außendurchmesser \wert{8}{mm}, Wandstärke \wert{1}{mm})
konzentriert zu halten. Dieses Rohr verlor durch Erosion pro
Entladung ca. \wert{1}{mm} seiner Länge, bis es ca. \wert{20}{mm}
in der Anode verborgen war und nicht mehr durch die Entladung
abgetragen wurde.
%
\par
\begin{figure}[H]
  \center
  \fbox{\importimage{C3-03}}
  \caption{Schnitte durch die Anode von SPEED~2 nach der Entladung Nummer 12.646}
  \label{fig:bohrung}
\end{figure}
%
%
\beginsubsection{Trigger von Treiber und Diagnostik}
%
\par
Die Abbildung \vref{fig:triggerung} zeigt den komplexen Ablauf in der
Ansteuerung der 40 Marx-Module des Treibers und der verschiedenen Diagnostiken
zur Untersuchung des entstehenden Plasmas.
\par
Die Kürzel in der Abbildung bezeichnen die eingesetzten Geräte:
 {\bf DSO}: Gould 4074 (4 Kanal, \wert{100}{MHz} Bandbreite,
\wert{400}{MS/s}), Gould Electronics, Essex, England
(\url{http://www.gould.co.uk/});
 {\bf HP}: Pulse Generator 8011A,
Hewlett Packard, Böblingen (\url{http://www.tmo.hp.com/});
 {\bf HV}: HV-Puls Generator für MCP
\cite{sopkin:92}, Institut für Spektroskopie, Troitzk, Rusland.
 {\bf MA}: H.V. Trigger Amplifier 40107, Maxwell Laboratories, San
Diego, USA;
 {\bf SL}: H.V. Trigger Generator 40108, Maxwell
Laboratories;
 {\bf ST}: Pulse Generator Model DG 535 (Options 01, 02),
Stanford Research Systems, Sunnyvale, USA (\url{http://www.srsys.com/});
 {\bf VS}: Ventilsteuerung,
Photoblitz-Kondensatoren über Thyristor geschaltet;
 {\bf ZG}: Puls Generator
\cite{maelzig:phd} als Verstärker und Schutz des ST vor elektrischen
Störungen.
\par
Alle Impulse werden über Koaxialleitungen mit 50~$\Omega$ Impedanz
über"-tra"-gen. Die Signallaufzeiten wurden ermittelt, bzw. die
Leitungslängen wurden entsprechend den Erfordernissen abgestimmt.
\par
Nach Aufladung der Kondensatorbatterie (Ladezeit typisch
\wert{12}{s}), erfolgt durch Tastendruck die Triggerung der
Geräte. Der Tastendruck löst das schnelle Magnetventil aus, über
ein Monitorsignal wird dann der zentrale Pulsgenerator {\bf ST}
ausgelöst.
\par
Ein Puls läuft durch den Pulser {\bf ZG}, als Verstärker und
Schutz des {\bf ST}, in die Triggerhierarchie des Treibers.
Mehrere kommerzielle Impulsverstärker {\bf MA}, {\bf SL} erzeugen
daraus die notwendigen 40 Hochspannungspulse zur gleichzeitigen
(innerhalb von \wert{10}{ns}) Triggerung der 120 Funkenstrecken in
den Marx-Modulen.
\par
Die anderen Pulse des zentralen Pulsgenerators {\bf ST} steuern
die zeitaufgelöst arbeitenden Diagnostiken an. Die scheinbar
widersinnige Verwendung eines \wert{30}{V} Pulses und eines
nachgeschalteten Dämpfers ist notwendig, damit der
Signal-Rauschabstand an dieser speziellen Stelle ausreichend groß
ist.
\par
Trotz der vielen Stufen in der Steuerung ist der Jitter mit ca.
\wert{\pm 30}{ns} zwischen der Fokusentladung und einer Diagnostik
gering.
%
\par
\begin{figure}[H]
  \center
  \fbox{\importimage{C3-06}}
  \caption{Triggerung der Batterie von SPEED~2 und der Diagnostiken.
     Beschreibung der Kürzel im Text.}
  \label{fig:triggerung}
\end{figure}
%
\par
Für eine genaue zeitliche Zuordnung werden von allen Diagnostiken
Monitorsignale zusammen mit dem Spannungs- und Stromsignal der
Entladung aufgezeichnet. Ausgelöst werden die zwei digitalen
Speicheroszilloskope ({\bf DSO}) durch einen Spannungspuls, der
über einen kapazitiven Spannungsteiler von einem der
Hochspannungspulse aus einem der {\bf SL} Generatoren gewonnen
wird.
