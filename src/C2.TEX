%
%  Entstehen von Pinchplasmen
%  Fokusanlagen
%  Instabilitäten
%  stabile Säule
%  Vergleich stabile Säule mit den Mikropinchen
%  Vergleich stabile Säule mit Entladungsformen anderer Anlagen
%
\beginsection{Pinchplasmen}
\label{sec:pinchplasmen}
%
Pinchplasmen können in verschiedensten Konfigurationen erzeugt
werden. Zur Zeit werden üblicherweise z-Pinch- oder
Plas"-ma"-fo"-kus-Geo"-me"-tri"-en verwendet. Beim Plasmafokus
handelt es sich um eine koaxiale Elektrodengeometrie
\cite{mather:65}. Die Bauformen des Plasmafokus werden in
Mather-Typ (Radius $<$ Länge) und Filippov-Typ (Radius $>$ Länge)
eingeteilt. Die hier untersuchten Pinchplasmen wurden in einem
Plasmafokus vom Mather-Typ erzeugt.
\par
Die wesentliche Dynamik des Plasmafokus läßt sich durch einfache
Überlegungen erschließen. Die Details des Entladungsablaufes
insbesondere bei der Zündung und in der Pinchphase sind erheblich
komplizierter und noch immer teilweise ungelöst.
\par
Die Abbildung \vref{fig:fokusschema} zeigt schematisch den Aufbau eines
Plasmafokus und  die Entwicklung der Plasmaschicht bis zur Bildung der
Pinchsäule.
%
\par
\begin{figure}[H]
  \center
  \fbox{\importimage{C2-02}}
  \caption{Schema eines Plasmafokus mit den Phasen (1) Zündung, (2) Laufphase,
     (3) Kompression, (4) Pinchphase einer Entladung.}
  \label{fig:fokusschema}
\end{figure}
%
\par
Der Entladungsablauf eines schnellen Plasmafokus läßt sich in 4 Phasen
unterteilen \cite{kies:phd}. {\bf (1) Zündung:} Der Schalter wird
geschlossen (als Schalter werden üblicherweise Funkenstrecken benutzt),
wobei die geladene Kondensatorbank mit den Elektroden verbunden wird.
In der Gasfüllung (typisch einige hPa) der Anlage kommt es nach dem
Zündverzug im ns-Bereich zur Ausbildung einer Gleitentladung über dem
Isolator. Die entstehende Plasmaschicht muß dünn und homogen sein,
damit eine effektive Pinchbildung möglich ist.
\par
{\bf (2) Laufphase:} Getrieben von Lorentz-Kräften
($\vec{j}\times\vec{B}$) hebt sich die Schicht vom Isolator ab und
dehnt sich aus (inverser Pincheffekt). Die Schicht besteht aus einer
ionisierenden Schockfront und einem stromleitenden Plasma. U.a. kann
durch die Anodenlänge die Zeitdauer der Laufphase eingestellt werden,
so daß die Pinchphase in der Nähe des Strommaximums liegt.
\par
{\bf (3) Kompression:} Am Ende der Anode wird die Schicht durch den
radialen Anteil der Lorentz-Kraft in Richtung der Achse gedrückt. Das
dabei ionisierte Gas wird aufgesammelt und komprimiert
(Schneepflugmodell).
\par
{\bf (4) Pinchphase:} Die Schicht erreicht mit hoher Geschwindigkeit
(typisch \fwert{5}{m/s}) die Achse. Durch Stöße wird die kinetische
Energie der Ionen in thermische Energie umgewandelt. Der Strom durch
die entstandene Pinchsäule trägt zusätzlich zur Heizung bei. Gegen den
Magnetfelddruck baut sich der Plasmadruck auf. Nach einer Lebensdauer
von etwa \wert{100}{ns} zerfällt die Pinchsäule.
\par
Das Verhalten nach der Pinchphase ist für die Untersuchung des
Pinchplasmas nicht von Interesse. Es kommt zu erneuten Zündungen
zwischen den Elektroden. Elektrisch verhält sich das System wie ein
gedämpfter LC-Schwingkreis. Im $\mu$s-Bereich ist die elektrische
Energie aufgebraucht und das Plasma kühlt ab.
\par
Es ist also nach der Kompression eine zylinderförmige Plasmasäule zu
erwarten. Trotzdem wird im Experiment oftmals keine Zylinderform
beobachtet. Experimentell werden überwiegend kleine und heiße Bereiche,
umgeben von einem wesentlich kälteren Plasma gefunden. Ursache dafür
ist das Auftreten von makroskopischen Instabilitäten. Schwache
Einschnürungen der Säule können sich innerhalb weniger ns auf
Durchmesser von unter \wert{0.1}{mm} zusammenziehen. Es gibt aber auch
Entladungen, bei denen diese Instabilitäten nicht beobachtet werden.
Die Art der Entladung kann durch die Betriebsparameter beeinflußt
werden, wenn der Betriebsbereich der Anlage hinreichend groß ist.
%
\beginsubsection{Entladungsmodi von SPEED~2}
%
Der Arbeitsbereich der schnellen und stromstarken Plasmafokusanlage SPEED~2 ist
ausreichend für den Betrieb in beiden Entladungsmodi.
\par
Die Entladungen mit Instabilitäten, die zu heißen, punktförmigen
Plasmabereichen mit einer Ausdehnung von unter \wert{0.1}{mm}
führen, werden dem sogenannten Mikropinchmodus (MPM für englisch
micropinch mode) zugeordnet. Entladungen, die zu einer heißen und
homogenen Plasmasäule führen, werden dem sogenannten stabilen
Säu"-len"-mo"-dus (SCM für englisch stable column mode)
zugeordnet.
\par
Da die heißen Plasmabereiche weiche Röntgenstrahlung ($\lambda < $
\wert{10}{nm}) abstrahlen, können die beiden Entladungsmodi auf
Bildern aus diesem Spektralbereich sofort identifiziert werden.
Die Abbildung \vref{fig:vergleich} zeigt entsprechende Aufnahmen
des Plasmas.
\par
Die obere Bildhälfte zeigt zwei Entladungen im MPM. Auf den beiden
Bildern sind die kleinen Strahlungszentren als Projektion des
Pinholes der Kamera zu sehen. Bei einer Entladung entstehen bis zu
10 dieser typisch \wert{30}{$\mu$m} großen Mikropinche. Sie wurden
detailliert untersucht, z.B. in der Arbeit von Röwekamp
\cite{roewe:phd}.
\par
Die untere Bildhälfte zeigt zwei Entladungen im SCM. Eine
ausführliche Untersuchung dieses Entladungsmodi ist im Kapitel
\vref{sec:untersuchung} beschrieben.
%
\par
\begin{figure}[H]
  \center
  \fbox{\importimage{C2-01}}
  \caption{Zeitintegrierte Röntgenpinholebilder. Oben: Mikropinchmodus,
     unten: stabiler Säu"-len"-mo"-dus, rechts/links: je zwei unmittelbar
     aufeinander folgende Entladungen. Die Mikropinchstrukturen sind
     im \wert{30}{$\mu$m} Bereich, der Durchmesser der Säule ist \wert{\le
     1}{mm}. (bzgl. der Rönt"-gen"-pin"-hole"-kame"-ra siehe Abschnitt \vref{pinholekamera})}
  \label{fig:vergleich}
\end{figure}
%
%
\par
\begin{table}[H]
  \center
  \begin{tabular}{|l|c|c|}
  \hline
                               & oben            & unten             \\
  \hline
    Ladespannung U             & \wert{180}{kV}  & \wert{210}{kV}      \\
    Bankenergie E              & \wert{63}{kJ}   & \wert{86}{kJ}       \\
    Füllgas                    & \multicolumn{2}{c|}{ Deuterium 2.7 }  \\
    Fülldruck p(D$_2$)         & \wert{4.8}{hPa} &  \wert{4.9}{hPa}    \\
    Injektionsgas              & Argon 4.6       &  Neon 4.0           \\
    Injektionsdruck p(Inj.)    & \multicolumn{2}{c|}{ \ewert{4.5}{5}{Pa} }   \\
    Injektionszeit \teff       & \wert{1.3}{ms}  & \wert{4.5}{ms}      \\
    Nummer                     & 12.343, 12.344 & 11.037, 11.038       \\
  \hline
  \end{tabular}
  \caption{Parameter der Entladungen in Abbildung \ref{fig:vergleich}. Die Parameter
    werden im Kapitel \ref{sec:speed2} ausführlich vorgestellt.}
  \label{tab:vergleich:para}
\end{table}
%
\par
Ein für Diagnostik und Anwendung wichtiger Unterschied zwischen den
Modi ist die gute Reproduzierbarkeit der Plasmasäule im SCM. Die beiden
unmittelbar aufeinanderfolgenden Entladungen im MPM zeigen erhebliche
Unterschiede in den Positionen der Strahlungszentren. Dagegen gleichen
sich die Plasmasäulen sehr stark.
\par
Diese gute Reproduzierbarkeit im SCM wird natürlich nur erreicht, wenn
die Anlage einen stabilen Betriebspunkt erreicht hat. Dieser stellt
sich aber relativ schnell nach typisch 6 Entladungen ein, wenn die
Anlage morgens in Betrieb genommen wird. Dagegen scheiterten alle
Versuche, die Mikropinche auf festen Positionen zu erzeugen
\cite{lucas:diplom}.
\par
Ein weiterer Vorteil des SCM ist die Lokalisierung der Säule auf der
Achse des Fokus. Daher kann die Plasmasäule end-on als lokalisierte
Quelle genutzt werden. Die Mikropinche, insbesondere anodenferne,
treten auch achsenfern auf.
\par
Der Wechsel zwischen den beiden Modi wurde in diesem Fall durch den
Wechsel des Gases von Deuterium/Argon (MPM) zu Deuterium/Neon (SCM)
ausgelöst. Die weiteren Betriebsparameter der Entladungen sind in der
Tabelle \vref{tab:vergleich:para} zusammengestellt.
\par
Die Gasmischung in der Plasmafokusanlage ist nicht der einzige
Parameter, der den Entladungsmodus bestimmt. In dieser Arbeit wurde
systematisch untersucht, bei welchen experimentellen Gegebenheiten der
SCM auftritt und ob neben diesen beiden bekannten Modi noch weitere
Entladungsformen zu finden sind.
\par
Damit es zur Ausbildung einer stabilen Säule kommen kann, muß es einen
Stabilisierungsmechanismus geben, weil eine perfekte Schichtbildung
ohne die geringsten Inhomogenitäten unmöglich erscheint. Ein möglicher
Mechanismus wird im Kapitel \vref{sec:mechanismus} beschrieben.
%
\beginsubsection{Vergleich mit anderen Maschinen}
\label{sec:vergleich}
%
Das Auftreten verschiedener Entladungsmodi ist weder auf den
Plasmafokus SPEED~2 beschränkt, noch ist es ein neuer Effekt. Der
Vergleich mit anderen Plasmafoki zeigt, daß der SCM durch seine
Stabilität und Homogenität ausgezeichnet ist. Vom Wechsel zwischen
einem Mikropinchmodus und einem Modus ohne Mikropinche, dem
Säu"-len"-mo"-dus (CM für englisch column mode), wird in
verschiedenen Arbeiten berichtet.
\par
Die Tabelle \vref{tab:maschinen} zeigt die Plasmafokus-Anlagen bei denen
ähnliche Entladungsmodi beobachtet wurde (ohne Anspruch auf Vollständigkeit).
%
\par
\begin{table}[H]
  \center
  \begin{tabular}{|l|l|l|l|l|}
  \hline
                       & SPEED~2    &         & DPF-78    & KPF-3 \\
    Standort           & Düsseldorf & Aachen  & Stuttgart & Sukhumi \\
  \hline
    Ladespannung U/kV  & 180        & 10-20   & 60        & ca. 40 \\
    Bankenergie E/KJ   & 63         & 1.5-6   & 28        & 100-150 \\
    Kapazität C/$\mu$F & 4.16       & 32      & 15        & ca. 500 \\
    Pinchstrom I/MA    & 1.0        & 0.1-0.4 & 0.9       & 1.5 \\
    $\phi$-Anode/cm    & 10         & 2       & 5         & ca. 10 \\
  \hline
  \end{tabular}
  \caption{Technische Daten von SPEED~2 im Vergleich zu anderen
     Plasmafokus-Anlagen \cite{lebert:95,antsiferov:95,koshelev:90,yurii:priv}.}
  \label{tab:maschinen}
\end{table}
%
\par
Der Plasmafokus an der RWTH Aachen (Lehrstuhl für Lasertechnik)
\cite{lebert:95} arbeitet mit einer stationären \wert{1-3}{hPa}
Gasfüllung, die weiteren Daten sind in der Tabelle \ref{tab:maschinen}
aufgeführt. Der Wechsel zwischen dem MPM und dem CM erfolgt durch Wahl
des Gases. Bei Gasen mit Z $\le$ 18 erfolgt die Entladung im CM und bei
Z $\ge$ 18 im MPM. Dieses Verhalten ist analog zum Fokus SPEED~2. Beim
SPEED~2 ist es darüberhinaus möglich, die Grenze in Z durch
verschiedene Betriebsparameter einzustellen (für Z = 18 (Ar), 14 (Si),
10 (Ne) experimentell gezeigt).
\par
Bei einer Argon-Gasfüllung ist der Entladungsmodus vom Fülldruck
abhängig. Bei hohem Fülldruck stellt sich der MPM ein, wie es auch in
analoger Weise beim SPEED~2 der Fall ist.
%
\par
\begin{figure}[H]
  \center
  \fbox{\importimage{C2-04}}
  \caption{Zeitintegriertes Röntgenpinholebild (\wert{\lambda < 1.2}{nm})
     eines Pinches in Argongas (\wert{1}{hPa}) an der Plasmafokus Anlage
     in Aachen \cite{lebert:95}.}
  \label{fig:lebert:pinhole}
\end{figure}
%
%
\par
\begin{figure}[H]
  \center
  \fbox{\importimage{C2-03}}
  \caption{Röntgenstreakbild mit radialer Auflösung (\wert{\lambda < 1.2}{nm})
     eines Pinches in Argongas (\wert{1}{hPa}) an der Plasmafokus Anlage
     in Aachen \cite{lebert:95}.}
  \label{fig:lebert:streak}
\end{figure}
%
\par
Der CM der Anlage in Aachen unterscheidet sich aber auch in
wesentlichen Punkten von dem SCM.
\par
Die Homogenität der Plasmasäule ist deutlich schlechter als beim
SCM von SPEED~2, vgl. Abbildung \vref{fig:lebert:pinhole} mit
Abbildung \vref{fig:vergleich}. Auch die Länge der Pinchsäule ist
mit \wert{4.5}{mm} deutlich kleiner, als die von SPEED~2 mit
\wert{10-30}{mm}. Die Durchmesser liegen im gleichen Bereich.
\par
An der Röntgenstreakaufnahme in Abbildung \vref{fig:lebert:streak} wird
deutlich, daß auch die zeitliche Stabilität des CM deutlich geringer
ist als die des SCM, vgl. Abbildung \vref{fig:schnitteA}. Die
Lebensdauer des SCM ist mit \wert{30-100}{ns} deutlich länger als die
ca. \wert{10}{ns} beim Aachener Plasmafokus.
\par
Antsiferov et al. \cite{antsiferov:95} berichteten von zwei
Entladungsmodi am Plasmafokus DPF-78 (Institut für
Plasmaforschung, Stuttgart). Der Fokus wird mit einem stationärem
Gasgemisch bei \wert{3}{hPa} betrieben. DPF-78 erreicht fast den
Pinchstrom von SPEED~2 (im typischen Betriebsbereich), aber mit
einer deutlich höheren Stromanstiegszeit von \wert{1500}{ns}
gegenüber \wert{400}{ns} bei dem Plasmafokus SPEED~2.
\par
Neben dem MPM und dem CM wird ein Übergangsmodus mit kurzen Pinchsäulen
(\wert{3-6}{mm}) und Mikropinchen beobachtet. In dieser Form existiert
der Übergangsbereich nicht bei SPEED~2 Entladungen. Dort gibt es einen
Bereich, in dem anfangs die Pinchsäule entsteht und sich zu späteren
Zeiten in dem Plasma Mikropinche ausbilden.
\par
Der Wechsel zwischen den Entladungsmodi erfolgt über den Anteil an
schwerem Gas (Ar, Kr, Xe). Analog zum Verhalten in SPEED~2 und anderen
Plasmafoki führt eine Erhöhung des Anteils zum Übergang vom CM in dem
MPM.
%
\par
\begin{figure}[H]
  \center
  \fbox{\importimage{C2-05}}
  \caption{Zeitintegriertes  Röntgenpinholebild (\wert{\lambda < 0.6}{nm}) eines Pinches
           am Plasmafokus DPF-78 in Stuttgart \cite{antsiferov:95}.}
  \label{fig:antsiferov:pinhole}
\end{figure}
%
\par
Die Struktur der Pinchsäule ist in der Abbildung
\vref{fig:antsiferov:pinhole} gezeigt. Das Bild zeigt nur einen
\wert{4}{mm} langen Ausschnitt der Säule, die mit einer Länge von
\wert{10-20}{mm} angegeben ist. Die Homogenität in diesem Bereich
ist vergleichbar mit dem SCM von SPEED~2. Ebenso liegt der
Durchmesser der Plasmasäule von \wert{0.5-1}{mm} im Bereich der
bei SPEED~2 auftretenden Säule.
\par
Für einen besseren Vergleich fehlen Angaben zur Lebensdauer und
Struktur der gesamten Säule.
\par
Ein weiterer Plasmafokus bei dem verschiedenen Betriebsmodi
beobachtet wurden der KPF-3 vom I.N.~Vekua Physicotechnical
Institute, Sukhumi \cite{koshelev:90}. Die angegebenen technischen
Daten sind wieder in der Tabelle \vref{tab:maschinen}
zusammengefaßt. Mit einem Arbeitsgas von D$_2$ + 1\% Ar bei
\wert{8-10}{hPa} wird eine Säu"-len"-struk"-tur beobachtet. Bei
einem höheren Anteil von Argon beginnt auch bei dieser Anlage der
MPM.
\par
Die Beimischung von Xenon-Gas zum D$_2$-Gas führt auch bei einem
kleinen Anteil zum MPM. Die Verschiebung der Grenze zu kleineren
Dichten bei Vergrößerung von Z wird auch bei SPEED~2 beobachtet.
\par
Informationen zur Struktur des CM beim KPF-3 liefert die Abbildung
\vref{fig:koshelev:pinhole}. Der Maßstab konnte dabei nur aus dem Text
konstruiert werden. Deutlich ist die schlechtere Homogenität der Säule
gegenüber den Entladungen \vref{fig:vergleich} von SPEED~2 zu erkennen.
\par
Die Pinchsäule ist kürzer (\wert{5-10}{mm} gegen \wert{10-30}{mm}) und hat
einen größeren Durchmesser (\wert{3}{mm} gegen \wert{\le1}{mm}).
\par
Alleine aufgrund der schlechteren Homogenität kann hier nicht von einer
stabilen Pinchsäule gesprochen werden. Angaben zu Lebensdauer und
Reproduzierbarkeit fehlen auch hier.
%
\par
\begin{figure}[H]
  \center
  \fbox{\importimage{C2-06}}
  \caption{Zeitintegriertes Röntgenpinholebild (\wert{\lambda < 3}{nm}) eines Pinches in
  Deuterium/Argon Gasgemisch an der Plasmafokus Anlage KPF-3 in Sukhumi \cite{koshelev:90}.}
  \label{fig:koshelev:pinhole}
\end{figure}
