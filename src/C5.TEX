%
%  Untersuchung des Lebensraums der stabilen Säule
%
\beginsection{Betriebsparameter}
\label{sec:betriebsparameter}
%
Dieses Kapitel \thesection\ beschäftigt sich mit der
Fragestellung, wie das Auftreten des SCM von den Parametern des
Experiments abhängt. Die Kenntnis dieser Ab"-hän"-gig"-keit ist
notwendig, um den SCM bei den nachfolgenden detaillierten
Untersuchungen zuverlässig zu reproduzieren und, ganz nach den
Notwendigkeiten der Meßvorhaben, in den MPM zu wechseln. So können
Meßreihen zuverlässig geplant werden und, auch bei dem Zeitdruck,
der bei Mitarbeit von internationalen Gästen am Experiment
auftritt, durchgeführt werden. Die Grenze zwischen dem MPM und dem
SCM wurde aber genauer betrachtet als für diese Anwendung
notwendig ist, damit eine Grundlage für die Entwicklung und
Kontrolle theoretischer Modelle gegeben wird.
\par
Diese Aufgabenstellung unterscheidet sich im Charakter von den bisher
üblichen Arbeiten an der Anlage SPEED~2. Bisher wurden vornehmlich
einzelne Entladungen betrachtet. Diese wurden mit einer Vielzahl von
Diagnostiken untersucht. Dieses Verfahren wird auch hier wieder im
Kapitel \ref{sec:untersuchung} aufgenommen. In diesem Kapitel
\thesection\ werden nicht einzelne Entladungen herausgearbeitet,
sondern vollständige Meßreihen aus vielen Entladungen werden ohne
Selektion vorgestellt. Es sind keine ausgewählten Entladungen, daher
sind die einzelnen Ergebnisse nicht immer ideal im Sinne der
wichtigsten physikalischen Effekte für den speziellen Entladungsmodus.
Die Schuß-zu-Schuß Schwankungen bei Experimenten dieser Art stören die
Meßreihen, aber selbst mit diesen großen quasi-statistischen
Schwankungen, ist der Einfluß der Betriebsparameter auf den
Entladungsmodus gut erkennbar. Durch die vollständige Beachtung aller
Einzelmessungen wird deutlich, daß es möglich ist, nach Wunsch gezielt
einen der beiden Modi sicher zu produzieren.
\par
Die große Anzahl von experimentell zugänglichen Parametern spannt den
zu untersuchenden multidimensionalen Raum auf. Ein vollständiges
Ausmessen der Hypergrenzfläche zwischen den beiden Modi ist daher sehr
aufwendig, das Isolatorproblem (siehe Abschnitt \vref{iso:problem})
macht einen solchen Ansatz sogar unmöglich. Einzelne Meßreihen mit mehr
als 225 Entladungen (abzüglich typisch 20\% Reinigungsentladungen)
können, aufgrund der mittleren Lebensdauer eines Isolators, nicht
realistisch geplant werden. Die großen Unterschiede beim Verhalten des
Experiments zwischen verschiedenen, baugleichen Isolatoren erlaubt kein
einfaches Vergleichen der Resultate. Die Experimente wurden daher auf
die wichtigsten Grenzpunkte und Grenzlinien beschränkt. Durch
Interpolation dieser Meßreihen, ergibt sich eine ausreichende Übersicht
über den gesamten Parameterraum.
\par
Neben der gezielten Untersuchung dieser Grenzlinien, wurden auch
Meßreihen durchgeführt, die sich von dieser Grenze weit entfernen.
Dabei wurde experimentell nachgewiesen, daß keine weitere Änderung der
Entladungscharakteristik in andere Modi erfolgt. Die Parameter wurden
oftmals bis in den Bereich nicht effizienter Entladungen, mit geringer
Röntgenstrahlungsausbeute und Neutronenproduktion, geführt.
\par
Bei der Durchmusterung wurde die Art der entstandenen
Rönt"-gen"-strah"-lungs"-quel"-le untersucht. Benutzt wurde eine
Rönt"-gen"-pin"-hole"-kame"-ra, (Abschnitt \vref{pinholekamera})
die pro Entladung vier unterschiedlich gefilterte, zeitintegrierte
Bilder aufgenommen hat. Diese Aufnahmen lassen die Geometrie des
strahlenden Plasmas erkennen, erlauben also die Unterscheidung
zwischen MPM und SCM. Weiterhin lassen diese Bilder eine erste
Abschätzung der Temperatur aufgrund der Härte der emittierten
Strahlung zu. Dieser Aspekt wird in diesem Kapitel \thesection\
nicht weiter ausgeführt.
\par
Diese einfache Diagnostik wurde gewählt, um die Meßreihen möglichst
zügig durchzuführen. Ein Filmwechsel bei der gewählten Kamera war nur
nach 12 untersuchten Entladungen notwendig. Aufwendigere Diagnostiken
sind für diese erste Fragestellung nicht erforderlich.
\par
Wie bei allen Entladungen an SPEED~2 wurden die U(t) und $\dot{\rm
I}$(t) Signale und die Werte der Silberaktivierungszähler auch bei
diesen Meßreihen aufgenommen, die in einigen Fällen zusätzliche
Hinweise lieferten.
\par
Die untersuchten Entladungen wurden in drei Gruppen geteilt: (1)
Entladungen mit Röntgenstrahlung unter oder knapp oberhalb der
Nachweisgrenze. Diese wurden nicht weiter ausgewertet oder
gezählt. (2) Entladungen, die auf dem Film nur eine Säule als
Rönt"-gen"-strah"-lungs"-quel"-le zeigen, wurden dem SCM
zugeordnet. (3) Entladungen, die einen oder mehrere kleine, fast
punktförmige, Strahlungsquellen zeigen, wurden dem MPM zugeordnet.
Das gilt auch für die Entladungen bei denen eine säulenförmige
Struktur und punktförmige Quellen erkennbar sind. Die Einteilung
in dieser Art wurde bereits in der Arbeit \cite{roewe:phd}
benutzt. Die Mischform kann auch in eine eigene Gruppe gefaßt
werden. In dieser Durchmusterung wurde diese feinere Einteilung
nicht vorgenommen, weil die Grenze zwischen \glqq Säule noch
schwach sichtbar\grqq\ und \glqq keine Säule\grqq\ schwer zu
treffen ist. Das Auftreten von punktförmigen Quellen ist dagegen
sehr sicher erkennbar.
\par
Eine kontinuierliche Teilung in der Art: $+1 \leftrightarrow$ SCM, $-1
\leftrightarrow$ MPM berechnet aus (Strahlungsenergie aus Säule -
Strahlungsenergie aus Mikropinchen) / (Strahlungsenergie aus gesamten Fokus)
ist eine weitere Alternative. Die Strahlungsenergie läßt sich aus der
Schwärzung des Films bestimmen, wenn durch entsprechende Filterung dafür
gesorgt wird, daß die Sättigung nicht erreicht wird. Für eine erste
Charakterisierung ist eine solche Auswertung sicherlich zu aufwendig.
%
\beginsubsection{Injektionszeit}
%
Ein leicht zugänglicher und in weiten Grenzen variabler Parameter ist die
Injektionszeit. Abhängig von der Injektionszeit ist die Anzahl der
Injektionsgasteilchen im Entladungsraum und die räumliche Ausdehnung der
Injektionsgaswolke in der Deuteriumgasfüllung.
\par
Die übrigen Parameter wurden für diese Meßreihe aus dem typischen
Arbeitsbereich des Experiments gewählt. Der Deuteriumfülldruck p(D$_2$)
wurde, wie üblich, auf maximale Neutronenproduktion bei unkritischen
Pinchspannungen eingestellt. Die Pinchspannung entsteht durch anomale
Widerstandserhöhung in der Pinchsäule. Sie kann ein mehrfaches der
Ladespannung erreichen und die internen Isolationen der Anlage und
insbesondere die Isolatoren zerstören.
\par
Durch die Optimierung auf maximale Neutronenproduktion, ist eine gute
Schichtbildung, bei der Zündung am Isolator, sichergestellt, weil sie
eine Bedingung für eine hohe Neutronenproduktion ist. Die Parameter
sind in der folgenden Tabelle \vref{tab:injektionszeit:para}
zusammengefaßt.
%
\par
\begin{table}[H]
  \center
  \begin{tabular}{|l|c|}
  \hline
    Ladespannung U             & \wert{180}{kV}     \\
    Bankenergie E              & \wert{67}{kJ}      \\
    Füllgas\footnotemark       & Deuterium 2.7      \\
    Fülldruck p(D$_2$)         & \wert{4.8}{hPa}    \\
    Injektionsgas              & Neon 4.0           \\
    Injektionsdruck p(Ne)      & \ewert{5.0}{5}{Pa} \\
    Injektionszeit \teff       & \wert{2.0-17.5}{ms}\\
    Nummer                     & 11.354--11.438     \\
  \hline
  \end{tabular}
  \caption{Parameter der Entladungen in Tabelle \ref{tab:injektionszeit}}
  \label{tab:injektionszeit:para}
\end{table}
\footnotetext{Neben dem Gas wird auch die Reinheit des Gases laut
  Hersteller (Messer Griesheim GmbH, Krefeld, \url{http://www.spezialgase.de/})
  in der üblichen Notation angegeben. Z.B. bedeutet Deuterium 2.7: 99.7 Vol.-\%
  Deuterium + 0.3 Vol.-\% Verunreinigungen. Der Fülldruck wird mit einer
  mechanischen Druckdose gemessen. Diese wurde im August 1999 mit einer
  elektronischen Druckdose kalibriert.}
%
\par
Die Injektionszeit wurde von \wert{2.0}{ms} bis \wert{17.5}{ms}
variiert. Kürzere Injektionszeiten wurden nicht in die Tabelle
aufgenommen, weil die Intensität der Röntgenstrahlung dabei erheblich
reduziert wird und in den Bereich der Nachweisgrenze fällt.
\par
Bei längeren Injektionszeiten als \wert{17.5}{ms} nimmt die Intensität
der Röntgenstrahlung ebenfalls erheblich ab. Dabei führt die große
Anzahl von Injektionsgasteilchen zu einem kälteren Pinchplasma, weil
die Ionisation der Neonatome einen erheblichen Teil der thermischen
Energie des Plasmas und der kinetischen Energie der einlaufenden
Plasmaschicht verbraucht.
\par
Noch längere Zeiten ergeben eine quasistatische
Neon-Deuterium-Fül"-lung. Entladungen in statischen Füllungen mit
geringen (wenige Prozent) Beimischungen von schweren Gasen wie
z.B. Neon oder Argon ergeben keine effektive Pinchbildung, weil
die Schichtbildung am Isolator gestört ist \cite{kies:86}.
%
\par
\begin{table}[H]
  \center
  \begin{tabular}{|c|c|c|}
    \hline
    Inj.-Zeit    &  \multicolumn{2}{c|}{Anzahl Entladungen} \\
    \teff /ms    &  \makebox[2cm]{SCM} & \makebox[2cm]{MPM} \\
    \hline
    2.0    & 3 & 0 \\
    2.3    & 3 & 0 \\
    3.5    & 4 & 0 \\
    5.5    & 0 & 1 \\
    8.5    & 0 & 2 \\
    11.5   & 0 & 4 \\
    13.5   & 0 & 2 \\
    15.5   & 0 & 2 \\
    17.5   & 0 & 4 \\
  \hline
  \end{tabular}
  \caption{Variation der Injektionszeit von Neon}
  \label{tab:injektionszeit}
\end{table}
%
\par
In der Tabelle \vref{tab:injektionszeit} sind die Ergebnisse der
Meßreihe aufgeführt. Die Bilder der Rönt"-gen"-pin"-hole"-kame"-ra
wurden, wie oben beschrieben, für jede Entladung einzeln
betrachtet und dann die Entladungen dem SCM oder dem MPM
zugeordnet. Die Anzahl dieser Entladungen wurde gegen die
Injektionszeit tabelliert.
\par
Nicht aufgeführt werden hier die Entladungen ohne auswertbare
Rönt"-gen"-ems"-sion. Überwiegend handelt es sich dabei um
Reinigungsentladungen, bei denen der D$_2$-Druck auf
\wert{15}{hPa} angehoben wurde.
\par
Bei dieser Meßreihe ist eine deutliche Grenze zwischen dem SCM und dem
MPM zu erkennen. Zwischen \wert{3.5}{ms} und \wert{5.5}{ms} ändert die
Entladung ihren Modus. Weitere Änderungen des Modus sind durch eine
Variation der Injektionszeit nicht erreichbar.
\par
Es wurden weitere Meßreihen dieser Art, aber mit anderen
Parametern, durchgeführt. Alle zeigten in einem Bereich kurzer
Injektionszeiten den SCM, oberhalb davon einen Bereich des MPM.
Andere Übergänge wurden nie beobachtet. Die Lage des Grenzpunkts
hat sich dabei abhängig von den anderen Parametern des Experiments
gezeigt. Diese Abhängigkeiten wurden alle nach dem gleichen Schema
untersucht. Die Ergebnisse sind in diesem Kapitel \thesection\
zusammengefaßt.
\par
Nur selten zeigten die Messungen einen Grenzpunkt. In der Regel ergab
sich ein Übergangsbereich, der von den Schuß-zu-Schuß Schwankungen
bestimmt wird, dann ist der Grenzpunkt die Idealisierung dieses
Bereiches.
%
\beginsubsection{Isolator}
\label{iso:problem}
%
Einer der einflußreichsten Parameter ist der Isolator bzw. der
Zustand des Isolators. Der Isolator beeinflußt nicht nur die Lage
des Grenzpunkts, sondern auch die Breite des
Über"-gangs"-be"-rei"-ches. In der Tabelle
\vref{tab:injektionszeit} ist eine scharfe Trennung zwischen den
beiden Bereichen erkennbar. (Die Sichtbarkeit der Trennung in der
Tabelle ist natürlich auch durch die Wahl der Injektionszeiten
beeinflußbar.) Bei der Meßreihe mit den Parametern aus der Tabelle
\vref{tab:iso1:para} ergibt sich ein deutlich anderes Verhalten im
Bereich des Grenzpunkts.
%
\par
\begin{table}[H]
  \center
  \begin{tabular}{|l|c|}
  \hline
    Ladespannung U             & \wert{180}{kV}       \\
    Bankenergie E              & \wert{67}{kJ}        \\
    Füllgas                    & Deuterium 2.7        \\
    Fülldruck p(D$_2$)         & \wert{4.0}{hPa}      \\
    Injektionsgas              & Neon 4.0             \\
    Injektionsdruck p(Ne)      & \ewert{5.0}{5}{Pa}   \\
    Injektionszeit \teff       & \wert{0.5-16.5}{ms}  \\
    Nummer                     & 11.622--11.687       \\
  \hline
  \end{tabular}
  \caption{Parameter der Entladungen in Tabelle \ref{tab:iso1}}
  \label{tab:iso1:para}
\end{table}
%
\par
Die folgende Tabelle \vref{tab:iso1} zeigt die Ergebnisse dieser
Meßreihe. Durch den anderen Isolator entstanden auch bei
Entladungen mit \teff $= 0.5$ und $1.5$ ms auswertbare Bilder mit
der Rönt"-gen"-pin"-hole"-kame"-ra.
%
\par
\begin{table}[H]
  \center
  \begin{tabular}{|c|c|c|}
    \hline
    Inj.-Zeit    &  \multicolumn{2}{c|}{Anzahl Entladungen} \\
    \teff /ms    &  \makebox[2cm]{SCM} & \makebox[2cm]{MPM} \\
    \hline
    0.5    & 2 & 0 \\
    1.5    & 2 & 0 \\
    3.5    & 2 & 0 \\
    5.5    & 2 & 2 \\
    6.5    & 1 & 4 \\
    8.5    & 1 & 3 \\
    9.5    & 1 & 2 \\
    11.5   & 0 & 1 \\
    13.5   & 0 & 2 \\
    15.5   & 0 & 2 \\
    16.5   & 0 & 2 \\
  \hline
  \end{tabular}
  \caption{Variation der Injektionszeit von Neon}
  \label{tab:iso1}
\end{table}
%
\par
Der Bereich des SCM wird von dem Bereich des MPM durch einen breiten
Übergangsbereich getrennt. Aus der Tabelle \vref{tab:injektionszeit}
ergibt sich ein Übergangsbereich von unter \wert{2}{ms}, dagegen zeigt
die Tabelle \vref{tab:iso1} einen Übergangsbereich von ca.
\wert{6}{ms}.
\par
Dieser Unterschied ist auf die verschiedenen Isolatoren zurückzuführen,
weil alle anderen Parameter nahezu gleich sind. Variationen der
Parameter in diesem Bereich zeigen nicht diese Änderungen im Verhalten
der Entladung. Dieses ist auch an den Meßreihen in den folgenden
Abschnitten ablesbar.
\par
Der bei den Entladungen 11.622--11.687 eingesetzte Isolator zeigte
schon bei der Durchführung der Messungen Probleme. Nur jede zweite
Entladung führte zur effektiven Pinchbildung mit einer gut meßbaren
Neutronenproduktion. Dieses Verhalten ist untypisch für einen
\AlO-beschichteten Isolator und deutet auf Probleme mit der
Konditionierung der Isolatoroberfläche hin.
\par
\label{sec:uebergangsbereich} Die Schichtbildung auf der
Oberfläche des Isolators wird auch in der Gruppe der effektiven
Entladungen starke Unterschiede aufgezeigt haben. Dieses wird auch
durch die Neutronenproduktion bestätigt, die bei Entladungen
innerhalb dieser Meßreihe, bei gleichen Parametern, Unterschiede
bis zum Faktor 3 gezeigt hat. Je nach Schichtqualität
(Energiedichte, Dicke, Homogenität) kommt es innerhalb des
Über"-gangs"-be"-rei"-ches zu einer Ausbildung von Mikropinchen
oder einer stabilen Säule. Erst außerhalb des
Über"-gangs"-be"-rei"-ches entstehen bei guten und schlechten
Schichten immer die Mikropinche bzw. die stabilen Säulen.
\par
Es wurde schon erwähnt, daß nicht nur die Breite des
Über"-gangs"-be"-rei"-ches, sondern auch die Lage der
(idealisierten) Grenze vom Isolator bzw. vom Zustand des Isolators
beeinflußt wird. Meßreihen mit verschiedenen Isolatoren sind sehr
zeitaufwendig und kostspielig. Es wurde daher der normale
Austausch der Isolatoren benutzt, um diese Messung durchzuführen.
Es konnten nicht bei allen Isolatoren, die bei Messungen für diese
Arbeit eingesetzt wurden, diese Meßreihe durchgeführt werden.
Abgesehen von den Isolatoren und der angeschlossenen Diagnostik,
gab es in diesem Zeitraum keine Änderungen am Experiment, daher
ist es zulässig, die Werte über diesen langen Zeitraum zu
vergleichen.
\par
In der Tabelle \vref{tab:isolatoren:para} sind die Parameter der
Entladungen aufgeführt. Die Unterschiede bei der Bankenergie sind
vernachlässigbar, wie dem Abschnitt \vref{sec:grenze:bankenergie}
zu entnehmen ist. Die Unterschiede beim Fülldruck entstehen durch
die Optimierung auf maximale Neutronenausbeute, die eine gute
Schichtqualität anzeigt. Da dieses Optimum eine Ab"-hän"-gig"-keit
vom Isolator zeigt, kann die Änderung dieses Parameters unter die
Änderung des Isolators eingeordnet werden. Zudem ist der Einfluß
des Fülldrucks auf die Betriebsmodi gering, wie dem Abschnitt
\vref{sec:grenze:fuelldruck} zu entnehmen ist.
%
\par
\begin{table}[H]
  \center
  \begin{tabular}{|l|c|c|c|}
  \hline
              & \multicolumn{3}{c|}{Isolator} \\
              & Dezember'95 & August'97 & August'98 \\
  \hline
    Ladespannung U             & \multicolumn{3}{c|}{ \wert{180}{kV} }             \\
    Bankenergie E              & \wert{67}{kJ} & \wert{67}{kJ} & \wert{65-66}{kJ}  \\
    Füllgas                    & \multicolumn{3}{c|}{ Deuterium 2.7 }              \\
    Fülldruck p(D$_2$)         & \wert{4.8}{hPa} & \wert{4.8-5.0}{hPa} & \wert{10}{hPa} \\
    Injektionsgas              & \multicolumn{3}{c|}{ Neon 4.0 }                   \\
    Injektionsdruck            & \multicolumn{3}{c|}{ \ewert{5.0}{5}{Pa} }         \\
    Injektionszeit \teff       & \multicolumn{3}{c|}{\wert{0.5-10.5}{ms}}          \\
    Nummer                     & 10.528-10.586 & 11.312-11.377 & 12.090-12.253  \\
  \hline
  \end{tabular}
  \caption{Parameter der Entladungen in Tabelle \ref{tab:isolatoren}}
  \label{tab:isolatoren:para}
\end{table}
%
\par
Bei allen drei Isolatoren handelt es sich um \AlO-beschichtete
Quarzglasisolatoren. Bei allen drei Isolatoren wurde die
Beschichtung durch die Firma LWK\footnote{LWK-Plasmakeramik, 51617
Gummersbach, \url{http://www.plasmaceramic.de/}}, nach dem
gleichen Verfahren, durchgeführt. Der Isolator Dezember'95 stammt
aus einer anderen Charge als die Isolatoren August'97 und
August'98.
%
\par
\begin{table}[H]
  \center
  \begin{tabular}{|c|c|c|c|c|c|c|c|c|c|}
    \hline
                & \multicolumn{6}{c|}{Isolator} \\
    Inj.-zeit   & \multicolumn{2}{c|}{Dezember'95} & \multicolumn{2}{c|}{August'97} & \multicolumn{2}{c|}{August'98} \\
    \teff /ms   & \makebox[1cm]{SCM} & \makebox[1cm]{MPM} & \makebox[1cm]{SCM} & \makebox[1cm]{MPM} & \makebox[1cm]{SCM} & \makebox[1cm]{MPM} \\
    \hline
    0.5    & 9 & 1 &  1 & 0 &  2 & 0 \\
    2.5    & 5 & 3 &  2 & 0 & 10 & 0 \\
    4.5    & 1 & 9 &  2 & 0 &  2 & 0 \\
    6.5    & 0 & 9 &  1 & 1 &  7 & 0 \\
    8.5    & 0 & 4 &  0 & 3 &  1 & 1 \\
    9.5    & - & - &  0 & 1 &  0 & 2 \\
    10.5   & - & - &  0 & 2 &  0 & 2 \\
    \hline
  \end{tabular}
  \caption{Verschiedene (baugleiche) Isolatoren}
  \label{tab:isolatoren}
\end{table}
%
\par
Die Tabelle \vref{tab:isolatoren} zeigt die Ergebnisse dieser Meßreihe.
Die ungewöhnlich hohe Anzahl von Entladungen entstand dadurch, daß
diese Meßreihen auch für andere Untersuchungen benutzt wurden.
\par
Der Grenzbereich zwischen dem SCM und dem MPM ist bei allen drei
Isolatoren kleiner als \wert{4.0}{ms}. Es wurden nur gute Isolatoren in
die Tabelle aufgenommen, die keine großen Schwankungen in ihrem
Entladungsverhalten zeigten.
%
\par
\begin{table}[H]
  \center
  \begin{tabular}{|c|c|c|c|}
    \hline
              & \multicolumn{3}{c|}{Isolator} \\
              & \makebox[2cm]{Dezember'95} & \makebox[2cm]{August'97} & \makebox[2cm]{August'98} \\
    \hline
    \teff                  & \wert{2.5}{ms}           & \wert{6.5}{ms}          & \wert{8.5}{ms}         \\
    p(Ne)$\cdot$ \teff     & \wert{1250}{Pa$\cdot$s}  & \wert{3250}{Pa$\cdot$s} & \wert{4250}{Pa$\cdot$s} \\
    n(Ne)                  & \ewert{5.7}{19}{m$^{-1}$}& \ewert{1.5}{20}{m$^{-1}$} & \ewert{1.9}{20}{m$^{-1}$} \\
    \hline
  \end{tabular}
  \caption{Grenze bei verschiedenen Isolatoren}
  \label{tab:grenze:isolatoren}
\end{table}
%
\par
Die Ursachen für die stark unterschiedlichen Grenzpunkte müssen bei den
Isolatoren liegen. Diese große Verschiebung der Grenze zwischen den
beiden Modi verlangt nach einem besseren Verständnis für die
Unterschiede zwischen den Isolatoren.
\par
Seit der Inbetriebnahme von SPEED~2 begleitet das Problem der
Isolatoren die Experimente, zuletzt wurde es in der Arbeit
\cite{roewe:phd} beschrieben. Ähnliche Probleme gab es beim
Vorgängerexperiment SPEED~1, und es gibt diese Probleme auch beim
Nachfolger SPEED~3 \cite{raacke:diplom}.
\par
Die festgestellte Ab"-hän"-gig"-keit gäbe natürlich Anlaß,
Isolatoren aus anderen Materialien und mit anderen Beschichtungen
zu untersuchen. Dieses ist ein sehr zeit- und kostenintensives
Vorhaben. Bereits zur Optimierung von SPEED~2 als Neutronen- und
Röntgenquelle wurden verschiedene Isolatortypen getestet. Die
Tabelle \vref{tab:uebersicht:isolatoren} gibt einen Überblick über
die eingesetzten Isolatoren seit 1988.
%
\par
\begin{table}[H]
  \center
  \begin{tabular}{|l|c|c|c|c|}
    \hline
    Typ & Anzahl       & \multicolumn{3}{c|}{Anzahl Entladungen} \\
        & Exemplare    & mittlere & min. & max. \\
    \hline
    \AlO auf Quarzglas   & 28 & 225 &  2 & 1569 \\
    Duranglas Isolator   &  7 & 184 & 17 &  621 \\
    \AlO Isolator        & 16 &  47 &  2 &  415 \\
    \AlO auf Duranglas   &  3 &  40 &  1 &   37 \\
    Komposit-Isolator    &  4 &  18 &  1 &   38 \\
    Quarzglas Isolator   &  1 &  17 & 17 &   17 \\
    \hline
    Alle                 & 59 & 145 &  1 & 1569 \\
    \hline
  \end{tabular}
  \caption{Eingesetzte Isolatoren bei SPEED~2 im Zeitraum 1988--1999. Angegeben
    ist die mittlere, minimale und maximale Anzahl von Entladungen pro Isolator.}
  \label{tab:uebersicht:isolatoren}
\end{table}
%
\par
Das \AlO\ wird im Plasmaspritz-Verfahren auf die Quarzglas-Zylinder
aufgebracht. Die \AlO-beschichteten Duranglas-Isolatoren entstanden
durch Aufstreuen von \AlO-Pulver und anschließendem Glühen des
Glas-Zylinders. Die Komposit-Isolatoren waren Versuche mit geklebten
Glas, Metall und \AlO-Zylindern.
\par
Eine noch deutlichere Vorstellung von den Schwierigkeiten mit der
Lebensdauer der Isolatoren, ganz abgesehen von der Qualität der
Plasmaschicht und der Reproduzierbarkeit der Entladungen, gibt die
ausführliche Aufstellung im Kapitel \vref{iso:liste}.
\par
Die Vielzahl der unterschiedlichen Isolatortypen deutet auf die
Schwierigkeit des Problems hin. Selbst der Einsatz der
\AlO-beschichteten Isolatoren ist nicht unproblematisch.
\par
Die ursprünglich weiße \AlO-Oberfläche der Isolatoren verändert
sich bei den ersten Entladungen. Diese Entladungen müssen sehr
sorgfältig durchgeführt werden. Der Deuteriumfülldruck darf nur
langsam von typisch \wert{15}{hPa} auf den Arbeitsdruck reduziert
werden. Die Oberfläche des Isolators belegt sich in dieser Phase
mit kleinen, untereinander isolierten, Inseln aus Kupfer und
Aluminium.
\par
Aufgrund der Mühen bis zu den \AlO-beschichteten
Quarzglas-Isolatoren und des Fehlens von theoretischen Modellen
zum Oberflächenproblem, wurde in dieser Arbeit nicht versucht, die
Isolatorproblematik zu lösen. Hier wird nur der Ist-Zustand
dokumentiert.
\par
Für die Ursachen der Unterschiede zwischen scheinbar gleichen Isolatoren gibt
es verschiedene Möglichkeiten.
\par
Eine mögliche Ursache für die große Ab"-hän"-gig"-keit der
Entladungsmodi von den verwendeten Isolatoren, sind die
unterschiedlichen Bedingungen, bei denen die Isolatoren
konditioniert und benutzt wurden. Die älteren Isolatoren wurden
überwiegend mit Argon-Injektion im MPM benutzt. Die neueren
Isolatoren wurden überwiegend mit Neon-Injektion im SCM benutzt.
Daher kann die Konditionierung der Isolatoroberfläche verschieden
sein. Eine Ab"-hän"-gig"-keit von der Konditionierung des
Isolators muß bestehen, weil die entstehenden Schichten von der
Konditionierung abhängig sind. Wie die Entladungen in den
verschiedenen Modi den Isolator konditionieren, ob es dabei
Unterschiede gibt, ist noch nicht untersucht.
\par
Eine weitere Möglichkeit sind kleine, nicht dokumentierte,
Änderungen im Herstellungsprozeß der Isolatoren, insbesondere bei
der Beschichtung. Die \AlO-Schicht wird bis zur Dicke von
\wert{0.2}{mm} in mehreren Lagen auf die sandgestrahlte
Quarzglasoberfläche aufgespritzt. So können leicht Unterschiede in
der Dicke der einzelnen Schichten und der Rauhigkeit der Schichten
auftreten.
\par
Im Wesentlichen bleibt nach diesem Abschnitt die Frage der
Ab"-hän"-gig"-keit vom Isolator offen. Es ist nur gesichert, daß
es diese Ab"-hän"-gig"-keit gibt und daß die Isolatoren noch immer
das Hauptproblem beim Betrieb des SPEED~2 darstellen.
\par
Wichtig für die anderen Meßreihen ist die Ab"-hän"-gig"-keit von
der Konditionierung des Isolators. Werden Meßreihen durchgeführt,
so kann sich die Konditionierung langsam verändern. Damit die
Variation des gewünschten Parameters gemessen wird und nicht die
Änderungen in der Konditionierung, wurden die Parameter nicht
kontinuierlich von Entladung zu Entladung von kleinen Werten zu
großen Werten geändert, sondern es wurde zwischen den Werten
zufällig gesprungen.
%
\beginsubsection{Injektionsdruck}
%
Neben dem nicht einstellbaren Einfluß des Isolators gibt es eine Reihe
weiterer gut einstellbarer Parameter. Die Injektionszeit ist nur eine
Möglichkeit, die injizierte Gasmenge zu ändern. Die Gasmenge kann auch
über den Injektionsdruck, also dem Gasdruck des Injektionsgases vor dem
Öffnen des schnellen Magnetventils, beeinflußt werden. Der wesentliche
Vorteil bei der Variation des Drucks ist die gute Proportionalität
zwischen der Injektionsgasmenge in der Pinchsäule und dem
Injektionsdruck. Nachteilig ist die wesentlich schlechtere
Einstellmöglichkeit des Drucks am Druckreduzierventil. Daher wurde der
Injektionsdruck nicht als Hauptparameter für die Meßreihen benutzt.
\par
In der Tabelle \vref{tab:injektionsdruecke:para} sind die Parameter der
Meßreihe aufgeführt, mit der dieser Zusammenhang untersucht wurde.
%
\par
\begin{table}[H]
  \center
  \begin{tabular}{|l|c|c|}
  \hline
      & \multicolumn{2}{c|}{Injektionsdruck p(Ar)} \\
      & \ewert{9.5}{5}{Pa} & \ewert{5.0}{5}{Pa}    \\
  \hline
    Ladespannung U             & \multicolumn{2}{c|}{ \wert{180}{kV} }    \\
    Bankenergie E              & \multicolumn{2}{c|}{ \wert{57}{kJ} }     \\
    Füllgas                    & \multicolumn{2}{c|}{ Deuterium 2.7 }     \\
    Fülldruck p(D$_2$)         & \multicolumn{2}{c|}{ \wert{9.5}{hPa} }   \\
    Injektionsgas              & \multicolumn{2}{c|}{ Argon 4.6 }         \\
    Injektionsdruck p(Ar)      & \ewert{5.0}{5}{Pa} & \ewert{9.5}{5}{Pa}  \\
    Injektionszeit \teff       & \multicolumn{2}{c|}{ \wert{0.5-9.5}{ms} }\\
    Nummer                     & 12.166--12.180 & 12.192--12.197          \\
  \hline
  \end{tabular}
  \caption{Parameter der Entladungen in Tabelle \ref{tab:injektionsdruecke}}
  \label{tab:injektionsdruecke:para}
\end{table}
%
\par
Der Injektionsdruck wurde von seinem üblichen Wert so weit erhöht, wie
es das vorhandene Druckreduzierventil erlaubte. Die Messungen mit den
verschiedenen Injektionsdrücken wurden nacheinander durchgeführt, weil
das genaue Einstellen des Injektionsdrucks aufwendig ist. Innerhalb
dieser beiden Teilmeßreihen wurden die verschiedenen Injektionszeiten
in zufälliger Reihenfolge eingestellt.
\par
Die gefundenen Entladungsformen sind in der folgende Tabelle
\vref{tab:injektionsdruecke} aufgeführt.
%
\par
\begin{table}[H]
  \center
  \begin{tabular}{|c|c|c|c|c|c|c|c|}
    \hline
              & \multicolumn{4}{c|}{Injektionsdruck p(Ar)} \\
    Inj.-zeit & \multicolumn{2}{c|}{ \ewert{5.0}{5}{Pa}} & \multicolumn{2}{c|}{ \ewert{9.5}{5}{Pa} } \\
    \teff /ms & \makebox[2cm]{SCM} & \makebox[2cm]{MPM} & \makebox[2cm]{SCM} & \makebox[2cm]{MPM} \\
    \hline
     0.5    &  - & - &  2 & 0 \\
     1.5    &  2 & 0 &  0 & 2 \\
     2.5    &  - & - &  0 & 2 \\
     3.4    &  0 & 2 &  - & - \\
     5.5    &  0 & 2 &  - & - \\
     7.5    &  0 & 2 &  - & - \\
     9.5    &  0 & 2 &  - & - \\
    \hline
  \end{tabular}
  \caption{Variation des Injektionsdrucks von Argon}
  \label{tab:injektionsdruecke}
\end{table}
%
\par
Da die Grenze zwischen dem MPM und dem SCM in dieser Meßreihe sehr scharf ist,
kommt diese Meßreihe mit wenigen Entladungen aus. Für die Lage der Grenzpunkte
ergeben sich Werte in der Tabelle \vref{tab:injektionsdruecke:result}.
%
\par
\begin{table}[H]
  \center
  \begin{tabular}{|c|c|c|}
    \hline
                           & \multicolumn{2}{c|}{Injektionsdruck p(Ar)} \\
                           & \ewert{5.0}{5}{Pa} & \ewert{9.5}{5}{Pa}    \\
    \hline
    \teff                  & \wert{3.5}{ms}           & \wert{1.0}{ms}          \\
    p(Ar)$\cdot$ \teff     & \wert{1750}{Pa$\cdot$s}  & \wert{950}{Pa$\cdot$s}  \\
    n(Ar)                  & \ewert{6.8}{19}{m$^{-1}$}& \ewert{3.7}{19}{m$^{-1}$} \\
    \hline
  \end{tabular}
  \caption{Grenze bei verschiedenen Injektionsdrücken von Argon}
  \label{tab:injektionsdruecke:result}
\end{table}
%
\par
Es zeigt sich also, daß der Entladungsmodus im wesentlichen von der
Anzahl der Injektionsgasteilchen in der Entladung abhängig ist und
nicht von der Injektionszeit oder dem Injektionsdruck alleine. Das
Resultat mit nur zwei verschiedenen Drücken, ist noch nicht
aussagekräftig genug. Daher wurde auch eine Meßreihe mit verringertem
Injektionsdruck durchgeführt. Aus organisatorischen Gründen konnte
diese Meßreihe aber nicht direkt anschließend durchgeführt werden.
Damit keine Langzeitdrift das Ergebnis verfälscht, wurde auch die
Meßreihe beim üblichen Injektionsdruck wiederholt. So können die
Grenzwerte direkt miteinander in einem kurzen Experimentierzeitraum
verglichen werden.
\par
Die eingestellten Parameter für diese Meßreihe können der Tabelle
\vref{tab:injektionsdruecke:2:para} entnommen werden. Sie entsprechen
im wesentlichen der ersten Meßreihe. Die Bankenergie ist geringfügig
höher, weil zwischenzeitlich defekte Module instand gesetzt wurden. Im
Gegenzug mußte der Fülldruck um \wert{0.4}{hPa} angehoben werden, damit
die Pinchspannungen nicht zu groß wurden.
%
\par
\begin{table}[H]
  \center
  \begin{tabular}{|l|c|c|}
  \hline
      & \multicolumn{2}{c|}{Injektionsdruck p(Ar)} \\
      & \ewert{5.0}{5}{Pa} & \ewert{3.5}{5}{Pa}    \\
  \hline
    Ladespannung U             & \multicolumn{2}{c|}{ \wert{180}{kV} }      \\
    Bankenergie E              & \multicolumn{2}{c|}{ \wert{66}{kJ} }       \\
    Füllgas                    & \multicolumn{2}{c|}{ Deuterium 2.7 }       \\
    Fülldruck p(D$_2$)         & \multicolumn{2}{c|}{ \wert{10.0}{hPa} }     \\
    Injektionsgas              & \multicolumn{2}{c|}{ Argon 4.6 }           \\
    Injektionsdruck p(Ar)      & \ewert{5.0}{5}{Pa} & \ewert{3.5}{5}{Pa}    \\
    Injektionszeit \teff       & \multicolumn{2}{c|}{ \wert{0.5-14.5}{ms} } \\
    Nummer                     & 12.255--12.267 & 12.269--12.297            \\
  \hline
  \end{tabular}
  \caption{Parameter der Entladungen in Tabelle \ref{tab:injektionsdruecke:2}}
  \label{tab:injektionsdruecke:2:para}
\end{table}
%
\par
Der Injektionsdruck von \ewert{3.5}{5}{Pa} wurde gewählt, weil bei
kleineren Drücken die Dichtigkeit des Magnetventils nachgelassen hat.
Das Magnetventil ist immer durch Kupferpartikel aus der
Entladungskammer gefährdet. Diese Partikel setzen sich auf die
Dichtfläche und verschlechtern die Dichtigkeit. Nur bei ausreichendem
Injektionsdruck, ist das Ventil dann noch hinreichend dicht. Die
Kontrolle des Zustands des Ventils erfolgt nach jeder Entladung
indirekt, durch den erreichten Enddruck beim Evakuieren des
Rezipienten.
\par
Die Auswertung der Bilder der Rönt"-gen"-pin"-hole"-kame"-ra ergab
die folgenden Tabelle \vref{tab:injektionsdruecke:2}.
%
\par
\begin{table}[H]
  \center
  \begin{tabular}{|c|c|c|c|c|}
    \hline
              & \multicolumn{4}{c|}{Injektionsdruck p(Ar)} \\
    Inj.-zeit & \multicolumn{2}{c|}{ \ewert{5.0}{5}{Pa}} & \multicolumn{2}{c|}{ \ewert{3.5}{5}{Pa} } \\
    \teff /ms & \makebox[2cm]{SCM} & \makebox[2cm]{MPM} & \makebox[2cm]{SCM} & \makebox[2cm]{MPM} \\
    \hline
     0.5    & 2 & 0 &  2 & 0 \\
     2.5    & 2 & 0 &  2 & 0 \\
     4.5    & 0 & 2 &  2 & 0 \\
     7.5    & 0 & 2 &  1 & 1 \\
     9.5    & - & - &  0 & 2 \\
    14.5    & 0 & 2 &  - & - \\
    \hline
  \end{tabular}
  \caption{Variation des Injektionsdrucks von Argon}
  \label{tab:injektionsdruecke:2}
\end{table}
%
\par
Zusammengefaßt ergibt sich die Tabelle \vref{tab:injektionsdrucke:2:result} mit
den Grenzpunkten.
%
\par
\begin{table}[H]
  \center
  \begin{tabular}{|c|c|c|}
    \hline
                           & \multicolumn{2}{c|}{Injektionsdruck p(Ar)} \\
                           & \ewert{5.0}{5}{Pa} &  \ewert{3.5}{5}{Pa}   \\
    \hline
    \teff                  & \wert{3.5}{ms}           & \wert{7.5}{ms}           \\
    p(Ar)$\cdot$ \teff     & \wert{1750}{Pa$\cdot$s}  & \wert{2625}{Pa$\cdot$s}  \\
    n(Ar)                  & \ewert{6.8}{19}{m$^{-1}$}& \ewert{1.0}{19}{m$^{-1}$}  \\
    \hline
  \end{tabular}
  \caption{Grenze bei verschiedenen Injektionsdrücken von Argon}
  \label{tab:injektionsdrucke:2:result}
\end{table}
%
\par
Die Reproduzierung des Grenzpunkts beim Injektionsdruck von
\ewert{5.0}{5}{Pa} ist gut sichtbar. Es gab also (noch) keine
Langzeitschwankung bei diesem Isolator. Es folgt auch, daß
Meßreihen mit über 100 Entladungen sinnvolle Ergebnisse zeigen
können, wenn das Verhalten des Isolators sorgfältig beobachtet
wird.
\par
Die Vermutung liegt nahe, daß die Grenze zwischen dem MPM und dem SCM
von der Injektionsgasteilchendichte n(Inj.) in der Pinchsäule abhängig
ist. Die Liniendichte dürfte in erster Näherung proportional sein zu
p(Inj.)$\cdot \tau_{\rm eff}$. Dieser Wert ist in den Tabellen auch
aufgeführt. Es ist sichtbar, daß dieser Wert nicht die Änderung der
Injektionszeit im gleichen Maße mitmacht. Die Injektionszeit
vergrößerte sich um den Faktor 7.5; der Wert von p(Inj.)$\cdot$\teff
vergrößerte sich um den Faktor 2.7. Der Wert des Produkts, damit auch
die Liniendichte n(Ar), ist deutlich schwächer abhängig von dem
Injektionsdruck.
\par
Aufgrund der vorliegenden Daten, kann nicht endgültig geklärt
werden, ob auf der Grenze die Liniendichte n(Inj.) konstant ist.
Die Hauptunsicherheit bei der Proportionalität zwischen n(Inj.)
und dem Produkt p(Inj.)$\cdot$\teff liegt in der
Ab"-hän"-gig"-keit von der Injektionszeit $\tau_{\rm eff}$. Die
Probleme bei der Bestimmung der Teilchendichte und der Ausdehnung
des Injektionsgasstroms wurden bereits in Kapitel
\vref{sec:plasmafokus} erläutert.
\par
Die unter vielen Vorbehalten abgeschätzte Liniendichte des
Injektionsgases in der Pinchsäule ist in allen Resultat-Tabellen in der
Zeile n(Ar) bzw. n(Ne) aufgeführt.
%
\beginsubsection{Ladespannung}
\label{sec:grenze:ladespannungen}
%
Nachdem die beiden Parameter des Gasinjektions-Systems behandelt
wurden, werden nun die Parameter des Treibers: Ladespannung U und
Bankenergie E untersucht.
\par
Bei der Variation der Ladespannung U wird automatisch die Bankenergie
nach $ E = \frac{1}{2} C \cdot U^2 $ verändert, wenn die Kapazität C
beibehalten wird. Bei SPEED~2 besteht die Möglichkeit die Kapazität C,
durch die Anzahl der eingeschalteten Module, zu ändern. Diese
Möglichkeit wurde aber nicht benutzt, daher hat sich die Bankenergie
bei den Meßreihen entsprechend der obigen Formel verhalten.
\par
Es wurden drei verschiedene Ladespannungen benutzt. Diese
Spannungswerte und die weiteren Parameter der Meßreihe sind in der
Tabelle \vref{tab:ladespannungen:para} zusammengefaßt.
%
\par
\begin{table}[H]
  \center
  \begin{tabular}{|l|c|c|c|}
  \hline
    Ladespannung U             & \wert{180}{kV} & \wert{200}{kV} & \wert{220}{kV}  \\
    Bankenergie E              & \wert{66}{kJ}  & \wert{81}{kJ}  & \wert{98}{kJ}   \\
    Füllgas                    & \multicolumn{3}{c|}{ Deuterium 2.7 }         \\
    Fülldruck p(D$_2$)         & \multicolumn{3}{c|}{\wert{4.8-5.4}{hPa} }    \\
    Injektionsgas              & \multicolumn{3}{c|}{ Neon 4.0 }              \\
    Injektionsdruck p(Ne)      & \multicolumn{3}{c|}{\ewert{5.0}{5}{Pa}}      \\
    Injektionszeit \teff       & \multicolumn{3}{c|}{\wert{0.5-16.5}{ms}}     \\
    Nummer                     & \multicolumn{3}{c|}{11.355--11.573}          \\
  \hline
  \end{tabular}
  \caption{Parameter der Entladungen in Tabelle \ref{tab:ladespannungen}}
  \label{tab:ladespannungen:para}
\end{table}
%
\par
Die Teilmeßreihen mit verschiedenen Ladespannungen wurden nacheinander
durchgeführt, innerhalb dieser Teilmeßreihen wurde die Injektionszeit
in zufälliger Reihenfolge eingestellt.
\par
In der Meßreihe \wert{U = 200}{kV} ist ein Strang von SPEED~2
ausgefallen, daher wurden der erste Teil der Messungen mit \wert{E
= 83}{kJ} und der zweite Teil der Messungen mit \wert{E = 81}{kJ}
durchgeführt. Diese Änderung um weniger als 3\% der Bankenergie
beeinflußt die Ergebnisse nur unwesentlich. Bei den Messungen mit
\wert{U = 220}{kV} wurden die Anzahl der Messungen reduziert, um
vor dem möglichen Ausfall weiterer Stränge die Meßreihe
abzuschließen. Dennoch wurde die Meßreihe durch die Zerstörung des
Isolators vorzeitig beendet. Das Fortsetzen einer Meßreihe mit
einem neuen Isolator ist nicht möglich (Abschnitt
\vref{iso:problem}).
%
\par
\begin{table}[H]
  \center
  \begin{tabular}{|c|c|c|c|c|c|c|c|c|c|}
    \hline
    Inj.-zeit    & \multicolumn{2}{c|}{\wert{U = 180}{kV}} & \multicolumn{2}{c|}{\wert{U = 200}{kV}} & \multicolumn{2}{c|}{\wert{U = 220}{kV}} \\
    \teff /ms    & \makebox[1cm]{SCM} & \makebox[1cm]{MPM} & \makebox[1cm]{SCM} & \makebox[1cm]{MPM} & \makebox[1cm]{SCM} & \makebox[1cm]{MPM} \\
    \hline
    0.5    & 1 & 0 &  2 & 0 &  - & - \\
    1.5    & - & - &  2 & 0 &  1 & 0 \\
    3.5    & 4 & 2 &  2 & 0 &  1 & 1 \\
    5.5    & 1 & 0 &  2 & 2 &  - & - \\
    6.5    & 1 & 1 &  1 & 4 &  0 & 2 \\
    8.5    & 0 & 3 &  1 & 3 &  0 & 1 \\
    9.5    & 0 & 1 &  1 & 2 &  - & - \\
    10.5   & 0 & 2 &  - & - &  - & - \\
    11.5   & 0 & 2 &  0 & 1 &  0 & 2 \\
    12.5   & 0 & 2 &  - & - &  - & - \\
    13.5   & 0 & 2 &  0 & 2 &  0 & 1 \\
    15.5   & 0 & 2 &  0 & 2 &  0 & 2 \\
    16.5   & 0 & 1 &  0 & 2 &  0 & 1 \\
    \hline
  \end{tabular}
  \caption{Variation der Ladespannung mit Neon-Injektion}
  \label{tab:ladespannungen}
\end{table}
%
\par
Auffällig bei der Meßreihe ist der Übergangsbereich mit einer
Breite von ca. \wert{3.0}{ms} ist. Er erreicht nicht die Breite
des Über"-gangs"-be"-rei"-ches bei den schlechtesten Isolatoren,
aber es ist ein Hinweis auf mögliche Probleme beim Isolator. Wie
schon erwähnt, endete diese Meßreihe mit der Zerstörung des
Isolators. Der Durchschlag im Fußbereich des Isolators kann aber
nicht direkt in Zusammenhang mit einer nachlassenden
Entladungsqualität gebracht werden. Ein indirekter Zusammenhang,
über die Reduzierung des D$_2$-Fülldrucks nach Entladungen mit
geringer Neutronenproduktion, gefolgt von einer sehr effizienten
Entladung mit hoher Pinchspannung und des damit ausgelösten
Durchschlags, erscheint plausibel.
\par
Die aus den Messungen bestimmten Grenzpunkte sind in der Tabelle
\vref{tab:ladespannungen:result} zusammengefaßt.
%
\par
\begin{table}[H]
  \center
  \begin{tabular}{|c|c|c|c|}
    \hline
                           & \multicolumn{3}{c|}{Ladespannung U} \\
                           & \wert{180}{kV}  &  \wert{200}{kV} & \wert{220}{kV}  \\
    \hline
    \teff                  & \wert{5.5}{ms}          & \wert{7.5}{ms}          & \wert{3.5}{ms}          \\
    p(Ne)$\cdot$ \teff     & \wert{2750}{Pa$\cdot$s} & \wert{3750}{Pa$\cdot$s} & \wert{1750}{Pa$\cdot$s} \\
    n(Ne)                  & \ewert{1.2}{19}{m$^{-1}$} & \ewert{1.7}{20}{m$^{-1}$} & \ewert{8.0}{19}{m$^{-1}$} \\
    \hline
  \end{tabular}
  \caption{Grenze bei verschiedenen Ladespannungen mit Neon-Injektion}
  \label{tab:ladespannungen:result}
\end{table}
%
\par
Das Ergebnis dieser Meßreihe ist widersprüchlich. Bei der Erhöhung der
Ladespannung und damit der Bankenergie von \wert{U = 180}{kV} bzw.
\wert{E = 66}{kJ} auf \wert{U = 200}{kV} bzw. \wert{E = 81}{kJ}
vergrößert sich der Bereich des SCM. Bei der Erhöhung von \wert{U =
180}{kV} auf \wert{U = 220}{kV} verkleinert sich der Bereich des SCM.
Ein Maximum der Grenze bei einer Ladespannung zwischen \wert{U =
180}{kV} und \wert{U = 220}{kV} erscheint sehr unwahrscheinlich.
\par
Eine Möglichkeit, ein Ergebnis aus diesen Messungen zu gewinnen,
ist das Streichen der Teilmeßreihe mit einer Ladespannung \wert{U
= 220}{kV}. Eine gute Begründung dafür ist die Zerstörung des
Isolators bei den Messungen. So könnte die gesamte Reihe
entfallen, weil die erhaltenen Daten schon von dem Ausfall
beeinflußt wurden. Auch zeigen die Isolatoren immer einen
Arbeitsbereich bzgl. der Ladespannug, in Ab"-hän"-gig"-keit ihrer
Konditionierung. Die Ladespannung von \wert{220}{kV} könnte
bereits an der Grenze für diesen Isolator gelegen haben. Aufgrund
des Ausfalls konnten Messungen mit weiter vergrößerter
Ladespannung nicht durchgeführt werden. Als Ergebnis bleibt dann,
daß die Erhöhung der Ladespannung U zur Vergrößerung des Bereichs
für den SCM führt.
\par
Eine andere Möglichkeit ist das Streichen der gesamten Meßreihe zur
Ladespannung, weil in der ersten Teilmeßreihe der Übergangsbereich
schon vergrößert war. Als Ergebnis bleibt dann festzustellen, daß
Meßreihen über so viele Entladungen in der Regel schwer durchführbar
sind, weil Alterungsprozesse des Isolators die Ergebnisse beeinflussen.
%
\beginsubsection{Bankenergie}
\label{sec:grenze:bankenergie}
%
Die Änderungen in der Ladespannung U führten durch konstant gehaltene
Kapazität C auch zu einer Änderung der Bankenergie E. Der modulare
Aufbau von SPEED~2 erlaubt es, die Energie E bei konstanter Spannung U
durch Änderung der Gesamtkapazität C zu variieren. Die genauen
Parameter dieser Meßreihe sind in der Tabelle
\vref{tab:bankenergie:para} aufgeführt.
%
\par
\begin{table}[H]
  \center
  \begin{tabular}{|l|c|c|}
  \hline
      & \multicolumn{2}{c|}{Bankenergie E} \\
      & \wert{57}{kJ} & \wert{66}{kJ}  \\
  \hline
    Ladespannung U             & \multicolumn{2}{c|}{ \wert{180}{kV} }      \\
    Bankenergie E              & \wert{57}{kJ}    & \wert{66}{kJ}           \\
    Füllgas                    & \multicolumn{2}{c|}{ Deuterium 2.7 }       \\
    Fülldruck p(D$_2$)         & \wert{9.5}{hPa}  & \wert{10.0}{hPa}        \\
    Injektionsgas              & \multicolumn{2}{c|}{ Argon 4.6 }           \\
    Injektionsdruck p(Ar)      & \multicolumn{2}{c|}{ \ewert{5.0}{5}{Pa} }  \\
    Injektionszeit \teff       & \multicolumn{2}{c|}{ \wert{0.5-16.5}{ms} } \\
    Nummer                     & 12.164--12.197 & 12.255--12.290            \\
  \hline
  \end{tabular}
  \caption{Parameter der Entladungen in Tabelle \ref{tab:bankenergie}}
  \label{tab:bankenergie:para}
\end{table}
%
\par
Es wurden Messungen mit 30, 34 und 39 Module entsprechend $E =$
\wert{50}{kJ}, $E =$ \wert{57}{kJ} und $E =$ \wert{66}{kJ}
durchgeführt. Beim Abschalten der Module wurde darauf geachtet,
daß die Rotationssymmetrie der Anlage möglichst wenig gestört
wird. Die maximale Anzahl der Module beträgt 39, weil zum
Zeitpunkt der Messung ein Modul defekt war.
\par
Die Messungen mit \wert{E = 50}{kJ} sind in den Tabellen nicht
aufgeführt, weil keine effizienten Pinchbildung möglich war. Es ist
eine Eigenschaft der Isolatoren, nur innerhalb eines Energiebereichs
effiziente Entladungen zu ermöglichen.
\par
Die Ergebnisse der Messungen mit 34 und 39 Module sind in der Tabelle
\vref{tab:bankenergie} zusammengefaßt.
%
\par
\begin{table}[H]
  \center
  \begin{tabular}{|c|c|c|c|c|}
    \hline
              & \multicolumn{4}{c|}{Bankenergie E} \\
    Inj.-zeit & \multicolumn{2}{c|}{ \wert{57}{kJ} } & \multicolumn{2}{c|}{ \wert{66}{kJ} } \\
    \teff /ms & \makebox[2cm]{SCM} & \makebox[2cm]{MPM} & \makebox[2cm]{SCM} & \makebox[2cm]{MPM} \\
    \hline
     0.5    & 2 & 0 &  4 & 0 \\
     1.0    & 2 & 0 &  - & - \\
     1.5    & 5 & 0 &  - & - \\
     2.5    & 2 & 2 &  4 & 0 \\
     4.5    & 0 & 2 &  2 & 2 \\
     7.5    & 0 & 2 &  1 & 3 \\
     9.5    & - & - &  0 & 2 \\
    10.5    & 0 & 2 &  - & - \\
    12.5    & 0 & 2 &  - & - \\
    14.5    & 0 & 2 &  - & - \\
    16.5    & 0 & 2 &  0 & 2 \\
    \hline
  \end{tabular}
  \caption{Variation der Bankenergie mit Argon-Injektion}
  \label{tab:bankenergie}
\end{table}
%
\par
Die Teilmeßreihen mit unterschiedlicher Bankenergie E entstanden
nacheinander, weil das Ab- und Anschalten von einem Modul einige Zeit
in Anspruch nimmt. Bei den Messungen mit verschiedenen Injektionszeiten
wurde dann wieder eine zufällige Reihenfolge gewählt.
\par
Nach den schlechten Erfahrungen bei den Messungen
\vref{sec:grenze:ladespannungen} wurde hier bei den Messungen mit
\wert{E = 66}{kJ} im wesentlichen nur der Bereich um den Grenzwert
untersucht, um die Anzahl der benötigten Entladungen gering zu halten.
%
\par
\begin{table}[H]
  \center
  \begin{tabular}{|c|c|c|}
    \hline
                           & \multicolumn{2}{c|}{Bankenergie E}\\
                           & \wert{57}{kJ} &  \wert{66}{kJ}    \\
    \hline
    \teff                  & \wert{3.5}{ms}           & \wert{4.5}{ms}           \\
    p(Ar)$\cdot$ \teff     & \wert{1750}{Pa$\cdot$s}  & \wert{2250}{Pa$\cdot$s}  \\
    n(Ar)                  & \ewert{6.8}{19}{m$^{-1}$}  & \ewert{8.7}{19}{m$^{-1}$}\\
    \hline
  \end{tabular}
  \caption{Grenze bei unterschiedlicher Bankenergie}
  \label{tab:bankenergie:result}
\end{table}
%
\par
Die beiden Grenzwerte sind in der Tabelle
\vref{tab:bankenergie:result} angegeben.
\par
Mit Erhöhung der Bankenergie, ist eine Verschiebung des Grenzpunkts zu
größeren Zeiten, d.h. zu mehr Injektionsgas im Pinch, sichtbar. Dieses
Ergebnis stimmt überein mit den ersten beiden Teilmessungen, bei der
Variation der Ladespannung (Kapitel \ref{sec:grenze:ladespannungen}).
So wird insbesondere die unsichere Messung bezüglich der Ladespannung
in ihrer Aussage bekräftigt.
\par
Zur Vollständigkeit sei hier angeführt, daß der Isolator 28 Entladungen
nach diesen Meßreihen ausgetauscht werden mußte. Stücke der
\AlO-Beschichtung hatten sich abgelöst, danach waren mit dem
beschädigten Isolator keine effizienten Entladungen mehr durchführbar.
Auswirkungen auf die hier vorgestellte Meßreihe sind eher
unwahrscheinlich, da zu dieser Zeit keine Schäden an der Beschichtung
sichtbar waren.
%
\beginsubsection{Fülldruck}
\label{sec:grenze:fuelldruck}
%
Bei allen bisher betrachteten Meßreihen wurde der Fülldruck so gewählt,
daß die Neutronenproduktion, bei einer für die Anlage ungefährlichen
Pinchspannung, maximal ist. Wenn möglich, wurde der Fülldruck nur
einmal vor Beginn der Meßreihen ausgewählt. Der Einfluß des
Deuteriumfülldrucks auf die Entladungsmodi wird jetzt untersucht.
\par
Der Fülldruck läßt sich nur in einem kleinen Bereich einstellen. Einige
Isolatoren erlauben effektive Entladungen nur in einem Fülldruckbereich mit
einer Breite von unter \wert{2}{hPa}. Bei der in diesem Abschnitt vorgestellten
Meßreihe wurde ein Isolator verwendet, der im Bereich von \wert{11.1}{hPa} bis
\wert{9.5}{hPa} effektive Entladungen ermöglicht. Fülldrücke im Bereich von
\wert{9.5}{hPa} und darunter führen zu sehr hohen Pinchspannungen, die wiederum
zu einer Zerstörung des Isolators oder sogar der Isolation im Hauptkollektor
von SPEED~2 führen können. Daher wurde der Bereich kleiner Fülldrücke bei der
Meßreihe gemieden.
\par
Die Parameter der Entladungen sind in der bekannten Form in Tabelle
\vref{tab:fuelldruck:para} zusammengestellt.
%
\par
\begin{table}[H]
  \center
  \begin{tabular}{|l|c|}
  \hline
    Ladespannung U             & \wert{180}{kV}       \\
    Bankenergie E              & \wert{57}{kJ}        \\
    Füllgas                    & Deuterium 2.7        \\
    Fülldruck p(D$_2$)         & \wert{9.5-11.1}{hPa} \\
    Injektionsgas              & Argon 4.6            \\
    Injektionsdruck p(Ar)      & \ewert{5.0}{5}{Pa}   \\
    Injektionszeit \teff       & \wert{2.5}{ms}       \\
    Nummer                     & 12.205--12.223       \\
  \hline
  \end{tabular}
  \caption{Parameter der Entladungen in Tabelle \ref{tab:fuelldruck}}
  \label{tab:fuelldruck:para}
\end{table}
%
\par
Da ein nicht optimal eingestellter Fülldruck, eine Reduzierung der
Neutronenproduktion und eine Reduzierung der
Rönt"-gen"-strah"-lungs"-in"-ten"-si"-tät zur Folge hat, läßt sich
diese Ab"-hän"-gig"-keit kaum sinnvoll ausnutzen. Zur Untersuchung
der Ab"-hän"-gig"-keit der Grenze zwischen den beiden
Entladungsmodi wurde daher eine Variante gewählt, die weniger
Entladungen benötigt, aber dafür etwas weniger aussagekräftig ist.
\par
Die Injektionszeit wurde so eingestellt, daß die Entladung bei
optimalen Fülldruck im SCM an der Grenze zum MPM liegt. Wird der
Fülldruck erhöht, ist ein Wechsel zum MPM zu erwarten, weil die
Energiedichte in der Plasmaschicht geringer wird.
\par
Die experimentellen Ergebnisse sind in der Tabelle \vref{tab:fuelldruck}
zusammengefaßt.
%
\par
\begin{table}[H]
  \center
  \begin{tabular}{|c|c|c|}
    \hline
    Fülldruck   & \multicolumn{2}{c|}{Anzahl Entladungen} \\
    p($D_2$)/hPa &  \makebox[2cm]{SCM} & \makebox[2cm]{MPM} \\
    \hline
    9.5    & 4 & 0 \\
    9.9    & 2 & 0 \\
    10.3   & 2 & 0 \\
    10.7   & 2 & 0 \\
    11.1   & 2 & 0 \\
  \hline
  \end{tabular}
  \caption{Variation des D$_2$-Fülldrucks mit Argon-Injektion}
  \label{tab:fuelldruck}
\end{table}
%
\par
Ein Wechsel des Entladungsmodus aus dem SCM ist nicht erfolgt. Die
Ab"-hän"-gig"-keit vom Deuteriumfülldruck ist also so klein, daß
in dem zur Verfügung stehenden Variationsbereich kein Wechsel des
Modus erreichbar ist.
\par
Um die Aussage abzurunden, fehlt noch eine Meßreihe, in der die
Entladungen, bei entsprechend eingestellter Injektionszeit und bei
Variation des Fülldrucks, im MPM bleiben. Diese Meßreihe wurde immer
wieder verschoben, zugunsten wichtigerer Untersuchungen, wie z.B. derer
im folgenden Abschnitt bezüglich verschiedener Injektionsgase.
%
\beginsubsection{Injektionsgas}
%
Die Ab"-hän"-gig"-keit des Entladungsmodus vom Injektionsgas wurde
als erstes bemerkt und schon in der Arbeit von Mälzig
\cite{maelzig:phd} dokumentiert. Nach dieser Arbeit wurde intensiv
der MPM untersucht. Erst wieder in der Arbeit von Lucas
\cite{lucas:diplom} wurde das Umschalten zwischen den Modi
beachtet. Anfangs wurde die Art des Injektionsgases als einziger
Parameter erkannt, danach wurden Experimente mit erhöhter
Bankenergie exemplarisch durchgeführt \cite{kies:97}. Später
findet sich sogar noch in der Arbeit von Engel \cite{engel:phd}
auf Seite 31 die falsche Behauptung, daß die Betriebsmodi an
SPEED~2 unabhängig von den Betriebsparametern sind.
\par
In der Arbeit von Lucas \cite{lucas:diplom} wurde untersucht, wie sich
der Modus ändert, wenn verschiedene Injektionsgase benutzt werden.
Allerdings wurden bei diesen Messungen auch andere Parameter verändert.
\par
Ergänzend wurde daher eine Meßreihe durchgeführt, bei der nur das
Injektionsgas verändert wurde und der Entladungsmodus in
Ab"-hän"-gig"-keit der Injektionszeit bestimmt wird. Die Parameter
der Entladungen sind in der Tabelle \vref{tab:injektionsgase:para}
aufgeführt.
%
\par
\begin{table}[H]
  \center
  \begin{tabular}{|l|c|c|}
  \hline
      & \multicolumn{2}{c|}{Injektionsgas } \\
      & Neon           & Argon              \\
  \hline
    Ladespannung U             & \multicolumn{2}{c|}{ \wert{180}{kV} }     \\
    Bankenergie E              & \multicolumn{2}{c|}{ \wert{66}{kJ} }      \\
    Füllgas                    & \multicolumn{2}{c|}{ Deuterium 2.7 }      \\
    Fülldruck p(D$_2$)         & \multicolumn{2}{c|}{ \wert{10.1}{hPa} }   \\
    Injektionsgas              & Neon 4.0         &     Argon 4.6          \\
    Injektionsdruck p(Inj.)    & \ewert{5.0}{5}{Pa} & \ewert{5.0}{5}{Pa}   \\
    Injektionszeit \teff       & \multicolumn{2}{c|}{ \wert{0.5-14.5}{ms} }\\
    Nummer                     & 12.090--12.253 & 12.255--12.267           \\
  \hline
  \end{tabular}
  \caption{Parameter der Entladungen in Tabelle \ref{tab:injektionsgase}}
  \label{tab:injektionsgase:para}
\end{table}
%
\par
Die Ergebnisse aus der Auswertung der Bilder der
Rönt"-gen"-pin"-hole"-kame"-ra sind in der Tabelle
\vref{tab:injektionsgase} zusammengefaßt. Bei der Teilmeßreihe mit
Argon-Injektion wurden nur die notwendigsten Werte eingestellt, um
die Anzahl der Entladungen klein zu halten.
\par
Zur Erinnerung sei an dieser Stelle noch einmal kurz erwähnt: die
effektive Injektionszeit \teff bezieht sich auf die Zeit, in der das
Gas in die Entladungskammer einströmt bis zur Triggerung der Zündung.
Die Zeit für das Strömen des Gases vom schnellen Magnetventil zur
Anodenbohrung ist bereits subtrahiert.
\par
Aufgrund des Aufwands beim Austausch des Injektionsgases wurden die
beiden Teilmeßreihen mit den verschiedenen Gasen nacheinander
durchgeführt. Innerhalb dieser Teilmeßreihen wurde die Reihenfolge der
Injektionszeiten wieder zufällig gewählt.
%
\par
\begin{table}[H]
  \center
  \begin{tabular}{|c|c|c|c|c|c|c|c|}
    \hline
              & \multicolumn{4}{c|}{Injektionsgas} \\
    Inj.-zeit & \multicolumn{2}{c|}{ Neon } & \multicolumn{2}{c|}{ Argon } \\
    \teff /ms & \makebox[2cm]{SCM} & \makebox[2cm]{MPM} & \makebox[2cm]{SCM} & \makebox[2cm]{MPM} \\
    \hline
     0.5    &  2 & 0 &  2 & 0 \\
     2.5    & 10 & 0 &  2 & 0 \\
     4.5    &  2 & 0 &  0 & 2 \\
     6.5    &  7 & 0 &  - & - \\
     7.5    &  - & - &  0 & 2 \\
     8.5    &  1 & 1 &  - & - \\
     9.5    &  0 & 2 &  - & - \\
    10.5    &  0 & 2 &  - & - \\
    14.5    &  - & - &  0 & 2 \\
    \hline
  \end{tabular}
  \caption{Verschiedene Injektionsgase}
  \label{tab:injektionsgase}
\end{table}
%
\par
Die Breite des Über"-gangs"-be"-rei"-ches liegt hier bei unter
\wert{3.0}{ms}. Die einzelnen Entladungen waren also gut
reproduzierbar. So können die Grenzpunkte zwischen dem MPM und dem
SCM gut abgelesen werden. Sie sind in der Tabelle
\vref{tab:injektionsgase:result} aufbereitet.
%
\par
\begin{table}[H]
  \center
  \begin{tabular}{|c|c|c|}
    \hline
                             & \multicolumn{2}{c|}{Injektionsgas} \\
                             & Neon                    &  Argon   \\
    \hline
    \teff                    & \wert{8.5}{ms}          & \wert{3.5}{ms}             \\
    p(Inj.)$\cdot$ \teff     & \wert{4250}{Pa$\cdot$s} & \wert{1750}{Pa$\cdot$s}    \\
    n(Inj.)                  & \ewert{1.9}{20}{m$^{-1}$} & \ewert{6.8}{19}{m$^{-1}$}\\
    \hline
  \end{tabular}
  \caption{Grenze bei verschiedenen Injektionsgasen}
  \label{tab:injektionsgase:result}
\end{table}
%
\par
Der Bereich des SCM ist bei der Neongas-Injektion deutlich größer als bei der
Argongas-Injektion.
\par
Die Versuche in der Arbeit \cite{lucas:diplom} mit anderen
Injektionsgasen wurden nicht als Meßreihen über die Injektionszeit
durchgeführt. Es wurde, entgegen dem Ansatz in diesem Kapitel
\thesection, nach typischen Entladungen bei verschiedenen Gasen
gesucht. Dabei wurden die Betriebsparameter zwischen den Versuchen
leider verändert, so daß die Ergebnisse nicht direkt vergleichbar sind.
\par
In der Tabelle \vref{tab:lucas:para} sind die Parameter der dort
vorgestellten Entladungen zusammengestellt. Die Angaben in der
Diplomarbeit wurden dazu um Werte aus dem Laborbuch ergänzt.
%
\par
\begin{table}[H]
  \center
  \begin{tabular}{|l|c|c|c|c|c|c|c|}
    \hline
    Gas & Z & $\frac{\rm U}{\rm kV}$ & $\frac{\rm E}{\rm kJ}$ & $\frac{\rm p(D_2)}{\rm hPa}$ & $\frac{\rm p(Inj.)}{\rm Pa}$ & $\frac{\rm t}{\rm ms}$ & $\frac{\tau_{\rm eff}}{\rm ms}$ \\
    \hline
     Ar    & 18 & 180 & 59 & 3.4 & \ewert{4.5}{5}{} & 6.0  & 0.5 \\
     TMS   & 14 & 180 & 64 & 4.0 & \ewert{1.0}{5}{} & 10.0 & 2.0 \\
     Ne    & 10 & 180 & 59 & 3.4 & \ewert{4.5}{5}{} & 5.5  & 1.0 \\
     N$_2$ &  7 & 180 & 68 & 4.2 & \ewert{4.5}{5}{} & 5.5  & 0.5 \\
     CH$_4$&  6 & 150 & 44 & 4.0 & \ewert{4.5}{5}{} & 5.0  & 1.0 \\
    \hline
  \end{tabular}
  \caption{Parameter der Entladungen in Tabelle \ref{tab:lucas}}
  \label{tab:lucas:para}
\end{table}
%
\par
Alle Entladungen wurden in einer Deuterium Gasfüllung gezündet. Die
betrachteten Entladungen liegen im Bereich Nummer 8856 - 9491. Sie
wurden alle mit dem gleichen Isolator durchgeführt.
\par
Die effektive Injektionszeit \teff wurde aus der eingestellten
Ver"-zö"-ge"-rungs"-zeit t abgeschätzt. Messungen des Zeitpunkts,
bei dem das Injektionsgas in der Entladung sichtbar wird, wurden
damals nicht durchgeführt. Zur Abschätzung wurde aus der Masse der
Moleküle die thermische Geschwindigkeit berechnet und damit die
Zeit für das Strömen vom schnellen Magnetventil zur Anodenbohrung
abgeschätzt.
\par
Die Angabe des Injektionsdrucks wurde von \ewert{3.5}{5}{Pa} auf
\ewert{4.5}{5}{Pa} anhand der Aufzeichnungen im Laborbuch korrigiert.
(Die üblichen Flaschendruckminderer geben den Gasdruck relativ zum
umgebenden Luftdruck an.)
\par
Das Kürzel Z in der Tabelle steht für die Ordnungszahl. Bei
Gasgemischen ist die Ordnungszahl des schwersten Elementes angegeben.
Das Kürzel TMS steht für Tetramethylsilan (SiC$_4$H$_{12}$). Das TMS
wurde benutzt, um ein Gas mit Z = 14 im Übergangsbereich zwischen dem
SCM und dem MPM einzubringen.
\par
Die Messungen mit den Injektionsgasen Helium und Wasserstoff sind
hier nicht aufgeführt, da bei diesen Injektionsgasen die
Rönt"-gen"-strah"-lungs"-in"-ten"-si"-tät zu gering ist.
%
\par
\begin{table}[H]
  \center
  \begin{tabular}{|c|c|c|c|c|c|c|}
    \hline
    Gas & \makebox[2cm]{SCM} & \makebox[2cm]{MPM} & \hfill Liniendichte \hfill \\
    \hline
     Ar       &    & X & n(Ar) = \ewert{8.7}{18}{m$^{-1}$} \\
     TMS      &  X & X & n(Si) = \ewert{5.2}{18}{m$^{-1}$}  \\
     Ne       &  X &   & n(Ne) = \ewert{2.0}{19}{m$^{-1}$} \\
     N$_2$    &  X &   & n(N)  = \ewert{1.7}{19}{m$^{-1}$} \\
     CH$_4$   &  X &   & n(C)  = \ewert{2.3}{19}{m$^{-1}$} \\
    \hline
  \end{tabular}
  \caption{Verschiedene Injektionsgase II}
  \label{tab:lucas}
\end{table}
%
\par
Die Tabelle \vref{tab:lucas} faßt die Ergebnisse in Bezug auf den
Entladungsmodus zusammen. Das Kreuz in der Tabelle gibt den
typischen Entladungsmodus, bei den genannten Betriebsparameter,
an.
\par
Zusätzlich ist in der Tabelle eine Abschätzung für die Liniendichte,
der entsprechenden Atome/Ionen in der Pinchregion angegeben. Die
Abschätzung beruht auf den Werten von Argon. Die Gasmenge pro
Injektionszeit und Injektionsdruck wurde dafür proportional zu
$\sqrt{\frac{M(Ar)}{M(X)}}$ angenommen (M(X) = Molmasse von X).
\par
Der Übergangsbereich wurde bei TMS mit Z = 14 gefunden. Dabei muß
natürlich beachtet werden, daß die übrigen Betriebsparameter auch
verändert wurden. Die Messung mit Methan (CH$_4$) fällt besonders durch
die reduzierte Ladespannung U und Bankenergie E aus der Reihe. Die
übrigen Änderungen im Fülldruck und in der Bankenergie sind nicht
entscheidend, wie die hier vorgestellten Messungen gezeigt haben.
\par
Der geringe Injektionsdruck bei TMS hat dagegen einen Einfluß. Zum Teil
wurde dies schon bei den Messungen, durch die vergrößerte
Injektionszeit, ausgeglichen. Das für die Anzahl der
Injektionsgasteilchen wichtige Produkt p(Inj.)$\cdot$\teff wurde so
annähernd konstant gehalten. Die hohe Masse der TMS Moleküle wird aber
dennoch, zu einer deutlich geringeren Injektionsteilchenanzahl geführt
haben, da die Strömungsgeschwindigkeit deutlich reduziert ist.
Entsprechend ist die geschätzte Liniendichte um einen Faktor 4 kleiner
als die Liniendichte bei Methan.
\par
TMS (SiC$_4$H$_{12}$) ist zudem schwierig in der Interpretation, da die
Anteile von Kohlenstoff mit Z = 6 sicher nicht vernachlässigt werden
können, im Gegensatz zu den Wasserstoffanteilen des Methans (CH$_4$).
\par
Für weitere Aussagen ist es daher sicherer, nur die beiden ausführlich
vermessenen Gase Neon und Argon zu betrachten.
%
\beginsubsection{Zusammenfassung}
%
Damit sind alle durchgeführten Untersuchungen, bezüglich der
Ab"-hän"-gig"-keit des Entladungsmodus von den Betriebsparametern
des Experiments, dargestellt worden. Insgesamt wurden innerhalb
des Zeitraumes August 1997 bis September 1998 über 1000
Entladungen, überwiegend für diese Meßreihen, mit dem Plasmafokus
SPEED~2 durchgeführt. Alle Entladungen zeigten entweder (1) eine
Säule - den SCM -, (2) kleine Zentren oder (3) eine Säule mit
kleinen Zentren - den MPM - als Rönt"-gen"-strah"-lungs"-quel"-le.
Es wurde keine andere Entladungsform in dem gesamten untersuchten
Bereich der Betriebsparameter gefunden.
\par
Bei allen untersuchten Einstellungen der Betriebsparametern wurde der
SCM immer bei kurzen Injektionszeiten gefunden und der MPM bei langen
Injektionszeiten. Bei der Variation der Injektionszeit ist, unter allen
untersuchten Betriebsbedingungen, immer nur ein Übergang vom SCM zum
MPM beobachtet worden.
\par
Zwischen den beiden Modi lag immer ein Bereich, in dem der
Entladungsmodus den Schuß-zu-Schuß Schwankungen unterlag.
Außerhalb dieses Über"-gangs"-be"-rei"-ches führten die
Schuß-zu-Schuß Schwankungen nicht zu einem Wechsel des Modus, nur
zu unterschiedlichen Intensitäten der Röntgenstrahlung und der
Neutronenproduktion.
\par
Die Breite des Über"-gangs"-be"-rei"-ches hat sich stark abhängig
von den Isolatoren gezeigt. \glqq Gute\grqq\ Isolatoren zeigen
geringe Schuß-zu-Schuß Schwankungen und somit kleine
Übergangsbereiche.
\par
Die Isolatoren beeinflussen auch die Lage der (idealisierten) Grenze
zwischen den beiden Entladungsmodi sehr stark. Die Möglichkeit der
gezielten Konditionierung der Isolatoren, für einen großen Bereich im
SCM oder im MPM haben sich angedeutet, konnten aber aus Zeitgründen
nicht weiter untersucht werden.
\par
Der Grenzpunkt, bei den Injektionszeiten zwischen dem SCM und dem MPM,
ist abhängig von den übrigen Betriebsparametern der Anlage. Unter
gewissen Unsicherheiten, die in den einzelnen Abschnitten diskutiert
wurden, konnten die Tendenzen bestimmt werden. Sie sind in der Tabelle
\vref{tab:tendenzen:injektionszeit} zusammengestellt.
%
\par
\begin{table}[H]
  \center
  \begin{tabular}{|l|c|c|}
      \hline
                                &            & Grenze           \\
    Betriebsparameter           & Variation  & Inj.-zeit         \\
      \hline
    Injektionsdruck p(Inj.)     & $\nearrow$ & $\searrow$        \\
    Ladespannung U + Energie E  & $\nearrow$ & $\nearrow$        \\
    Bankenergie E               & $\nearrow$ & $\nearrow$        \\
    Fülldruck p(D$_2$)          & $\nearrow$ & $\leftrightarrow$ \\
    Injektionsgas Z(Inj.)    & $\nearrow$ & $\searrow$        \\
      \hline
  \end{tabular}
  \caption{Tendenzen bei der Lage der Grenze bzgl. Injektionszeit}
  \label{tab:tendenzen:injektionszeit}
\end{table}
%
\par
Die kurze Symbolschreibweise in der Tabelle bedeutet ausführlich: Der
Grenzpunkt der Injektionszeit verschiebt sich zu längeren Zeiten, wenn
die Ladespannung U (und damit die Bankenergie E) vergrößert wird oder
die Bankenergie E alleine vergrößert wird. Der Grenzpunkt verschiebt
sich zu kürzeren Zeiten, wenn der Injektionsdruck p(Inj.) vergrößert
wird oder die Kernladungszahl des Injektionsgases Z(Inj.) erhöht wird.
Keine Verschiebung ergibt sich, wenn der Deuteriumfülldruck p(D$_2$)
verändert wird. Streng gelten diese Aussagen nur für die durchgeführten
Meßreihen, aber es ist wahrscheinlich, daß sie sich im gesamten
Parameterbereich der effektiven Entladungen als gültig herausstellen.
\par
Zusammen mit den Abschätzungen der
In"-jek"-tions"-gas-Li"-ni"-en"-dich"-te n im Pinch folgen
Tendenzen für den Grenzpunkt bei diesem Parameter. Diese sind
wieder in der kurzen Symbolschreibweise in der Tabelle
\vref{tab:tendenzen:teilchen} zusammengefaßt.
%
\begin{table}[H]
  \center
  \begin{tabular}{|l|c|c|}
      \hline
                                &            & Grenze            \\
    Betriebsparameter           & Variation  & Liniendichte      \\
      \hline
    Injektionsdruck p(Inj.)     & $\nearrow$ & $\leftrightarrow$ \\
    Ladespannung U + Energie E  & $\nearrow$ & $\nearrow$        \\
    Bankenergie E               & $\nearrow$ & $\nearrow$        \\
    Fülldruck p(D$_2$)          & $\nearrow$ & $\leftrightarrow$ \\
    Injektionsgas Z(Inj.)    & $\nearrow$ & $\searrow$        \\
      \hline
  \end{tabular}
  \caption{Tendenzen bei der Lage der Grenze bzgl. Liniendichte}
  \label{tab:tendenzen:teilchen}
\end{table}
%
\par
Eine noch offene Aufgabe ist die bessere Bestimmung der Gasmischung vor
dem Einlaufen der Plasmaschicht, also die Teilchendichte n(r,h) als
Funktion der Höhe über der Anode und des Abstands von der Achse. Ohne
diese, schwer zu bestimmende Größe bleiben die Aussagen mit Bezug auf
die Injektionsgas-Dichte mit erheblichen Unsicherheiten behaftet.
