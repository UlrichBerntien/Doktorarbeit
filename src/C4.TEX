%
%  Zusammenstellung der benutzten Diagnostiken
%
\beginsection{Diagnostiken}
\label{sec:diagnostiken}
%
Die Diagnostik besteht überwiegend aus Meßgeräten für die
Detektion im Bereich der weichen Röntgenstrahlung. Mit einer
Rönt"-gen"-pin"-hole"-kame"-ra wurden zeitintegrierte Bilder
aufgenommen. Eine Röntgenstreakkamera diente zur zeitaufgelösten
Messung. Über ein Vielschichtspiegel-System (MLM) konnten
wellenlängenselektiv Bilder mit einer Belichtungszeit von
\wert{10}{ns} aufgenommen werden. Ein Kristallspektrometer
lieferte ein Röntgenspektrum im gleichen Zeitfenster.
\par
Die Fusionsneutronen wurden zeitintegriert (Silberaktivierungszähler) und
zeitaufgelöst (Szintillator mit Photomultiplier) registriert.
\par
Am Fokus wurde die Spannung U(t) abgegriffen und zusammen mit dem
differenzierten Strom $\dot{\rm I}$(t) zeitaufgelöst aufgezeichnet. Für
die zeitliche Zuordnung der zeitaufgelösten Diagnostiken wurden deren
Monitorsignale mit aufgezeichnet. Zur Aufzeichnung aller elektrischen
Meßsignale dienten zwei digitale Speicheroszilloskope Gould 4074.
\par
Keine der Diagnostiken wurde bei den Experimenten neu in Betrieb
genommen. Es sind die bereits in den vorhergehenden Arbeiten
\cite{maelzig:phd,roewe:phd} beschriebenen Geräte. Zur vollständigen
Beschreibung der Laborumgebung ist dieses Kapitel \thesection\ aber
notwendig. Die/Der informierte Leserin/Leser kann dieses Kapitel
\thesection\  überspringen, oder sich anhand der maßstabsgetreuen
Zeichnungen einen Überblick verschaffen. Die Zeichnungen sind nicht
immer optimal für die Darstellung der Diagnostik, aber durch ihre
Maßstabstreue geben sie einen Eindruck von der experimentellen
Situation. Schematische Zeichnungen zum Funktionsprinzip der
Diagnostiken sind in vielen Quellen z.B. in
\cite{roewe:phd,maelzig:phd} zu finden.
%
\par
\begin{figure}[H]
  \center
  \fbox{\importimage{C4-02}}
  \caption{Ausrichtung der Diagnostiken. In der Mitte die durchbohrte Anode,
  umgeben von den Kathodenstäben. (Details vereinfacht)}
  \label{fig:Richtung:Diagnostiken}
\end{figure}
%
\par
Die Abbildung \vref{fig:Richtung:Diagnostiken} zeigt die Ausrichtung
der Diagnostiken auf den Plasmafokus. Die Bohrung in der Anode wird bei
den Entladungen durch die hohe Belastung verändert. Über die Einnordung
der Zeichnung, kann die leichte Asymmetrie der vergrößerten Bohrung in
Bezug zu den Diagnostiken gebracht werden. (vgl. Abbildung
\vref{fig:bohrung}).
%
\beginsubsection{Elektrische Diagnostiken}
%
Der Abgriff für das Spannungssignal und die Lage der Pickup-Spule für
das $\rm\dot{I}$ sind in der Abbildung \vref{fig:Richtung:Diagnostiken}
eingezeichnet.
\par
Der Spannungsabgriff erfolgt am Hauptkollektor. Durch die
Kollektorkapazität werden sehr schnelle Spannungsänderungen im
Fokus-Bereich gedämpft, daher wird typisch nur $1/3$ der Pinchspannung
gemessen \cite{stein:staat}. Die Spannung wird über einen kapazitiven
Teiler (1:7100) und einen ohmschen Teiler (1:10) für das DSO reduziert.
\par
Die Pickup-Spule ist in den Kupferflansch der Kathode eingebettet, sie
liefert das $\rm\dot{I}$-Signal. Gemessen wird der gesamte Strom im
Fokus, der Pinchstrom wird nicht getrennt gemessen.
%
\beginsubsection{Neutronendiagnostik}
\label{sec:neutronendiagnostik}
%
Bei einer effizienten Pinchbildung kommt es zur Bildung von
Fusionsneutronen. Die Anzahl der entstandenen Neutronen, bei einer
Entladung, gibt sofort einen Eindruck über den Zustand des Experiments,
insbesondere über die Konditionierung des Isolators. Bei dieser
Bewertung ist auch die Injektionsgasmenge und die Bankenergie zu
beachten. Die Fusionsneutronen können über zwei verschiedene
Reaktionswege enstehen \cite{dtv:70}, wobei, in der Situation der
Fokusentladung, der Reaktionsweg über das Tritium vernachlässigbar ist:
\par
\begin{tabular}{rl}
  ${2 \atop 1}{\rm D} \ +\ {2 \atop 1}{\rm D} \longrightarrow
  $&${3 \atop 2}{\rm He}\ (0.8 {\rm\ MeV}) \ +\ \rm n\ (2.5 {\rm\ MeV}) $\\[1.5mm]

  ${2 \atop 1}{\rm D} \ +\ {2 \atop 1}{\rm D} \longrightarrow
  $&${3 \atop 1}{\rm T}\ (1.0 {\rm\ MeV}) \ +\ \rm p\ (3.0 {\rm\ MeV}) $\\[1.5mm]

  &${3 \atop 1}{\rm T} \ +\ {2 \atop 1}{\rm D} \ \longrightarrow \
  {4 \atop 2}{\rm He}\ (3.5 {\rm\ MeV}) \ +\ \rm n\ (14 {\rm\ MeV}) $\\[1.5mm]
\end{tabular}
\par
Eingesetzt wurden zur Detektion der Neutronen zwei Silberaktivierungszähler und
eine Szintillator-Photomultiplier Kombination.
\par
Die Silberaktivierungszähler\footnote{Firma Herfuth GmbH,
    Typ H 1359 A-K.
    Herfurth ist in die Rados Technology GmbH, Hamburg aufgegangen
    (\url{http://www.rados.de/}).}
bestehen aus einer Silberplatte und einem
Geiger-Müller-Zählrohr\cite{steinmetz:80}. Die Neutronen aktivieren die Silberatome:
\par
$\begin{array}{rcl}
  {109 \atop \hfill 47}{\rm Ag} \ +\ \rm n \ \longrightarrow \
  &{110 \atop \hfill 47}{\rm Ag}& \ +\ \gamma \\[1.5mm]

  &{110 \atop \hfill 47}{\rm Ag}& \ \longrightarrow \
  {110 \atop \hfill 48}{\rm Cd}
  \ +\ \beta^- \ (2.8 {\rm\  MeV}) \ +\ \bar\nu_e \\[1.5mm]

  \rm {107 \atop \hfill 47}{\rm Ag} \ +\ {\rm n} \ \longrightarrow \
  &{108 \atop \hfill 47}{\rm Ag}& \ +\ \gamma \\[1.5mm]

  &{108 \atop \hfill 47}{\rm Ag}& \ \longrightarrow \
  {108 \atop \hfill 48}{\rm Cd}
  \ +\ \beta^- \ (1.8 {\rm\ MeV}). \ +\ \bar\nu_e \\[1.5mm]
\end{array}$
\par
Die $\beta$-Strahlung, beim Zerfall der angeregten Silberatome, wird
mit dem Zählrohr im Proportionalzählbereich gemessen. Die Zählpulse
werden mit zwei Digitalzählern (Universal counter 5316A, Hewlett
Packard) über einen Zeitraum von \wert{10}{s} nach einer Entladung
gezählt. Die beiden Silberaktivierungszähler stehen in einem Abstand
von \wert{2.5}{m} bzw. \wert{9}{m} zum Fokus.
\par
Die Neutronen wurden zusätzlich zeitaufgelöst registriert. Dazu
wurde die Kombination eines schnellen Szintillators (NE 111,
Zylinder \wert{r = 2.5}{cm}, \wert{d = 1.2}{cm}) vor einem
Photomultiplier (Typ R 12944-01 von Hamamatsu Photonics,
Herrsching, \url{http://www.hamamatsu.com/}) benutzt. Diese
Kombination ermöglicht eine Zeitauflösung von etwa \wert{1}{ns}
\cite{kies:phd}, aber die verwendeten DSO haben nur eine
Zeitauflösung von \wert{2.5}{ns}.
\par
Der Aufbau dieses Detektors an SPEED~2 ist in der Abbildung
\vref{fig:Photomultiplier} dargestellt.
\par
Die Abschirmung aus Bleikacheln verhindert fast vollständig die Störung
der Neutronenmessung durch Röntgenstrahlung, für die der
Photomultiplier ebenfalls empfindlich ist. Bei einer Neutronenenergie
von \wert{1}{MeV} beträgt die Halbwertsdicke von Blei \wert{6}{cm}, für
Röntgenstrahlung beträgt die Halbwertsdicke nur \wert{1}{cm} bei der
gleichen Energien \cite{dtv:70}. Zu kleineren Energien wird das
Verhältnis noch günstiger. Die Metallabschirmung dient zum Schutz vor
sichtbarem Licht und zur elektromagnetischen Abschirmung.
%
\par
\begin{figure}[H]
  \center
  \fbox{\importimage{C4-03}}
  \caption{Aufbau Szintillator-Photomultiplier Kombination.
   (maßstabsgetreu, Details vereinfacht)}
  \label{fig:Photomultiplier}
\end{figure}
%
\par
Der Photomultiplier wurde mit einer Spannung von \wert{2.5}{kV} betrieben
und sein Ausgangssignal auf einem DSO aufgezeichnet.
\par
Der Abstand zum Fokus ist in der Zeichung mit \wert{1.0}{m} angegeben.
Wenn es die Anordnung zugelassen hat, wurde auch mit einem Abstand von
\wert{0.75}{m} gearbeitet. Die emittierten Neutronen zeigen eine
typische Energieverteilung \cite{steinmetz:80} von \wert{1}{MeV} bis
\wert{3}{MeV} und damit eine entsprechende Geschwindigkeitsverteilung.
Ein geringer Abstand zur Quelle ist daher für die zeitaufgelöste
Bestimmung der Neutronenproduktion wichtig.
\par
Ein nicht bei dieser Arbeit verwendete Verfahren erlaubt, aus den
Signalen von mehreren Szintillator-Photomultiplier-Kombinationen in
verschiedenen Abständen das Energiespektrum zeitlich aufgelöst zu
bestimmen \cite{tiseanu:96}.
%
\beginsubsection{Röntgenpinholekamera}
\label{pinholekamera}
%
Die Rönt"-gen"-pin"-hole"-kame"-ra ist die wichtigste
Rönt"-gen"-dia"-gnos"-tik bei diesem Experiment. Aufgrund der
einfachen Handhabung und des einfachen Aufbaus, ist diese Kamera
auch die zu"-ver"-läs"-sig"-ste Rönt"-gen"-dia"-gnos"-tik.
\par
Das Arbeitsprinzip der optischen Lochkamera läßt sich sofort auf
die Rönt"-gen"-pin"-hole"-kame"-ra über"-tra"-gen. Als
Wandmaterial für das Loch wurde Kupfer, Eisen oder Molybdän
benutzt. In Kupferfolie lassen sich die Löcher mit einer Nadel
stechen; in Eisen können Löcher gebohrt werden. Molybdän-Blenden
werden für den Einsatz in Elektronenmikroskopen kommerziell
gefertigt.
\par
Die Lichtschwäche der Pinhole-Abbildung stört bei diesen Experimenten
nicht, weil die Röntgenstrahlung des Plasmas sehr intensiv ist. Es
wurden Pinholes bis zu einem minimalen Durchmesser von
\wert{20}{$\mu$m} benutzt. Beugungseffekte sind durch die kurze
Wellenlänge des Röntgenlichts (Bereich unter \wert{5}{nm}) nicht
störend.
\par
Die Rönt"-gen"-pin"-hole"-kame"-ra liefert ein zeitintegriertes
Bild über die gesamte Entladung. So entfallen alle
Synchronisationsprobleme mit der Entladung. Dennoch zeigt die
Aufnahme nur ein kurzes Zeitfenster der Entladung, weil über
geeignete Filter nur Röntgenstrahlung aus der heißen Pinchsäule
den Film erreichen kann. Die einlaufenden Schichten, vor der
Kompression, sind auf den Aufnahmen nicht zu sehen.
%
\par
\begin{figure}[H]
  \center
  \fbox{\importimage{C4-06}}
  \caption{Aufbau der Rönt"-gen"-pin"-hole"-kame"-ra (maßstabsgetreu, Details vereinfacht)}
  \label{fig:Pinholekamera}
\end{figure}
\par
%
Durch einen Transmissionsfilter hinter dem Pinhole wird der
beobachtete Wel"-len"-län"-gen"-be"-reich in Richtung langer
Wellenlängen abgeschnitten. In Richtung kurzer Wellenlängen
begrenzt der Detektor (Röntgenfilm) den beobachtbaren Bereich. Die
Transmissionskurven der Filter sind im Abschnitt
\vref{sec:Transmissionsfilter} für alle Diagnostiken
zusammengefaßt.
\par
Die Abbildung \vref{fig:Pinholekamera} zeigt den Aufbau der Kamera
an dem SPEED~2 Experiment. Die Abstände Fokus -- Pinhole, Pinhole
-- Film sind aus den benutzten Einstellungen beispielhaft
ausgewählt.
\par
Bei entsprechender Verkleinerung ist es möglich, nicht nur ein
Bild mit Hilfe eines Pinholes aufzunehmen. Die
Rönt"-gen"-pin"-hole"-kame"-ra wurde mit Doppelpinholes
(Durchmesser \wert{100}{$\mu$m}) und Vierfachpinholes (Durchmesser
\wert{150}{$\mu$m}, eines \wert{20}{$\mu$m}), versehen mit
verschiedenen Filtern, benutzt. Über die unterschiedlichen Filter
werden Bereiche der Pinchsäule entsprechend der Wellenlänge ihrer
Röntgenemission unterscheidbar.
\par
Zur Aufzeichnung wurde der Röntgenfilm Kodak DEF-5 (direct
exposure film) benutzt und mit dem Röntgenfilmentwickler Kodak
LX~24 entwickelt. Der Film ist identisch in seinen Eigenschaften
dem Kodak DEF-2. (Laut eines Kodak Mitarbeiters, bezeichnet die
Nummer nur die Verpackung. Der DEF-2 bzw. DEF-II ist im Handel
nicht mehr erhältlich.)
%
\beginsubsection{Transmissionsfilter}
\label{sec:Transmissionsfilter}
%
Es wurden verschiedene Transmissionsfilter bei der
Rönt"-gen"-pin"-hole"-kame"-ra und der Röntgenstreakkamera
eingesetzt. Durch die Wahl der Filter wird die registrierte
Wellenlänge mitbestimmt.
\par
Die kurzwellige Grenze ist durch den benutzten Detektor bestimmt. Die
langwellige Grenze wird durch den Filter festgelegt. Bei den
Diagnostiken wurden dünne Metallfolien als Filter benutzt.
\par
Transmissionskurven von verschiedenen Festkörpern und Gasen werden
vom Center for X-Ray Optics, Material Science Division, E.O.
Lawrence Berkeley National Lab, zur Verfügung gestellt. Die Kurven
und Daten können über \url{http://www-cxro.lbl.gov/} abgerufen
werden. Nützlich dabei ist die Formel: $\lambda / \rm nm = 1239.5
/ (\rm E/eV)$.
%
\par
\begin{figure}[H]
  \center
  \fbox{\importimage{C4-07A}}
  \caption{Transmissionskurven von \wert{10}{$\mu$m}, \wert{135}{$\mu$m}
    und \wert{500}{$\mu$m} dicken Berylliumfolien}
  \label{fig:Filter:A}
\end{figure}
%
\par
\begin{figure}[H]
  \center
  \fbox{\importimage{C4-07B}}
  \caption{Transmissionskurven von \wert{0.5}{$\mu$m} Kupferfolie
     und einer Kombination aus \wert{10}{$\mu$m} Be mit \wert{0.5}{$\mu$m} Cu}
  \label{fig:Filter:B}
\end{figure}
%
\par
Das Verhalten der Filter läßt sich im wesentlichen durch die Lage der
Absorptionskante beschreiben. Bei der Diskussion der Meßergebnisse wird
immer die Wellenlänge bei 5\% Transmission angegeben. Die Wahl des 5\%
Punktes ist willkürlich. Je nach Intensität der Röntgenstrahlung und
Empfindlichkeit des Detektors, kann auch Strahlung oberhalb der Grenze
sichtbar werden. Daher wird bei den Meßergebnissen das Filtermaterial
und diese Wellenlängengrenze angegeben.
%
\par
\begin{table}[H]
  \center
  \begin{tabular}{|c|c|}
  \hline
     Filter                 & Wellenlängengrenze \\
  \hline
     \wert{500}{$\mu$m} Be  & \wert{\lambda < 0.5}{nm} \\
     \wert{135}{$\mu$m} Be  & \wert{\lambda < 0.8}{nm} \\
     \wert{10}{$\mu$m} Be   & \wert{\lambda < 2.0}{nm} \\
     \wert{10}{$\mu$m} Be + \wert{0.5}{$\mu$m} Cu & \wert{\lambda < 2.0}{nm} \\
  \hline
  \end{tabular}
  \caption{Transmissonsbereiche der benutzten Filter}
  \label{tab:Filtergrenzen}
\end{table}
%
\beginsubsection{Röntgenstreakkamera}
\label{sec:streakkamera}
%
Zur zeitaufgelösten Untersuchung der Röntgenquelle wurde eine
Rönt"-gen"-streak"-kame"-ra vom Typ \glqq Low magnification x-ray
streak camera\grqq\ der Firma Kentech Instruments
Ltd.\footnote{\url{http://www.kentech.co.uk/}} (England) benutzt.
%
\par
\begin{figure}[H]
  \center
  \fbox{\importimage{C4-05}}
  \caption{Interner Aufbau der Röntgenstreakkamera (schematisch) \cite{kentech:93}}
  \label{fig:Streakkamera:intern}
\end{figure}
%
\par
Die Arbeitsweise der Röntgenstreakkamera läßt sich an ihrem internen
Aufbau, in Abbildung \vref{fig:Streakkamera:intern}, erläutern. Das
Elektrodensystem ist bis auf die Ablenkplatten rotationssymmetrisch.
\par
Die Kathode der Streakkamera ist auswechselbar. Bei den Messungen für
diese Arbeit wurde eine low-density CsJ-Kathode benutzt. Diese Kathode
besteht aus schaumartigem Cäsiumjodid auf einer \wert{6}{$\mu$m} dicken
Mylar-Folie. Dieses Material erreicht eine besonders hohe
Quantenausbeute (= Anzahl Elektronen pro Anzahl Photonen), die in der
Abbildung \vref{fig:Streakkamera:Ausbeute} als Funktion der Wellenlänge
der Röntgenphotonen dargestellt ist.
%
\par
\begin{figure}[H]
  \center
  \fbox{\importimage{C4-08}}
  \caption{Quantenausbeute (Elektronen/Photon) für die Photokathode
     (\wert{102}{nm} CsJ auf \wert{6}{$\mu$m} Mylar) der Streakkamera
     \cite{roewe:phd}}
  \label{fig:Streakkamera:Ausbeute}
\end{figure}
%
\par
Nachdem die auftreffenden Röntgenphotonen in der Kathode, entsprechend
der Quantenausbeute, Elektronen ausgelöst haben, werden diese durch das
elektrische Feld beschleunigt. Die Anordnung der Elektroden in der
Kamera bildet eine Elektronenoptik, die das Röntgenbild von der
Kathode, leicht vergrößert, auf den Phosphorschirm abbildet.
\par
Durch die Ablenkplatten kann das Bild auf dem Phosphorschirm
verschoben werden. Eine Schlitz vor der Kathode begrenzt das Bild
auf einen schmalen Streifen (typisch \wert{0.5}{mm} x
\wert{3}{cm}). Durch einen Spannungspuls, mit einem linearen
Bereich im Anstieg, auf die Ablenkplatten erhält man ein
Schmierbild dieses Streifens. Die Ablenkgeschwindigkeit kann bei
der vorhandenen Kamera in sechs Stufen von \wert{1.3}{ns/mm} bis
\wert{11}{ps/mm} eingestellt werden. Zur Untersuchung des stabilen
Säu"-len"-mo"-dus (SCM) wurde immer nur die langsamste Einstellung
benutzt, weil die Lebensdauer der Säule mit bis zu \wert{100}{ns}
relativ groß ist.
%
\par
\begin{figure}[H]
  \center
  \fbox{\importimage{C4-04}}
  \caption{Aufbau der Röntgenstreakkamera (maßstabsgetreu, Details vereinfacht)}
  \label{fig:Streakkamera}
\end{figure}
%
\par
Die Breite des Streifens bestimmt die zeitliche Auflösung. Bei den
Untersuchungen an der Plasmasäule ist die Quelle bereits sehr
schmal, ein enger Spalt kann bei Streaks senkrecht zur Säule
entfallen. Dieses vereinfacht die Justierung der Kamera am
Experiment. Die Abbildung \vref{fig:Streakkamera} zeigt den Aufbau
der Kamera am Experiment. Die Streakkamera ist von dem Rezipienten
elektrisch isoliert und in einem Faraday-Käfig aufgebaut.
\par
Die Streakkamera ist mit einem Vakuumpumpensystem verbunden, damit der
maximale Betriebsdruck von \ewert{1}{-2}{Pa} nicht überschritten wird.
Der Filter zwischen Entladungskammer (\wert{> 100}{Pa}) und
Streakkamera muß aus diesem Grund ebenfalls vakuumdicht sein. Bei zu
hohen Drücken kommt es zu einer Entladung zwischen der Photokathode und
dem Gitter in der Kamera (vgl. Abbildung
\vref{fig:Streakkamera:intern}).
\par
Zur Justierung der Kamera wurde ein Justierlaser benutzt, der die
optische Achse für die Pinhole-Abbildung und die Streakkamera vorgab.
So konnte vom Rezipient aus, in Richtung des Justierlasers, der Aufbau
schrittweise auf die Achse ausgerichtet werden.
\par
Durch eine Pinhole-Abbildung wird das strahlende Plasma auf die
Photokathode der Kamera abgebildet. Der Schlitz der Streakkamera kann
parallel zur Pinchsäule liegen, dann wird das zeitliche Verhalten des
Plasmas auf der Achse sichtbar. Wird der Schlitz der Streakkamera
senkrecht zur Pinchsäule angeordnet, dann wird die zeitliche
Entwicklung des Pinchradius bei einer bestimmten Höhe sichtbar.
Alternativ können in dieser Anordnung auch zwei Bilder des Plasmas über
zwei Pinholes auf die Photokathode abgebildet werden. So wird die
Entwicklung des Radius bei zwei verschiedenen Höhen oder in zwei
verschiedenen Wellenlängenbereichen gleichzeitig aufgezeichnet.
\par
Benutzt wurde ein rundes Pinhole mit einem Durchmesser von
\wert{300}{$\mu$m} oder zwei runde Pinholes mit je einem
Durchmesser von \wert{250}{$\mu$m}. Über den Filter hinter dem
Pinhole kann der gewünschte Wel"-len"-län"-gen"-be"-reich, der zu
messenden Röntgenstrahlung, bestimmt werden.
\par
An die Streakkamera ist ein Bildverstärker (intern eine MCP) montiert
(Typ \glqq 50/40 Image intensifier for a Kentech x-ray streak
camera\grqq\  der Firma Kentech). Die MCP wurde \wert{1}{$\mu$s} vor
der Streakkamera ausgelöst, damit sich die Spannung an der MCP aufbauen
konnte. Da die Streakkamera mit einer fallenden Flanke und der
Bildverstärker mit einer steigenden Flanke getriggert wird, ist nur ein
\wert{1}{$\mu$s} langer Rechteckpuls für beide Geräte erforderlich. Die
genau Länge des Pulses muß eingehalten werden, weil erst am Ende die
Streakkamera ausgelöst wird. Benutzt wurde dafür ein analoger
Pulsgenerator ({\bf HP}, siehe Abbildung \vref{fig:triggerung}), der
einen kleinen Jitter zeigt, wenn sich der Regler für die Pulslänge am
Anschlag befindet.
\par
Aufgezeichnet wurden die Bilder mit einem schwarz/weiß Sofortbildfilm,
Typ 667 von Polaroid (\url{http://www.polaroid.de/}). So konnte nach
jeder Entladung sofort die zeitliche Einstellung und die Justierung der
Kamera überprüft werden. Für die Justierung ist die Arbeitsweise ohne
zeitliche Ablenkung, also mit ausgeschalteter Sweep-Elektronik, sehr
hilfreich.
\par
Die Bilder wurden dann mit einem \wert{300}{dpi} Schwarz/weiß-Scanner
eingelesen. Eine höhere Auflösung ist nicht notwendig, weil die
Auflösung des Polaroid-Bilder bei \wert{12-14}{Linien/mm} liegen
(Herstellerangabe). Höher auflösende Polaroid-Filme (z.B. Typ 665,
Negativ \wert{160-180}{Linien/mm}) konnten nicht benutzt werden, weil
die Empfindlichkeit (ISO 80 gegen ISO 3000) nicht ausreichte.
\par
\label{streaknachbearbeitung} Einige Streakbilder zeigen das, aus
anderen Arbeiten \cite{maelzig:phd,roewe:phd} schon bekannte, Artefakt.
Es entsteht durch Röntgenstrahlung, die den Phosphorschirm erreicht.
Dieses ist genau dann möglich, wenn eine Sichtlinie vom Phosphorschirm
durch die Kamera und das Pinhole auf die Plasmasäule besteht. Also
nicht bei den Bildern die mit zwei Pinholes erstellt wurden.
\par
Die Abbildung \vref{fig:bildbearbeitung} zeigt, wie in dieser Arbeit
der störende Fleck mit einem Grafikprogramm entfernt wurde. Mit der
Radierfunktion des Programms wurden die Grauwerte solange reduziert,
bis sich ein kontinuierlicher, sinnvoller Bildinhalt ergab.
\par
Auf den nachbearbeiteten Bildern läßt sich das Wesentliche schneller
erkennen. Im Zentrum weisen sie natürlich Ungenauigkeiten durch die
Nachbearbeitung auf. Wie das Differenzbild zwischen dem Original und
dem bearbeiteten Bild zeigt, wurde im Artefakt die Sättigung des
Scanners oder des Kamerasystems erreicht. Das Subtrahieren eines
Nullbilds (d.h. Bild nur mit Artefakt) würde daher nicht zum
gewünschten Effekt führen. Die Ungenauigkeiten im Zentrum des Bildes
werden hingenommen, um einen leicht interpretierbaren Eindruck zu
erreichen.
\par
Eine weitere Ungenauigkeit ist gegeben durch eine leichte Verzerrung im
Bildverstärker. Der Hersteller gibt dazu die Beziehung $r' = 0.0002128
{\rm mm}^{-2} \cdot r^3 + 0.667 \cdot r$ mit $r, r'$ = radiale Position
auf der Eintritts- bzw Austrittsseite an. Diese Verzerrung wurde in den
Abbildungen nicht korrigiert. In den Randbereichen wird der Fehler
maximal, dort beträgt der relative Fehler 11\%.
%
\par
\begin{figure}[H]
  \center
  \fbox{\importimage{C4-01}}
  \caption{Nachbearbeitung bei Bildern der Streakkamera, die ein Artefakt zeigen}
  \label{fig:bildbearbeitung}
\end{figure}
%
\beginsubsection{Mikrokanalplatte (MCP)}
%
Die Mikrokanalplatte (MCP für englisch microchannel plate) wird zur
Bildverstärkung und/oder zur Kurzzeitbelichtung verwendet. Verschiedene
Typen wurden bei den Messungen in Kombination mit der
Röntgenstreakkamera, dem Kristallspektrometer und dem MLM-System
eingesetzt.
\par
Eine MCP \cite{wiza:79} besteht aus einer Vielzahl von einzelnen
Kanälen in einer Glasplatte. Die Kanäle besitzen einen Durchmesser im
Bereich \wert{10-100}{$\mu$m} und sind \wert{0.1-4.0}{mm} lang. Der
Durchmesser und damit der Abstand der Kanäle, ist eine Grenze für das
Auflösungsvermögen einer MCP.
\par
Die Abbildung \vref{fig:MCP:Kanal} zeigt die Arbeitsweise eines
Kanals. Ein Photon trifft auf die Photokathode und löst dort ein
Elektron aus (entsprechend der Quanteneffizienz der Photokathode).
Die langwellige Grenze der Empfindlichkeit liegt bei \wert{\lambda
< 120}{nm} und kann durch Beschichtung mit CsJ auf \wert{\lambda <
200}{nm} erweitert werden \cite{martin:82}. Die Messung von
Schmitz \cite{schmitz:diplom} haben gezeigt, daß die sichtbaren
Strukturen auf den MCP-Bildern überwiegend aus dem
Wel"-len"-län"-gen"-be"-reich \wert{\lambda < 20}{nm} stammen. Das
Elektron wird durch das elektrische Feld im Inneren des Kanals
beschleunigt. Beim Auftreffen auf die Wand werden
Sekundärelektronen erzeugt. Jeder Kanal stellt eine Art
kontinuierliches Dynodensystem eines Photomultipliers dar. Auf der
Austrittsseite werden die Elektronen durch einen Phosphorschirm
wieder in Photonen umgesetzt.
%
\par
\begin{figure}[H]
  \center
  \fbox{\importimage{C4-10}}
  \caption{Elektronenlawine in einem Kanal der MCP}
  \label{fig:MCP:Kanal}
\end{figure}
%
\par
Aus diesen einzelnen Kanälen setzt sich die gesamte MCP zusammen, wie
die Abbildung \vref{fig:MCP:Schema} zeigt. Die Kathodenseite der MCP
ist leitfähig mit allen Kanälen verbunden, damit die Elektronen ersetzt
werden können. Die Anode ist, je nach Typ der MCP, als Schicht auf dem
Glassubstrat oder separat auf dem Phosphorschirm angebracht. In dem
dann entstehenden Raum zwischen Glasplatte und Anode werden alle
Elektronen, auch die nahe an der Austrittsfläche ausgelösten,
zusätzlich beschleunigt.
%
\par
\begin{figure}[H]
  \center
  \fbox{\importimage{C4-09}}
  \caption{Schematischer Aufbau der MCP}
  \label{fig:MCP:Schema}
\end{figure}
%
\par
Hinter dem Phosphorschirm ist üblicherweise eine Fiberoptik, die das
Licht auf einen Film überträgt. Systeme mit angeschlossenem CCD-Chip
wurden hier nicht eingesetzt.
\par
Die Spannung an der MCP kann gepulst werden, dann wird neben der
Verstärkung auch eine Zeitauflösung erreicht.
\par
Beim Betrieb mit dem MLM-System und dem Kristallspektrometer wurde
diese Möglichkeit ausgenutzt. Pro Entladung konnte so ein Bild bzw. ein
Spektrum mit einer Belichtungszeit von \wert{10}{ns} aufgenommen
werden.
%
\beginsubsection{Kristallspektrometer}
\label{sec:kristallspektrometer}
%
Das Kristallspektrometer wurde von Sidelnikov (Institut für
Spektroskopie, Troitzk) am SPEED~2 Experiment betrieben. Als erfahrener
Spektroskopiker hat er auch die Auswertung der gewonnenen Spektren
durchgeführt.
\par
Das benutzte Spektrometer arbeitet mit einem KAP Kristall. Dieser
Kristall ist ein organischer Einkristall, chemische Formel: $\rm
C_6H_4(COOH)(COOK)$, mit einer Gitterkonstanten von \wert{2d =
2.26}{nm}. Die nutzbare Fläche (100-Ebene) des Kristalls beträgt
\wert{3}{cm} \wert{\cdot\ 1}{cm}.
\par
Der Kristall wird in Johann-Anordnung \cite{johann:31} benutzt.
Der Bragg-Winkel $2{\rm d} \cdot \sin(\varphi) = {\rm n} \cdot
\lambda$, n = 1 wurde auf $22^\circ$, entsprechend einer
Wellenlängen um \wert{\lambda = 1.0}{nm}, eingestellt. Der
beobachtbare Wel"-len"-län"-gen"-be"-reich beträgt bei diesem
Spektrometer \wert{0.16}{nm}.
\par
Der Aufbau des Spektrometers am SPEED~2 gibt schematisch die Abbildung
\vref{fig:Kristallspektrometer} wieder.
%
\par
\begin{figure}[H]
  \center
  \fbox{\importimage{C4-12}}
  \caption{Maßstabsgetreue Darstellung des Kristallspektrometers am SPEED~2.
     Details, das Gehäuse und der Faraday-Käfig sind vernachlässigt.}
  \label{fig:Kristallspektrometer}
\end{figure}
%
\par
Durch einen \wert{0.2}{mm} Spalt, der in der Ebene des Rowland-Kreises
liegt, wird eine räumliche Auflösung erzielt. Ähnlich der Anordnung der
Röntgenstreakkamera, gibt es hier wieder zwei Möglichkeiten der
Beobachtung. Die Pinchsäule kann mit einer z-Auflösung, längs ihrer
Achse, oder mit einer radialen Auflösung beobachtet werden. Dazu muß
das komplette Spektrometer mit dem Spalt um $90^\circ$ gedreht werden.
Für die beiden Fälle ergibt sich eine unterschiedliche Auflösung
aufgrund der unterschiedlichen Ausleuchtung der Kristallfläche: Bei
z-Auflösung \ewert{\Delta\lambda = 5.0}{-4}{nm} und bei radialer
Auflösung \ewert{\Delta\lambda = 6.6}{-4}{nm}.
\par
Das Spektrum wurde mit einer MCP aufgenommen. Von einem entsprechenden
Spannungspuls gesteuert, betrug die Belichtungszeit \wert{10}{ns}. Über
einen Spannungsabgriff wurde ein Monitorsignal zur zeitlichen Zuordnung
des Spektrums aufgenommen. Mit der MCP wurde dann ein
\wert{120}{mm}-Rollfilm Agfa (\url{http://www.agfa.com/}) APX 400
(Empfindlichkeit ISO 400, schwarz/weiß) belichtet.
%
\beginsubsection{Vielschichtspiegel-System (MLM)}
\label{sec:mlm}
%
Das Vielschichtspiegel-System (MLM für englisch multilayer mirror)
wurden von Simanovskii (A.-F.-Ioffe-Institut, St. Petersburg) am
SPEED~2 Experiment aufgebaut \cite{bobashev:97}.
\par
Die Vielschichtspiegel bestehen aus 50 bis 100 Atomlagen W/Sc auf
einem \wert{0.5}{mm} \wert{\cdot\ 20}{mm} \wert{\cdot 50}{mm}
Siliziumsubstrat. Durch den Wechsel zwischen Wolfram und Scandium
entstehen Schichten mit unterschiedlichem Brechungsindex. Die
Absorption bei weicher Röntgenstrahlung (\wert{1.21-1.35}{nm}) ist
klein genug, daß die Strahlung die Schichten durchdringen kann. Na
jeder Schichtgrenze wird ein Teil der Strahlung reflektiert. Es
kommt zu Interferenzeffekten zwischen den reflektierten Wellen,
die u.a. vom Winkel der einfallenden Strahlen zum Spiegel abhängig
sind (Bragg-Reflexion). Für den genutzten
Wel"-len"-län"-gen"-be"-reich liegt der Winkel für eine
konstruktive Interferenz im Bereich 31 - 35$^o$. Die Reflektivität
beträgt dabei 0.7\% für einen Spiegel.
\par
Eine Abbildung wird erreicht durch Krümmung der ursprünglich
ebenen Spiegel. Die Anordnung der beiden MLMs und der
resultierende Strahlengang ist in der Abbildung \vref{fig:MLM}
schematisch dargestellt. Der Abstand zwischen den Spiegeln beträgt
\wert{1.6}{cm}, der Abstand der MLMs zum Plasmafokus \wert{1.2}{m}
und zur MCP \wert{0.6}{m}. Die Krümmungsradien der Spiegel
(\wert{1.8}{m} und \wert{0.5}{m}) sind so gewählt, daß beide
Spiegel zusammen, wie ein Hohlspiegel mit Krümmungsradius
\wert{1}{m} wirken \cite{chen:90}.
\par
Mit dieser Anordnung wird eine räumliche Auflösung von ca.
\wert{1}{mm} und eine spektrale Auflösung von $\lambda /
\Delta\lambda = 50-100$ erreicht.
%
\par
\begin{figure}[H]
  \center
  \fbox{\importimage{C4-11}}
  \caption{Schematische Darstellung des MLM-Systems am SPEED~2 \cite{bobashev:97}}
  \label{fig:MLM}
\end{figure}
%
\par
Vor dem MLM-System wurde ein Filter (Be \wert{10}{$\mu$m} dick)
angebracht. Dieser Filter ist nicht durchlässig für UV-Strahlung, die
von den Spiegeln wellenlängenunabhängig reflektiert wird. Der Filter
dient auch als Vakuumdichtung gegen die Entladungskammer, (Fülldruck
typisch \wert{500}{Pa}) damit, das für die MCP notwendige Vakuum (unter
\ewert{1}{-3}{Pa}) mit einer kleinen Turbo-Molekularpumpe erreicht
werden kann.
\par
Eine MCP wird benötigt, weil der Abstand zur Röntgenquelle relativ
groß ist und die Reflektivität der Spiegel gering ist (verglichen
mit Spiegeln für den sichtbaren Wel"-len"-län"-gen"-be"-reich).
Der Abstand konnte nicht (wesentlich) verkleinert werden, weil das
Spiegelsystem mit MCP und Vakuumpumpe in einem Faraday-Käfig
isoliert von der Entladungskammer aufgestellt werden muß. Benutzt
wurde eine doppelte MCP mit hoher Verstärkung. Die Belichtungszeit
betrug auch bei dieser MCP \wert{10}{ns}, weil sie zusammen mit
der MCP des Spektrometers am gleichen Pulsgenerator betrieben
wurde.
\par
Die Bilder wurden auf einem Kleinbildfilm Fujifilm Neopan
Professional 1600 (ISO 1600, schwarz/weiß) aufgenommen.
