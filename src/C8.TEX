%
%   Ausblick
%   + Isolatorproblem
%   + Injektionsgasverteilung
%   + Spektroskopische Untersuchungen
%   + Messung Energie/Leistung der Röntgenstrahlung
%   + kleine Geometrie -> kl. Spannung/Energie
%   + andere Experimente -> SPEED3
%   - Linienverstärkung in der Säule
%   + Laserinduzierte Plasmen
%   + theoretische Beschreibung
%
\beginsection{Ausblick}
\label{sec:ausblick}
%
In dieser Arbeit wurden wesentliche experimentelle Daten über den
stabilen Säu"-len"-mo"-dus (SCM) ermittelt. Wie bei fast allen
Arbeiten bleiben auch hier noch einige Lücken für weitere
Untersuchungen.
\par
Die spektroskopischen Untersuchungen zur zeitaufgelösten und
ortsaufgelösten Messung von Dichte und Temperatur werden zur Zeit
durchgeführt. Auch stehen noch genauere, spektral-aufgelöste Messungen
der Energie oder der Leistung der Röntgenstrahlung aus.
\par
Die Hoch-Z-Ionen im Pinch stammen aus dem injizierten Gas.
Axial-aufgelöste Messungen der Gasdichte in Ab"-hän"-gig"-keit der
Injektionszeit fehlen noch für eine genauere Angabe des Anteils
des injizierten Gases im resultierenden Pinchplasma. Die
Röntgenabsorptionsmessungen von Mälzig am injizierten Gasstrahl
könnten dazu weitergeführt und ergänzt werden.
\par
Der SCM verlangt eine hohe Energiedichte. Der Aufbau von elektrischen
Treibern mit hoher Bankenergie und Spannung ist durch das modulare
Konzept des Marx-Generators und durch modulare Kombination von
Marx-Modulen relativ einfach, verglichen mit dem Aufbau des Fokus. Der
für die Zündung entscheidende Isolator wird mit wachsender
Energiedichte zum Hauptproblem. Die Entwicklung von Plasmafokusanlagen
höherer Energie erscheint daher schwierig. Neue und erfolgreiche Ideen
zur Untersuchung des Zündvorgangs auf der Isolatoroberfläche und zum
Isolatorproblem würden die Plasmafokusanlagen einen entscheidenden
Schritt weiterbringen.
\par
Bei kleineren Energien ist der Fokusbetrieb weniger kritisch, daher
sind Untersuchungen an SPEED~2 mit reduzierter Bankenergie und
verkleinertem Anodendurchmesser geplant. Dabei besteht die Erwartung,
den SCM zu erreichen, weil die hohe Energiedichte erhalten bleibt. Daß
ein Umbau auf einen kleineren Radius erforderlich sein kann, haben die
Versuche mit reduzierte Bankenergie in dieser Arbeit bereits
angedeutet. Der Bereich effektiver Pinchbildung wurde, bei dieser
Geometrie, bei Reduzierung der Bankenergie zu schnell für eine sichere
Aussage zur Skalierung verlassen.
\par
In der Arbeitsgruppe wird auch versucht, den SCM an kleineren
Experimenten zu realisieren. Zur Zeit laufen Experimente an der
kleineren SPEED~3-Anlage. Das Table-Top-System SPEED~4 ist bereits
für den Experimentierbetrieb einsatzbereit. Diese Experimente sind
durch ihre wesentlich reduzierte Größe eher geeignet industriell,
als Rönt"-gen"-strah"-lungs"-quel"-le eingesetzt zu werden. Ein
weiterer wesentlicher Vorteil dieser Systeme ist ihre höhere
Entladungsfrequenz von typisch 100 Entladungen/Tag gegen typisch
50 Entladungen/Woche bei SPEED~2. Noch ist aber die
Reproduzierbarkeit bei den kleineren Anlagen um ein Vielfaches
schlechter als bei SPEED~2, was diese Zahlen wieder relativiert.
\par
Ein großer Nachholbedarf besteht bei den theoretischen Arbeiten zum
SCM. Ein Ansatz für einen Stabilisierungsmechanismus wurde in dieser
Arbeit angegeben, aber dieser ist noch weiter auszubauen. Berechnungen
zu den Parameterbereichen der beiden Betriebsmodi, die bei einer
Skalierung auf andere Maschinen Voraussagen ermöglichen, fehlen. Ebenso
fehlt die Theorie zur Konditionierung der Isolatoren und des
Zündvorganges auf der Isolatoroberfläche. Diese würde bei der
experimentell sehr aufwendigen Suche nach dem optimalen Isolator
helfen.
\par
Es bleiben also noch genügend experimentelle Fragestellungen offen, um
die Plasmafokusanlage SPEED~2 die nächsten Jahre auszulasten. Eine
erfolgreiche Realisierung des SCM auf kleineren Maschinen bleibt zu
erhoffen, ebenso die theoretische Behandlung und Modellierung der
Fokusentladung, insbesondere mit dem Wechsel zwischen den Betriebsmodi.
