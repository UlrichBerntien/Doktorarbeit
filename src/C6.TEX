%
%  Untersuchung des stabilen Säu"-len"-mo"-dus
%
\beginsection{Untersuchung des stabilen Säulenmodus}
\label{sec:untersuchung}
%
Im Kapitel \ref{sec:betriebsparameter} wurde gezeigt, das der SCM und
der MPM zuverlässig über die Betriebsparameter eingestellt werden kann.
Experimentell liefern die untersuchten Abhängigkeiten zwischen den
eingestellten Parametern und dem resultierenden Entladungsmodus wenig
Informationen für die Erklärung des SCM. Daher wurden mehr Diagnostiken
eingesetzt, um die Entstehung der Säule zu untersuchen.
\par
Es wird auch eine andere Vorgehensweise als im Kapitel
\ref{sec:betriebsparameter} eingesetzt. Nicht mehr komplette Meßreihen
werden als ganzes betrachtet, sondern spezielle Entladungen werden
herausgesucht, um an diesen die als grundlegend erkannten Effekte zu
präsentieren. Dabei werden natürliche viele aufgenommene Meßdaten, an
dieser Stelle, nicht dokumentiert. (Der Umfang der vorliegenden Daten
ließe ein solchen Versuch auch nicht zu.) Die Selektion erfolgt zum
Teil subjektiv, aufgrund der Erfahrung im Umgang mit dem Experiment und
den Diagnostiken, aber auch objektiv aufgrund des zu zeigenden Effekts,
bereinigt von den Effekten der Schuß-zu-Schuß Schwankungen durch die
Isolatorproblematik. Unter anderen Gesichtspunkten lassen sich auch
andere Zusammenstellungen der vorzuführenden Entladungen finden. In
diesem Sinn wird bei den meisten Entladungen verkürzend von \glqq
typischen Entladungen\grqq\ gesprochen.
%
\beginsubsection{Frühe Pinchphase}
%
\par
Bei den typischen Entladungen sind zu Beginn der Kompressionsphase
keine wesentlichen Unterschiede zwischen dem SCM und dem MPM
sichtbar. Da die Gasinjektion den Entladungsmodus mitbestimmt,
können sich die Unterschiede erst am Ende der Kompression
ausbilden. Die Filamente (\wert{-10}{ns}, \wert{-5}{ns}) treten im
SCM nicht auf.
\par
Die Abbildung \vref{fig:MCP:anfang} zeigt die Kompressionsphase
der Fokusentladung. Die Bilder wurden mit einer MCP ohne Filter
aufgenommen (\wert{\lambda < 20}{nm}). Die Zeiten der einzelnen
Bilder sind relativ zum Zeitpunkt der maximalen Kompression
angegeben.
\par
\wert{40}{ns} vor der maximalen Kompression hat die Schicht einen
Durchmesser von ca. \wert{1.5}{cm}. Vor diesem Zeitpunkt ist ihre
Helligkeit erheblich geringer, weil noch kein Hoch-Z-Gas von der
Schicht aufgenommen wurde. In dieser Phase läuft die Schicht mit
typisch \ewert{3}{5}{m/s} auf die Achse zu.
%
\par
\begin{figure}[H]
  \center
  \fbox{\importimage{C6-05}}
  \caption{Kompressionsphase aufgenommen mit der MCP $\lambda <$ (\wert{20}{nm}).
     Belichtungszeit \wert{3}{ns}, Zeitangaben relativ zur maximalen Kompression.}
  \label{fig:MCP:anfang}
\end{figure}
%
\par
\begin{table}[H]
  \center
  \begin{tabular}{|l|c|c|}
  \hline
                               & \wert{-40}{ns} und \wert{-20}{ns} & \wert{-10}{ns} und \wert{-5}{ns} \\
  \hline
    Ladespannung U             & \multicolumn{2}{c|}{ \wert{180}{kV} }     \\
    Bankenergie E              & \multicolumn{2}{c|}{ \wert{55}{kJ} }      \\
    Füllgas                    & \multicolumn{2}{c|}{ Deuterium 2.7 }      \\
    Fülldruck p(D$_2$)         & \multicolumn{2}{c|}{ \wert{3.5}{hPa} }    \\
    Injektionsgas              & \multicolumn{2}{c|}{ Argon 4.6 }          \\
    Injektionsdruck p(Ar)      & \multicolumn{2}{c|}{ \ewert{3.5}{5}{Pa} } \\
    Injektionszeit \teff       & \multicolumn{2}{c|}{ \wert{1.5}{ms} }     \\
    Nummer                     & 6.651        & 6.705                      \\
  \hline
  \end{tabular}
  \caption{Parameter der Entladungen in Abbildung \ref{fig:MCP:anfang}}
  \label{tab:MCP:anfang}
\end{table}
%
\par
Im MPM verhindern die m=0-Instabilitäten die Bildung einer
homogenen Plasmasäule. Die Abbildung \vref{fig:MCP:mpm} zeigt die
Pinchphase einer typischen Entladung im MPM. Die Einschnürungen
entwickeln sich im ns-Bereich. In ihnen kann es zu einem lokalen
Strahlungskollaps kommen, der durch seine intensive
Röntgenstrahlung mit der Rönt"-gen"-pin"-hole"-kame"-ra sichtbar
wird (rechtes Teilbild).
\par
Die Bildung der Mikropinche beginnt in Anodennähe. Nicht alle
Einschnürungen führen dabei zu Mikropinchen. Die Position und die
Anzahl unterliegen starken Schwankungen, weil sie durch die
zufälligen Instabilitäten ausgelöst werden \cite{roewe:phd}.
%
\par
\begin{figure}[H]
  \center
  \fbox{\importimage{C6-06}}
  \caption{Bildung von Mikropinchen aus Instabilitäten.
    Die drei linken Aufnahmen sind MCP-Bilder (ungefiltert,
    \wert{\lambda < 20}{nm}) mit einer Belichtungszeit von \wert{3}{ns}.
    Die rechte Aufnahme ist ein zeitintegriertes Pinholebild
   (\wert{130}{$\mu$m} Be Filter, \wert{\lambda < 0.7}{nm}).}
  \label{fig:MCP:mpm}
\end{figure}
%
\par
\begin{table}[H]
  \center
  \begin{tabular}{|l|c|}
  \hline
    Ladespannung U             & \wert{180}{kV}    \\
    Bankenergie E              & \wert{57}{kJ}     \\
    Füllgas                    & Deuterium 2.7     \\
    Fülldruck p(D$_2$)         & \wert{3.7}{hPa}   \\
    Injektionsgas              & Argon 4.6         \\
    Injektionsdruck p(Ar)      & \ewert{3.5}{5}{Pa}\\
    Injektionszeit \teff       & \wert{1.5}{ms}    \\
    Nummer                     & 6.641             \\
  \hline
  \end{tabular}
  \caption{Parameter der Entladung in Abbildung \ref{fig:MCP:mpm}}
  \label{tab:MCP:mpm}
\end{table}
%
\par
Im SCM zeigt sich ein völlig anderes Verhalten in der Pinchphase.
Nach dem Wechsel des Injektionsgases von Argon nach Neon (die
anderen Betriebsparameter in der Tabelle \vref{tab:MCP:scm})
arbeitet die Anlage im SCM. Die Abbildung \vref{fig:MCP:scm} zeigt
die Entwicklung der Plasmasäule unter diesen Bedingungen.
\par
Innerhalb der dargestellten \wert{10}{ns} sind keine Anzeichen
einer Instabilität erkennbar. Größe und Form der Plasmasäule sind
nahezu konstant. Das Maximum der Helligkeit der Säule schiebt sich
langsam von der Anode weg. Diese Stabilität zeichnet den SCM aus,
ganz im Gegensatz zum CM anderer Anlagen (vgl. Abschnitt
\vref{sec:vergleich}). Auf dem zeitintegrierten Bild der
Rönt"-gen"-pin"-hole"-kame"-ra (\wert{\lambda < 2}{nm}, in der
Abbildung rechts) wird die Säulenform sichtbar.
%
\par
\begin{figure}[H]
  \center
  \fbox{\importimage{C6-07}}
  \caption{Pinchsäule im SCM ohne Instabilitäten.
     Die drei linken Aufnahmen sind MCP-Bilder (ungefiltert,
     \wert{\lambda < 20}{nm}) mit einer Belichtungszeit von \wert{3}{ns}.
     Die rechte Aufnahme ist ein zeitintegriertes Pinholebild
     (\wert{10}{$\mu$m} Be Filter, \wert{\lambda < 2}{nm}).}
  \label{fig:MCP:scm}
\end{figure}
%
\par
\begin{table}[H]
  \center
  \begin{tabular}{|l|c|}
  \hline
    Ladespannung U             & \wert{180}{kV}    \\
    Bankenergie E              & \wert{59}{kJ}     \\
    Füllgas                    & Deuterium 2.7     \\
    Fülldruck p(D$_2$)         & \wert{3.4}{hPa}   \\
    Injektionsgas              & Neon 4.0          \\
    Injektionsdruck p(Ne)      & \ewert{3.0}{5}{Pa}\\
    Injektionszeit \teff       & \wert{1.5}{ms}    \\
    Nummer                     & 9.491             \\
  \hline
  \end{tabular}
  \caption{Parameter der Entladung in Abbildung \ref{fig:MCP:scm}}
  \label{tab:MCP:scm}
\end{table}
\par
%
Die Aufnahmen der MCP-Kamera geben einen breiten
Wel"-len"-län"-gen"-be"-reich im VUV- und SXR-Bereich wieder. Mit
der MLM-Optik lassen sich Bilder aus einem schmalen
Wellenlängenintervall ($\lambda/\Delta\lambda = 50-100$) gewinnen.
Die Intensität der Röntgenstrahlung ist durch die doppelte
Reflexion in der Spiegeloptik so weit abgeschwächt, daß eine
schlechtere Bildqualität resultiert.
\par
Die Abbildung \vref{fig:MLM:Kompression} zeigt eine Sequenz von
MLM-Aufnahmen bei \wert{\lambda = 0.85}{nm} in dem Bereich der
Kontinuumsstrahlung des Neon/Deuterium-Plasmas. Die Aufnahmen
zeigen die Entstehung der Säule im SCM. Der anfängliche
Hohlzylinder (\wert{t = -30}{ns}) wird, von der Anode beginnend
(\wert{t = -10}{ns}), zur stabilen Pinchsäule (\wert{t = +30}{ns})
komprimiert. Hier wird ein Durchmesser von ca. \wert{4}{mm}
erreicht.
%
\par
\begin{figure}[H]
  \center
  \fbox{\importimage{C6-10}}
  \caption{Bildung der stabilen Plasmasäule. Bilder aufgenommen mit dem
     MLM-System bei \wert{\lambda = 0.85}{nm}. Die Zeiten sind bezogen
     auf den Zeitpunkt der maximalen Kompression. Die Belichtungszeiten sind
     \wert{10}{ns}.}
  \label{fig:MLM:Kompression}
\end{figure}
%
%
\par
\begin{table}[H]
  \center
  \begin{tabular}{|l|c|c|c|}
  \hline
                               & \wert{-30}{ns} & \wert{-10}{ns} & \wert{+30}{ns} \\
  \hline
    Ladespannung U             & \multicolumn{3}{c|}{ \wert{180}{kV} }     \\
    Bankenergie E              & \multicolumn{3}{c|}{ \wert{62}{kJ} }      \\
    Füllgas                    & \multicolumn{3}{c|}{ Deuterium 2.7 }      \\
    Fülldruck p(D$_2$)         & \wert{8.2}{hPa} & \wert{7.9}{hPa} & \wert{8.2}{hPa} \\
    Injektionsgas              & \multicolumn{3}{c|}{Neon 4.0}             \\
    Injektionsdruck p(Ne)      & \multicolumn{3}{c|}{ \ewert{9.0}{5}{Pa} } \\
    Injektionszeit \teff       & \multicolumn{3}{c|}{ \wert{1.5}{ms} }     \\
    Nummer                     & 12.574      & 12.582   & 12.576           \\
  \hline
  \end{tabular}
  \caption{Parameter der Entladungen in Abbildung \ref{fig:MLM:Kompression}}
  \label{tab:MLM:Kompression}
\end{table}
%
\par
Auch bei diesen Aufnahmen ist die gute Homogenität der Säule
sichtbar. Die gute Reproduzierbarkeit der Entladungen im SCM
ermöglicht die Kombination der Aufnahmen von verschiedener
Entladungen zu einer Bildsequenz.
\par
Weitere Informationen über das Plasma können mit der Spektroskopie
gewonnen werden. Erste Messungen zur Bestimmung der Elektronendichte
und Elektronentemperatur werden hier vorgestellt.
\par
Die Abbildung \vref{fig:spektrum} zeigt ein Spektrum, aufgenommen
\wert{20}{ns} vor der maximalen Kompression, mit einer
Belichtungszeit von \wert{10}{ns}. Aus dem Abfall der Intensität
zu kleinen Wellenlängen hin, folgt (mit Beachtung der
Empfindlichkeitskurven von MCP und Scanner) die
Elektronentemperatur zu \wert{\rm kT_{\rm e} \approx 300}{eV}. Die
Elektronendichte kann u.a. aus der Starkverbreiterung bestimmt
werden. Aus der Linie vom Übergang n=8 nach n=1 (die höchste
sichtbare Linie) folgt die Elektronendichte zu \ewert{\rm n_{\rm
e} = 2.5}{26}{m$^{-3}$}.
%
\par
\begin{figure}[H]
  \center
  \fbox{\importimage{C6-08}}
  \caption{Das Spektrum, aufgenommen mit dem Kristallspektrometer, zeigt
     Linien von wasserstoffähnlichem Neon. Belichtungszeit \wert{10}{ns}
     um den Zeitpunkt \wert{t=-20}{ns}.}
  \label{fig:spektrum}
\end{figure}
%
\par
\begin{table}[H]
  \center
  \begin{tabular}{|l|c|}
  \hline
    Ladespannung U             & \wert{180}{kV}    \\
    Bankenergie E              & \wert{64}{kJ}     \\
    Füllgas                    & Deuterium 2.7     \\
    Fülldruck p(D$_2$)         & \wert{8.2}{hPa}   \\
    Injektionsgas              & Neon 4.0          \\
    Injektionsdruck p(Ne)      & \ewert{9.0}{5}{Pa}\\
    Injektionszeit \teff       & \wert{1.5}{ms}    \\
    Nummer                     & 12.573            \\
  \hline
  \end{tabular}
  \caption{Parameter der Entladung in Abbildung \ref{fig:spektrum}}
  \label{tab:spektrum}
\end{table}
%
\par
Dichte und Temperatur werden in den noch folgenden \wert{20}{ns} der
Kom\-pres\-sions\-phase sicherlich weiter steigen, weil der
Pinchdurchmesser um ca. den Faktor 2 abnimmt. Entsprechende
spektroskopische Messungen werden zur Zeit durchgeführt.
%
\beginsubsection{Pinchphase}
%
\par
Die Dynamik der Plasmasäule kann gut mit der Streakkamera
untersucht werden. Die Streakkamera liefert räumlich nur eine
1-dimensionale Auflösung, aber bei der einfachen Struktur der
Strahlungsquelle reicht diese aus.
\par
Das Zeitfenster der verwendeten Streakkamera (vgl. Abschnitt
\vref{sec:streakkamera}) ist maximal \wert{40}{ns} groß, daher
müssen in der Regel mehrere Aufnahmen von verschiedenen
Entladungen zusammen betrachtet werden, um die Säule während ihrer
gesamten Lebensdauer zu beobachten. Dieses ist aufgrund der guten
Reproduzierbarkeit der Entladung im SCM möglich. Die
Reproduzierbarkeit wurde durch Aufnahmen zu gleichen Zeiten und
durch überlappende Zeitfenster überprüft.
%
\par
\begin{figure}[H]
  \center
  \fbox{\importimage{C6-01}}
  \caption{Gestreakte Schnitte durch die Pinchsäule mit unterschiedlichen
      Filtern aufgenommen. Obere Hälfte mit \wert{135}{$\mu$m} Be gefiltert
      (\wert{\lambda < 0.8}{nm}), untere Hälfte mit \wert{10}{$\mu$m}
      Be gefiltert (\wert{\lambda < 2}{nm}). Durch den Abbildungsfehler der
      Streakkamera kommt es zur scheinbaren Kompression im mittleren Bild.}
  \label{fig:schnitteA}
\end{figure}
%
%
\par
\begin{table}[H]
  \center
  \begin{tabular}{|l|c|c|c|}
  \hline
                               & oben        & mitte      & unten          \\
  \hline
    Ladespannung U             & \multicolumn{3}{c|}{ \wert{180}{kV} }     \\
    Bankenergie E              & \multicolumn{3}{c|}{ \wert{66}{kJ} }      \\
    Füllgas                    & \multicolumn{3}{c|}{ Deuterium 2.7 }      \\
    Fülldruck p(D$_2$)         & \multicolumn{3}{c|}{ \wert{9.2}{hPa} }    \\
    Injektionsgas              & \multicolumn{3}{c|}{ Neon 4.0 }           \\
    Injektionsdruck p(Ne)      & \multicolumn{3}{c|}{ \ewert{5.0}{5}{Pa} } \\
    Injektionszeit \teff       & \multicolumn{3}{c|}{ \wert{3.0}{ms} }     \\
    Nummer                     & 12.352      & 12.353   & 12.360           \\
  \hline
  \end{tabular}
  \caption{Parameter der Entladungen in Abbildung \ref{fig:schnitteA}}
  \label{tab:schnitteA:para}
\end{table}
%
\par
Die Abbildung \vref{fig:schnitteA} kombiniert drei Aufnahmen der
Streakkamera. Der \wert{0.55}{mm} breite Spalt der Kamera wurde so
gelegt, daß eine radiale Auflösung der Plasmasäule sichtbar wird.
Die Skizze rechts oben in der Abbildung deutet die Lage des
Spaltes \wert{8}{mm} oberhalb der Anode an. Die Plasmasäule wurde
mit zwei Pinholes (\wert{\phi = 250}{$\mu$}m) und
unterschiedlichen Filtern auf die Spaltebene abgebildet. Damit
wird der gleiche Schnitt durch die Pinchsäule in zwei
verschiedenen Wellenlängenbereichen (\wert{\lambda < 0.8}{nm} und
\wert{\lambda < 2}{nm}) dargestellt.
\par
Die sichtbaren Kompression bei der mittleren Streakaufnahme ist
durch den Abbildungsfehler des Bildverstärkers (siehe Abschnitt
\ref{sec:streakkamera}) bedingt. Der Durchmesser am Rand und in
der Mitte ist bis auf \wert{0.1}{mm} gleich.
%
\par
\begin{figure}[H]
  \center
  \fbox{\importimage{C6-02}}
  \caption{Gestreakte Schnitte durch die Pinchsäule mit unterschiedlichen
     Filtern. Obere Hälfte \wert{\lambda < 0.8}{nm}, \wert{135}{$\mu$m}
     Be Filter; untere Hälfte \wert{\lambda < 2}{nm}, \wert{10}{$\mu$m}
     Be Filter.}
  \label{fig:schnitteB}
\end{figure}
%
\par
\begin{table}[H]
  \center
  \begin{tabular}{|l|c|c|}
  \hline
                               & oben            & unten                   \\
  \hline
    Ladespannung U             & \multicolumn{2}{c|}{ \wert{180}{kV} }     \\
    Bankenergie E              & \multicolumn{2}{c|}{ \wert{66}{kJ} }      \\
    Füllgas                    & \multicolumn{2}{c|}{ Deuterium 2.7 }      \\
    Fülldruck p(D$_2$)         & \multicolumn{2}{c|}{ \wert{9.4}{hPa} }    \\
    Injektionsgas              & \multicolumn{2}{c|}{ Neon 4.0 }           \\
    Injektionsdruck p(Ne)      & \multicolumn{2}{c|}{ \ewert{5.0}{5}{Pa} } \\
    Injektionszeit \teff       & \multicolumn{2}{c|}{ \wert{7.5}{ms} }     \\
    Nummer                     & 12.368        & 12.369                    \\
  \hline
  \end{tabular}
  \caption{Parameter der Entladungen in Abbildung \ref{fig:schnitteB}}
  \label{tab:schnitteB:para}
\end{table}
%
\par
Der Pinch wird im kürzeren Wel"-len"-län"-gen"-be"-reich
\wert{5-10}{ns} später sichtbar. Die höchste Temperatur wird in
einer Zeitdauer von \wert{\le 25}{ns} erreicht. Im Bereich
\wert{\lambda < 2}{nm} ist die Säule \wert{90}{ns} lang sichtbar.
Der Durchmesser der Quelle liegt dabei zwischen \wert{5.5}{mm} und
\wert{4.7}{mm}. Der Durchmesser der Quelle bei \wert{\lambda <
0.8}{nm} ist mit \wert{2.3}{mm} deutlich kleiner.
\par
Die Lebensdauer und der Durchmesser der Plasmasäule ist abhängig
vom Neonanteil im Plasma.
\par
Die Streakbilder in Abbildung \vref{fig:schnitteB} wurden bei
Entladungen mit größerem Neonanteil aufgenommen. Der Neonanteil wurde
durch eine längere Injektionszeit (\teff \wert{= 7.5}{ms} gegen
\wert{3.0}{ms}) erreicht. Die anderen Betriebsparameter wurden
beibehalten, siehe dazu Tabelle \vref{tab:schnitteA:para} und
\vref{tab:schnitteB:para}. Die Liniendichte des Neons hat sich dabei
von geschätzten \ewert{\rm n(Ne) = 6.8}{19}{m$^{-1}$} auf \ewert{\rm n(Ne) =
1.7}{20}{m$^{-1}$} erhöht.
%
\par
\begin{figure}[H]
  \center
  \fbox{\importimage{C6-03}}
  \caption{Gestreakte Schnitte durch die Pinchsäule bei unterschiedlichen
     Höhen (\wert{3}{mm} und \wert{13}{mm}) über der Anode gefiltert mit
     \wert{10}{$\mu$m} Be + \wert{0.5}{$\mu$m} Cu (\wert{\lambda < 2}{nm}).
     Das Plasma wird bei größerem Abstand zur Anode später sichtbar.}
  \label{fig:schnitteC}
\end{figure}
%
\par
\begin{table}[H]
  \center
  \begin{tabular}{|l|c|c|}
  \hline
                               & oben            & unten                   \\
  \hline
    Ladespannung U             & \multicolumn{2}{c|}{ \wert{180}{kV} }     \\
    Bankenergie E              & \multicolumn{2}{c|}{ \wert{64}{kJ} }      \\
    Füllgas                    & \multicolumn{2}{c|}{ Deuterium 2.7 }      \\
    Fülldruck p(D$_2$)         & \multicolumn{2}{c|}{ \wert{6.0}{hPa} }    \\
    Injektionsgas              & \multicolumn{2}{c|}{ Neon 4.0 }           \\
    Injektionsdruck p(Ne)      & \multicolumn{2}{c|}{ \ewert{4.0}{5}{Pa} } \\
    Injektionszeit \teff       & \multicolumn{2}{c|}{ \wert{3.0}{ms} }     \\
    Nummer                     & 12.470        & 12.463                    \\
  \hline
  \end{tabular}
  \caption{Parameter der Entladungen in Abbildung \ref{fig:schnitteC}}
  \label{tab:schnitteC:para}
\end{table}
%
\par
Die Vergrößerung des Neonanteils führte zu einer kleineren
Lebensdauer: von \wert{90}{ns} auf \wert{60}{ns} bei \wert{\lambda
< 2}{nm} und von \wert{25}{ns} auf \wert{20}{ns} bei \wert{\lambda
< 0.8}{nm}. Der Durchmesser der Quelle wurde dabei fast halbiert:
von \wert{4.7}{mm} auf \wert{2.9}{mm} bei \wert{\lambda < 2}{nm}
und von \wert{2.3}{mm} auf \wert{1.2}{mm} bei \wert{\lambda <
0.8}{nm}. Die Verzögerung zwischen dem Einsetzen der Strahlung bei
\wert{\lambda < 2}{nm} und \wert{\lambda < 0.8}{nm} hat sich
ebenfalls verkleinert.
\par
Eine weitere Erhöhung des Neonanteils führt zum Übergang vom SCM
in den MPM.
\par
Bei radial-aufgelösten Streaks können nicht nur unterschiedliche
Wellenlängenbereiche untersucht werden, es können auch
unterschiedliche Schnitte durch die Pinchsäule betrachtet werden.
Dabei ist es in der Regel sinnvoll, die beiden abbildenden
Pinholes mit gleichen Filtern auszustatten.
\par
Die Abbildung \vref{fig:schnitteC} zeigt solche Streakaufnahmen.
Die Schnitte durch die Plasmasäule wurden in \wert{13}{mm} und
\wert{3}{mm} Abstand zur Anode gelegt, wie die Skizze in der
Abbildung andeutet. Benutzt wurden ein Filter \wert{10}{$\mu$m} Be
+ \wert{0.5}{$\mu$m} Cu mit \wert{\lambda < 2}{nm} und ein
\wert{0.55}{mm} breiter Spalt.
\par
Beim Abstand \wert{3}{mm} ist die Bildung der heißen Pinchsäule
besonders gut sichtbar. Die Kompression erfolgt immer noch mit einer
Radialgeschwindigkeit von \ewert{1}{5}{m/s}, die schon auf den frühen
MCP-Bildern (vgl. Abbildung \vref{fig:MCP:anfang}) ablesbar war. Die
Lebensdauer der Quelle ist mit \wert{30}{ns} deutlich kleiner als bei
der Höhe \wert{8}{mm} und bei vergleichbaren Betriebsparameter (Tabelle
\ref{tab:schnitteA:para} und \ref{tab:schnitteC:para}). Beim Abstand
\wert{13}{mm} setzt die Strahlung um \wert{20}{ns} verzögert ein, was
einer Geschwindigkeit von \ewert{5}{5}{m/s} entspricht. Der Durchmesser
der Quelle nimmt in Richtung Anode leicht zu: von \wert{3}{mm} bei
\wert{h = 13}{mm} auf \wert{4}{mm} bei \wert{h = 3}{mm}.
%
\beginsubsection{Späte Pinchphase}
\label{sec:zweiteKompression}
%
\par
Neben der radialen Auflösung der Plasmasäule, ist auch eine axiale
Auflösung möglich. Bei Filterung mit \wert{0.5}{mm} Beryllium
(\wert{\lambda < 0.5}{nm}) ist nur der heiße Kern der Säule sichtbar.
Der Durchmesser der sichtbaren Säule (\wert{\le 2}{mm}) wird durch die
Abbildung, mit dem Pinhole (\wert{\phi = 300}{$\mu$m}), auf die
Spaltebene der Streakkamera auf \wert{\le 0.8}{mm} reduziert. Daher
kann auf einen schmalen Spalt verzichtet werden und mit der ganzen
Fläche der Photokathode (Breite ca. \wert{5}{mm}) gearbeitet werden,
wodurch die Justierung erheblich vereinfacht wird.
\par
Die Abbildung \vref{fig:laengs} zeigt verschiedene
axial-aufgelöste Streakaufnahmen von der Plasmasäule. Bei der
Anode beginnend wächst die Quelle mit \ewert{3}{5}{m/s} bis auf
eine Länge von ca. \wert{9}{mm}. Über einen Zeitraum von
\wert{15-25}{ns} existiert eine \wert{5-9}{mm} lange Säule im
Bereich \wert{\lambda < 0.5}{nm}. Die Intensität der Quelle
schwankt von Entladung zu Entladung, aber die Lebensdauer und die
Größe bleibt ähnlich.
\par
Auffällig ist bei einigen Entladungen ein zweiter
Strahlungsausbruch nach typisch \wert{30}{ns}. Selten werden bis
zu drei dieser intensiven Strahlungsquellen beobachtet. Wenn sie
auftreten, dann in Anodennähe, bei einem maximalen Abstand von
\wert{7}{mm} zur Anode.
%
\par
\begin{figure}[H]
  \center
  \fbox{\importimage{C6-04}}
  \caption{Axial-aufgelöste Streaks der Pinchsäule, gefiltert mit
     \wert{0.5}{mm} dickem Beryllium-Filter (\wert{\lambda < 0.5}{nm}).
     Diese Bilder wurden wie auf Seite \pageref{streaknachbearbeitung}
     beschrieben nachbearbeitet.}
  \label{fig:laengs}
\end{figure}
%
%
\par
\begin{table}[H]
  \center
  \begin{tabular}{|l|c|c|c|}
  \hline
                               & oben        & mitte      & unten          \\
  \hline
    Ladespannung U             & \multicolumn{3}{c|}{ \wert{180}{kV} }     \\
    Bankenergie E              & \multicolumn{3}{c|}{ \wert{62}{kJ} }      \\
    Füllgas                    & \multicolumn{3}{c|}{ Deuterium 2.7 }      \\
    Fülldruck p(D$_2$)         & \wert{7.9}{hPa} & \wert{8.2}{hPa} & \wert{7.9}{hPa} \\
    Injektionsgas              & \multicolumn{3}{c|}{Neon 4.0}             \\
    Injektionsdruck p(Ne)      & \multicolumn{3}{c|}{ \ewert{9.0}{5}{Pa} } \\
    Injektionszeit \teff       & \multicolumn{3}{c|}{ \wert{1.5}{ms} }     \\
    Nummer                     & 12.606      & 12.609   & 12.611           \\
  \hline
  \end{tabular}
  \caption{Parameter der Entladungen in Abbildung \ref{fig:laengs}}
  \label{tab:laengs:para}
\end{table}
%
\par
Die Abbildung \vref{fig:zweiteKompression} zeigt eine
zeitaufgelöste MLM-Aufnahme und eine zeitintegrierte
Röntgenpinholeaufnahme von dieser Quelle. Das Röntgenpinholebild
ist ebenfalls mit \wert{0.5}{mm} Beryllium (\wert{\lambda <
0.5}{nm}) gefiltert. Die MLM-Optik wurde auf \wert{\lambda =
0.85}{nm} (Kontinuumsstrahlung) eingestellt. Die Aufnahme entstand
\wert{30}{ns} nach der maximalen Kompression mit einer
Belichtungszeit von \wert{10}{ns}.
\par
Auf dem Bild der Rönt"-gen"-pin"-hole"-kame"-ra ist im
wesentlichen die Säu"-len"-struk"-tur sichtbar, weil die Aufnahme
zeitintegriert ist, aber im anodennahen Bereich ist die Schwärzung
des Filmes stärker als in der übrigen Säu"-len"-struk"-tur. Durch
die gewählte Darstellung ist dieses deutlich sichtbar.
\par
Das Bild des MLM-Systems zeigt den Effekt besser, weil es sich um
eine zeitaufgelöste Aufnahme handelt. Auf einer Länge von
\wert{7}{mm} ist die sichtbare Plasmasäule auf einen Durchmesser
von \wert{1}{mm} komprimiert und dabei aufgeheizt worden.
%
\par
\begin{figure}[H]
  \center
  \fbox{\importimage{C6-11}}
  \caption{Links: Aufnahme der MLM-Optik bei \wert{\lambda = 0.85}{nm},
     \wert{t = 30}{ns} nach maximaler Kompression, Belichtungszeit \wert{10}{ns}.
     Rechts: zeitintegrierte Aufnahme der Rönt"-gen"-pin"-hole"-kame"-ra (\wert{0.5}{mm} Be Filter,
     \wert{\lambda < 0.5}{nm}). Die Darstellung ist mit schwarz $\approx$ opt. Dichte
     1, weiß $\approx$ opt. Dichte 0.85 extrem gewählt, damit die anodennahe
     zweite Kompression gut sichtbar ist.}
  \label{fig:zweiteKompression}
\end{figure}
%
\par
\begin{table}[H]
  \center
  \begin{tabular}{|l|c|}
  \hline
    Ladespannung U             & \wert{180}{kV}    \\
    Bankenergie E              & \wert{64}{kJ}     \\
    Füllgas                    & Deuterium 2.7     \\
    Fülldruck p(D$_2$)         & \wert{8.2}{hPa}   \\
    Injektionsgas              & Neon 4.0          \\
    Injektionsdruck p(Ne)      & \ewert{5.0}{5}{Pa}\\
    Injektionszeit \teff       & \wert{2.5}{ms}    \\
    Nummer                     & 12.551            \\
  \hline
  \end{tabular}
  \caption{Parameter der Entladung in Abbildung \ref{fig:zweiteKompression}}
  \label{tab:zweiteKompression}
\end{table}
%
\beginsubsection{Neutronenproduktion}
%
\par
Die Fokusentladung in Deuterium/Hoch-Z-Gas ist nicht nur eine VUV- und
SXR-Quelle, sondern auch eine Neutronenquelle. Die Neutronen entstehen
aus Fusionsprozessen zwischen den Deuteronen (siehe Abschnitt
\vref{sec:neutronendiagnostik}). Mit der
Photomultiplier/Szintillator-Kombination wurden die Neutronen
zeitaufgelöst und mit dem Sil\-ber\-akti\-vie\-rungs\-zähler
zeitintegriert gemessen.
\par
Zur Untersuchung der Unterschiede in der Neutronenproduktion zwischen
dem SCM und dem MPM wurden Messungen durchgeführt, bei denen der
Entladungsmodus durch Änderung der Injektionszeit und des
Injektionsdrucks beeinflußt wurde. Der Anteil des Injektionsgases wurde
auf verschiedene Arten und auf verschiedene Werte eingestellt, um die
Variationen der Entladungen innerhalb eines Modus zu erfassen.
\par
Der Entladungsmodus wurde bei jeder untersuchten Entladung am Bild
der Rönt"-gen"-pin"-hole"-kame"-ra festgestellt. Entladungen mit
zu geringer Strahlung wurden aussortiert, ebenso wurden
Entladungen mit zu geringer Neutronenproduktion anhand des
PM/Szintillator-Signals aussortiert. Die Betriebsparameter der
Meßreihe sind in der Tabelle \vref{tab:neutronen:alle} aufgeführt.
Von den 107 Entladungen waren 48 Entladungen für die Auswertung
verwendbar.
%
\par
\begin{table}[H]
  \center
  \begin{tabular}{|l|c|}
  \hline
    Ladespannung U             & \wert{180}{kV}                          \\
    Bankenergie E              & \wert{67}{kJ} -- \wert{62}{kJ}          \\
    Füllgas                    & Deuterium 2.7                           \\
    Fülldruck p(D$_2$)         & \wert{8.8}{hPa} -- \wert{14}{hPa}       \\
    Injektionsgas              & Argon 4.6                               \\
    Injektionsdruck p(Ar)      & \ewert{5.0}{5}{Pa} -- \ewert{9.5}{5}{Pa}\\
    Injektionszeit \teff       & \wert{0.5}{ms} -- \wert{9.5}{ms}        \\
    Nummer                     & 12.122 -- 12.228                        \\
  \hline
  \end{tabular}
  \caption{Parameter der Entladung, an denen die Neutronenproduktion
     untersucht wurde.}
  \label{tab:neutronen:alle}
\end{table}
%
\par
Die Pulse aus dem Silberaktivierungszähler (side on, Abstand
\wert{2.5}{m}, siehe dazu Abschnitt \vref{sec:neutronendiagnostik})
sind proportional zu den produzierten Neutronen (1 Puls = ca.
\fwert{7}{Neutronen}).
\par
Die Tabelle \vref{tab:neutronen:gruppen} gibt die Mittelwerte der
gezählten Pulse pro Entladung eingeteilt in verschiedene Gruppen. Die
Entladungen im SCM zeigen eine deutlich größere Neutronenproduktion als
die Entladungen im MPM. Der Abstand zwischen den Mittelwerten ist mit
$2.6\ \cdot $ Intervallbreite der Meßunsicherheit deutlich. Die nach
Injektionszeit aufgeteilten Gruppen zeigen eine noch deutlichere
Trennung bei einem Abstand von $2.9\ \cdot $ Intervallbreite der
Meßunsicherheit zwischen den Mittelwerten.
\par
Der erhöhten Neutronenproduktion beim SCM liegt also der Trend
zugrunde, daß die Neutronenproduktion bei Reduzierung des
Injektionsgases ansteigt. Dieser Trend ist nicht nur zwischen den
Entladungsmodi sichtbar, er deutet sich auch innerhalb eines
Entladungsmodus an, aber die großen Schuß-zu-Schuß Schwankungen machen
diesen Zusammenhang nur schwer meßbar.
\par
Die Vergrößerung der Injektionsgasmenge führt also zu einer
Reduzierung der Neutronenproduktion und beim Überschreiten eines
Grenzwertes zum Wechseln vom SCM in den MPM.
%
\par
\begin{table}[H]
  \center
  \begin{tabular}{|l|c|c|}
  \hline
     Gruppe                                 & Anzahl  & Mittelwert der Pulse \\
  \hline
     alle Entladungen                       & 48      & 6346 $\pm$ 442 \\
  \hline
     Entladungen im SCM                     & 23      & 7860 $\pm$ 541 \\
     Entladungen im MPM                     & 25      & 4954 $\pm$ 565 \\
  \hline
     Injektionszeit \teff \wert{< 3.5}{ms}  & 28      & 6290 $\pm$ 513 \\
     Injektionszeit \teff \wert{\ge 3.5}{ms}& 20      & 3079 $\pm$ 582 \\
  \hline
  \end{tabular}
  \caption{Mittelwerte der Pulse aus dem Silberaktivierungszähler. Vergleich von verschiedenen
     Gruppen der ausgewählten 48 Entladungen mit Argon-Injektion.}
  \label{tab:neutronen:gruppen}
\end{table}
%
\par
Ein Blick auf die Histogramme in den Abbildungen
\vref{fig:neutronen:statistikA} bis \ref{fig:neutronen:statistikC}
zeigt die große Variation der gezählten Pulse. Dabei handelt es
sich nicht nur um die Schuß-zu-Schuß Schwankungen, es sind auch
die Änderungen in den Betriebsparametern beteiligt.
\par
Die Bereiche ähnlicher Neutronenproduktion im SCM und MPM
über"-wie"-gen, aber dennoch ist der Trend zu höheren Werten im
SCM erkennbar. Die Entladung mit maximaler Pulsanzahl erfolgte im
SCM und die Entladungen mit minimaler Pulsanzahl erfolgten im MPM.
Das häufigste Intervall liegt im SCM bei einer höheren
Neutronenproduktion als beim MPM.
\par
Ein Vergleich der Histogramme bei den Gruppen nach Injektionszeit
führt auf die entsprechenden Ergebnisse.
%
\par
\begin{figure}[H]
  \center
  \fbox{\importimage{C6-09A}}
  \caption{Histogramm der Pulse aus dem Silberaktivierungszähler mit der Intervallbreite 2000 Pulse.
     Alle Entladungen sind zusammengefaßt.}
  \label{fig:neutronen:statistikA}
\end{figure}
%
\par
\begin{figure}[H]
  \center
  \fbox{\importimage{C6-09B}}
  \caption{Histogramm der Pulse aus dem Silberaktivierungszähler mit der Intervallbreite 2000 Pulse.
     Die Entladungen sind in zwei Gruppen nach dem Entladungsmodus SCM/MPM eingeteilt.}
  \label{fig:neutronen:statistikB}
\end{figure}
%
\par
\begin{figure}[H]
  \center
  \fbox{\importimage{C6-09C}}
  \caption{Histogramm der Pulse aus dem Silberaktivierungszähler mit der Intervallbreite 2000 Pulse.
     Die Entladungen sind in zwei Gruppen nach der effektiven Injektionszeit \teff eingeteilt.}
  \label{fig:neutronen:statistikC}
\end{figure}
%
\par
Die Ab"-hän"-gig"-keit der Neutronenproduktion von der dem Anteil
des Injektionsgases ist deutlich zu trennen von der
Ab"-hän"-gig"-keit der Neutronenproduktion von der Schichtbildung.
Bei konstanten Betriebsparametern zeichnen sich gute Entladungen
mit guter Schichtbildung durch vergleichsweise hohe
Neutronenproduktion und hohe
Rönt"-gen"-strah"-lungs"-in"-ten"-si"-tät aus. Bei Variation der
Betriebsparameter, z.B. Vergrößerung der Injektionszeit, kann die
Neutronenproduktion reduziert werden, aber die
Röntgenstrahlungsintensität zunehmen.
\par
Der Einfluß der Schuß-zu-Schuß Schwankungen auf die Neutronenproduktion
und den Entladungsmodus wurde bereits in Kapitel
\vref{sec:uebergangsbereich} behandelt. Im Übergangsbereich zwischen
SCM und MPM entscheiden diese Schwankungen über den resultierenden
Modus. Bei \glqq guten\grqq\ Entladungen ist die Neutronenproduktion
hoch und der SCM ist sichtbar, bei \glqq schlechten\grqq\ Entladungen
ist die Neutronenproduktion deutlich geringer und der MPM ist sichtbar.
Wie \glqq gut\grqq\ bzw. \glqq schlecht\grqq\ eine Entladung sein muß,
entscheiden die Betriebsparameter.
%
\par
Zeitaufgelöst, mit der Photomultiplier/Szintillator-Kombination (side
on, Abstand \wert{0.7}{m}) gemessen, ergeben sich auch Unterschiede
zwischen dem SCM und dem MPM. Das Integral des Signals zeigt die
gleichen Effekte, die auch der Siber\-akti\-vierungs\-zähler mißt.
Daher wurden die Signale auf die gleiche Fläche normiert, damit die
Unterschiede im Zeitverhalten besser sichtbar werden. Zuvor wurde noch
eine Nullmessung subtrahiert, weil die kleinen Signale (um
\wert{0.5}{V}) einen systematischen Fehler durch Einkoppeln der
Kondensatorbatteriespannung zeigten.
\par
Der Mittelwert der Breite der Pulse (FWHM) bei Entladungen im SCM ist
mit \wert{(66 \pm 3)}{ns} um \wert{16}{ns} größer als der Mittelwert im
MPM mit \wert{50 \pm 3}{ns}.
%
\par
\begin{figure}[H]
  \center
  \fbox{\importimage{C6-12A}}
  \caption{Signale der Photomultiplier/Szintillator-Kombination. Alle Signale
     sind auf die gleiche Fläche normiert.}
  \label{fig:plots:pmalle}
\end{figure}
%
\par
Die Abbildung \vref{fig:plots:pmalle} zeigt in einem Plot alle
Signale der SCM bzw. der MPM Entladungen. Es ist sichtbar, daß die
Schwankungen im SCM kleiner sind als im MPM. Der Startzeitpunkt
der Pulse schwanken im SCM und in MPM im Bereich \wert{727 -
804}{ns}. Die Endzeitpunkte liegt beim SCM in einem etwas späteren
Intervall \wert{821-878}{ns} gegen \wert{804-859}{ns}.
\par
Der Abbildung \vref{fig:plots:pmmittel} zeigt den Mittelwert über
alle Signale im SCM bzw. MPM. Die mittleren Pulse sind aufgrund der
Unterschiede zwischen den Einzelpulsen verbreitert. Die Verbreiterung
beträgt beim SCM \wert{23}{ns} und beim MPM \wert{30}{ns}. Dies ist
ein weiteres Zeichen dafür, daß der SCM die bessere Reproduzierbarkeit
besitzt.
%
\par
\begin{figure}[H]
  \center
  \fbox{\importimage{C6-12B}}
  \caption{Mittelwert der normierten Signale der PM/Szintillator-Kombination
     im Zeitbereich mit der maximalen Kompression. Die FWHM beträgt im MPM
     \wert{80}{ns} und \wert{89}{ns} im SCM.}
  \label{fig:plots:pmmittel}
\end{figure}
%
\par
Das Maximum im MPM ist um \wert{14}{ns} nach hinten verschoben. Eine
Verschiebung um ca. \wert{10}{ns} tritt nicht nur beim PM-Signal auf,
auch beim Spannungssignal ist diese Verschiebung sichtbar.
%
\beginsubsection{Pinchdynamik}
%
Die Abbildung \ref{fig:plots:ualle} und \vref{fig:plots:umittel} zeigen
die gemittelten Kurven der Spannungssignale von Entladungen im MPM bzw
SCM.
%
\par
\begin{figure}[H]
  \center
  \fbox{\importimage{C6-12C}}
  \caption{Gemittelte Spannungssignale für jeden Entladungsmodus.}
  \label{fig:plots:ualle}
\end{figure}
%
\par
\begin{figure}[H]
  \center
  \fbox{\importimage{C6-12D}}
  \caption{Gemittelte Spannungssignale für jeden Entladungsmodus im
     Zeitbereich mit der maximalen Kompression.}
  \label{fig:plots:umittel}
\end{figure}
%
\par
Das Maximum des Spannungsspikes beim Pinch ist im MPM
\wert{10}{ns} später als beim SCM. Bis zur Pinchphase zeigen die
Kurven der Spannung alle einen ähnlichen Verlauf.
\par
Für die folgende Tabelle \vref{tab:zeit:spannungspeak} wurden die
Zeitpunkte der Pinchspannungen bei den einzelnen Signalen gemessen.
Danach wurden die Zeitpunkte entsprechend der aufgeführten Gruppe
gemittelt.
%
\par
\begin{table}[H]
  \center
  \begin{tabular}{|l|c|c|c|}
  \hline
     Gruppe                                 & minimal        & maximal        & Mittelwert \\
  \hline
     alle Entladungen                       & \wert{754}{ns} & \wert{812}{ns} & \wert{776 \pm 2}{ns} \\
  \hline
     Entladungen im SCM                     & \wert{754}{ns} & \wert{800}{ns} & \wert{772 \pm 3}{ns} \\
     Entladungen im MPM                     & \wert{756}{ns} & \wert{812}{ns} & \wert{780 \pm 3}{ns} \\
  \hline
     Injektionszeit \teff \wert{< 3.5}{ms}  & \wert{754}{ns} & \wert{812}{ns} & \wert{777 \pm 3}{ns} \\
     Injektionszeit \teff \wert{\ge 3.5}{ms}& \wert{756}{ns} & \wert{800}{ns} & \wert{775 \pm 3}{ns} \\
  \hline
  \end{tabular}
  \caption{Zeitpunkt des Spannungsspikes bezogen auf Zeitpunkt der Triggerung der Anlage (inkl.
           Laufzeiten). Minimaler, maximaler und Mittelwert von alle Entladungen einer Gruppe.}
  \label{tab:zeit:spannungspeak}
\end{table}
%
\par
Die Zeitdifferenzen zwischen SCM und MPM beträgt \wert{8}{ns} = $2.7\
\cdot$ Meßunsicherheit. Dieser Effekt kann nicht durch dir vergrößerte
Argonmenge erklärt werden, weil die Gruppen, nach Injektionszeit
getrennt, keine signifikante Abweichung vom Mittelwert über alle
Entladungen zeigen. Bei Entladungen im SCM muß es also einen
Mechanismus geben, der die Kompression früher anhält als Entladungen im
MPM. Der im Kapitel \vref{sec:mechanismus} vorgeschlagene
Stabilisierungsmechanismus hat eine solche Wirkung.
\par
Die $\rm\dot{I}$-Signale zeigen auch eine Tendenz zur verringerten
Dynamik bei Entladungen im SCM, aber nicht so signifikant, wie bei
den $U$-Signalen.
%
\par
\begin{figure}[H]
  \center
  \fbox{\importimage{C6-12E}}
  \caption{Gemittelte $\rm\dot{I}$-Signale für jeden Entladungsmodus.}
  \label{fig:plots:ialle}
\end{figure}
%
\par
\begin{figure}[H]
  \center
  \fbox{\importimage{C6-12F}}
  \caption{Gemittelte $\rm\dot{I}$-Signale für jeden Entladungsmodus im
     Zeitbereich mit der maximalen Kompression.}
  \label{fig:plots:imittel}
\end{figure}
%
\par
Die Abbildungen \ref{fig:plots:ialle} und \vref{fig:plots:imittel}
zeigen die gemittelten Kurven bei den betrachteten Entladungen.
\par
Die gemittelten $\rm\dot{I}$-Signale zeigen: das Minimum wird beim MPM
\wert{8}{ns} nach dem Minimum in SCM erreicht.Der Minimalwert ist beim
MPM um \wert{0.7}{kA/ns} geringer als beim SCM. Der längere und tiefere
Stromeinbruch beim MPM deutet auf eine längere und größere Kompression
der Plasmas. Auch hier wir angedeutet, daß es bei Entladungen im SCM
einen Mechanismus gibt, der der Kompression entgegenwirkt.
\par
Die genaue Auswertung zeigt aber, daß die bei den $\rm\dot{I}$-Signalen
auftretenden Unterschiede genau auf der Grenze der statistischen
Signifikanz liegen.
\par
Die Auswertung des Zeitpunktes und des Werts des
$\rm\dot{I}$-Minimums bei den einzelnen Entladungen sind in den
Tabellen \ref{tab:zeit:stromminimum} und
\vref{tab:wert:stromminimum} zusammengefaßt.
%
\par
\begin{table}[H]
  \center
  \begin{tabular}{|l|c|c|c|}
  \hline
     Gruppe                                 & minimal        & maximal        & Mittelwert \\
  \hline
     alle Entladungen                       & \wert{736}{ns} & \wert{792}{ns} & \wert{759 \pm 2}{ns} \\
  \hline
     Entladungen im SCM                     & \wert{736}{ns} & \wert{780}{ns} & \wert{756 \pm 3}{ns} \\
     Entladungen im MPM                     & \wert{738}{ns} & \wert{792}{ns} & \wert{761 \pm 3}{ns} \\
  \hline
     Injektionszeit \teff \wert{< 3.5}{ms}  & \wert{738}{ns} & \wert{792}{ns} & \wert{762 \pm 3}{ns} \\
     Injektionszeit \teff \wert{\ge 3.5}{ms}& \wert{736}{ns} & \wert{786}{ns} & \wert{756 \pm 3}{ns} \\
  \hline
  \end{tabular}
  \caption{Zeitpunkt des Minimums in $\rm\dot{I}$-Signals bezogen auf Zeitpunkt der Triggerung der Anlage
     (inkl. Laufzeiten). Minimaler, maximaler und Mittelwert von alle Entladungen einer Gruppe.}
  \label{tab:zeit:stromminimum}
\end{table}
%
\par
\begin{table}[H]
  \center
  \begin{tabular}{|l|c|c|c|}
  \hline
     Gruppe                                 & minimal             & maximal            & Mittelwert \\
  \hline
     alle Entladungen                       & \wert{-11.4}{} & \wert{-6.3}{} & \wert{-8.7 \pm 0.2}{} \\
  \hline
     Entladungen im SCM                     & \wert{-11.2}{} & \wert{-7.3}{} & \wert{-8.4 \pm 0.3}{} \\
     Entladungen im MPM                     & \wert{-11.4}{} & \wert{-6.3}{} & \wert{-9.0 \pm 0.3}{} \\
  \hline
     Injektionszeit \teff \wert{< 3.5}{ms}  & \wert{-11.3}{} & \wert{-6.3}{} & \wert{-8.3 \pm 0.3}{} \\
     Injektionszeit \teff \wert{\ge 3.5}{ms}& \wert{-11.4}{} & \wert{-7.3}{} & \wert{-9.3 \pm 0.3}{} \\
  \hline
  \end{tabular}
  \caption{Wert des Minimums im $\rm\dot{I}$-Signal in der Einheit ka/ns.
           Minimaler, maximaler und Mittelwert von alle Entladungen einer Gruppe.}
  \label{tab:wert:stromminimum}
\end{table}
%
\par
Der Mittelwert bei den verschiedenen Gruppen sowohl vom Zeitpunkt als
auch vom Wert des $\rm\dot{I}$-Minimums liegen um den Mittelwert aller
Entladungen innerhalb der Meßunsicherheit.
\par
Umfangreichere Meßreihen um die gefundenen Tendenzen, die alle mit dem
vorgeschlagenen Stabilisierungsmechanismus übereinstimmen, waren
geplant, konnten aber aufgrund kurzlebiger Isolatoren nicht
durchgeführt werden. Für eine ausreichende Anzahl von Entladungen ist
ein Isolator mit überdurchschnittlicher Lebensdauer erforderlich.
