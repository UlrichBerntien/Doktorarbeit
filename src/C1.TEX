%
%  Zusammenfassung
%  und Einleitung
%
\setcounter{page}{1}
\beginsection{Zusammenfassung}
\label{sec:zusammenfassung}
%
Bei den Plasmafokusentladungen am Experiment SPEED~2 werden zwei
verschiedene Entladungsformen beobachtet. Die beiden
Entladungsmodi können mit den zeitintegrierten Bildern der
Rönt"-gen"-pin"-hole"-kame"-ra (\wert{\lambda < 1}{nm})
identifiziert werden. Im Mikropinchmodus (MPM für englisch
micropinch mode) zeigen diese Bilder mehrere nahezu punktförmige
(\wert{r < 0.1}{mm}) Strahlungsquellen im Gebiet des Pinches. Im
stabilen Säu"-len"-mo"-dus (SCM für englisch stable column mode)
zeigen die Bilder eine zylinderförmige (\wert{r \approx 1}{mm},
\wert{l \approx 30 }{mm}) Strahlungsquelle. Die unterschiedlichen
Formen wurden in der Arbeit von Mälzig \cite{maelzig:phd} zuerst
dokumentiert. Danach wurde fast ausschließlich der MPM untersucht.
Die vorliegende Arbeit beschäftigt sich mit dem SCM. (Ausführliche
Darstellung $\rightarrow$ Kapitel \vref{sec:einleitung})
\par
Der SCM zeigt gegenüber dem MPM eine sehr gute Reproduzierbarkeit.
Im MPM treten die einzelnen Mikropinche zufällig in dem
Pinchzeitraum und zufällig in der Pinchregion auf
\cite{roewe:phd}. Die Säulen in verschiedenen Entladungen zeigen
dagegen eine konstante Position und Ausrichtung. Der SCM am
Experiment SPEED~2 unterschiedet sich deutlich von dem
Säu"-len"-mo"-dus, der bei anderen Anlagen beobachtet wird. An den
Plasmafokusanlagen in Aachen, Bochum und Sukhumi werden ebenfalls
zwei Entladungsmodi beobachtet. Die entstehenden Plasmasäulen
erreichen aber nicht die Homogenität, Reproduzierbarkeit und
Lebensdauer des SCM. ($\rightarrow$ Kapitel
\vref{sec:pinchplasmen})
\par
Der Plasmafokus SPEED~2 ist durch seine kurze Stromanstiegszeit
(\wert{\tau/4 = 400}{ns}) ausgezeichnet. Für die Untersuchung des SCM
wurde die Anlage typisch mit \wert{180}{kV} Ladespannung und
\wert{66}{kJ} Bankenergie betrieben. Dabei wird ein Pinchstrom von
typisch \wert{1}{MA} erreicht.
\par
Die Entladung wird in D$_2$-Gas (typisch \wert{5}{hPa}) gezündet.
In der Pinchregion wird ein Hoch-Z-Gas (in dieser Arbeit: Neon
oder Argon) injiziert, um eine effektive
Rönt"-gen"-strah"-lungs"-quel"-le zu erhalten. Der modulare Aufbau
von SPEED~2 macht die Anlage flexibel und zuverlässig.
($\rightarrow$ Kapitel \vref{sec:speed2})
\par
Zur Charakterisierung des SCM wurden verschiedene Diagnostiken
eingesetzt. Neben der Messung von U(t) und $\rm\dot I$(t) am Fokus
wurden die produzierten Neutronen und die Röntgenstrahlung
gemessen. Die Fusionsneutronen wurden zeitintegriert mit
Silberaktivierungszählern und zeitaufgelöst mit einer schnellen
Szin"-til"-la"-tor/Photo"-multi"-pli"-er-Kombination detektiert.
Mit einer Rönt"-gen"-pin"-hole"-kame"-ra wurden zeitintegrierte
Bilder der Plasmasäule aufgenommen. Zeitaufgelöste Messungen
konnten mit einer Röntgenstreakkamera durchgeführt werden. Ergänzt
wurden die Messungen durch zeitaufgelöste Untersuchungen mit einem
Kristallspektrometer (Dr. Y.~Sidelnikov). Auch konnten
wellenlängenselektiv und zeitaufgelöst Bilder mit einer
Vielschichtspiegeloptik (MLM) im Röntgenbereich (\wert{\lambda =
0.85}{nm}) aufgenommen werden (Dr. D.~Simanovskii). (Details der
Diagnostiken $\rightarrow$ Kapitel \vref{sec:diagnostiken})
\par
Der Entladungsmodus kann bei SPEED~2 zuverlässig über die
Betriebsparameter eingestellt werden. Die Ab"-hän"-gig"-keit des
Modus von den Parametern wurde systematisch untersucht. Die Menge
der Hoch-Z-Ionen im Pinchplasma kann über den Druck des
injizierten Gases und die Injektionszeit eingestellt werden. Im
untersuchten Betriebsbereich führte die Vergrößerung der
Hoch-Z-Gasmenge zum Wechsel des Entladungsmodus von SCM in den
MPM. Beide Modi wurden sowohl mit Neon- als auch mit
Argon-Injektion beobachtet. Als ein sehr wichtiger Parameter für
den resultierenden Entladungsmodus hat sich der Isolator, auf dem
die Fokusentladung zündet, gezeigt. ($\rightarrow$ Kapitel
\vref{sec:betriebsparameter})
\par
Auf MCP-Bildern (\wert{\lambda < 20}{nm}) ist die frühe
Kompressionsphase sichtbar. Mit der MLM-Optik wurde die Bildung der
Plasmasäule bei \wert{\lambda = 0.85}{nm} dokumentiert. Die
Radialgeschwindigkeit der Schicht beträgt bei der Kompression
\ewert{v_s = 1-3}{5}{m/s}. Die Bestimmung der Elektronendichte $n_e$
und Elektronentemperatur $T_e$ bei maximaler Kompression steht noch
aus. \wert{20}{ns} vor der maximalen Kompression wurden die Werte
\ewert{n_e \approx 2.5}{26}{m$^{-3}$} und \wert{kT_e \approx 300}{eV}
spektrometrisch bestimmt.
\par
Die gute Reproduzierbarkeit des SCM erlaubte die Bestimmung der
Lebensdauer der Plasmasäule mit der Röntgenstreakkamera. Eine
Lebensdauer der Quelle von \wert{90}{ns} wurde durch die Kombination
von mehreren Aufnahmen (\wert{\lambda < 2}{nm}, Zeitfenster je
\wert{40}{ns}) ermittelt. Eine Vergrößerung der injizierten
Neongasmenge führt zur Verkürzung der Lebensdauer und zur Verkleinerung
des Durchmessers der Quelle. Bei einigen Entladungen wurde ein zweites
Aufleuchten im Röntgenbereich (\wert{\lambda < 0.5}{nm}) nach dem
Zerfall der Säule beobachtet. Bei der Neutronenproduktion wurde eine
Abnahme bei Vergrößerung der injizierten Hoch-Z-Gasmenge gemessen.
Damit ergibt sich eine geringere Neutronenproduktion im MPM verglichen
mit dem SCM. (Details $\rightarrow$ Kapitel \vref{sec:untersuchung})
\par
Bei Entladungen im SCM muß ein Stabilisierungsmechanismus die Bildung
der stabilen Plasmasäule ermöglichen. Bei geeignet geänderten
Betriebsparametern darf dieser Mechanismus nicht mehr aktiv werden,
damit Mikropinche aus Instabilitäten entstehen können. Ein solcher
Mechanismus wird von Kies \cite{kies:99} beschrieben. In der
Kompressionsphase können Ionen am treibenden magnetischen Kolben
reflektiert werden. Dabei wird die Geschwindigkeit der Ionen um $2 v_s$
vergrößert. Unter geeigneten Bedingungen thermalisieren diese schnellen
Ionen nicht in der Plasmaschicht. Mehrfache Reflexionen in dem
komprimierenden Magnetfeld sind dann möglich und es bildet sich eine
schnelle Ionenkomponente (\glqq gyro-reflexion acceleration
mechanism\grqq ). Ist die Anzahl der schnellen Ionen ausreichend groß,
dann wirken sie stabilisierend auf den Pinch. (Ausführliche Darstellung
$\rightarrow$ Kapitel \vref{sec:mechanismus})
%
%
\beginsection{Einleitung}
\label{sec:einleitung}
%
In der Kurz- und Zusammenfassung wurde der Inhalt dieser Arbeit
bereits vorgestellt, daher wird jetzt das Umfeld dieser Arbeit
detailliert dargestellt. Die vorliegende Arbeit kann als dritter
Band in einer Trilogie von Dissertationen
\cite{maelzig:phd,roewe:phd}, die sich alle mit der
Rönt"-gen"-dia"-gnos"-tik am Plasmafokus SPEED~2 beschäftigen,
eingeordnet werden.
\par
Begleitet wurden die Dissertationen bisher immer von einigen
Diplomarbeiten und Staatsexamensarbeiten u.a.
\cite{doll:diplom,stein:staat,roewe:diplom,schmitz:diplom,
nadolny:diplom,lucas:diplom,soll:diplom}. Die jetzt vorliegende Arbeit
mußte ohne eine solche Unterstützung auskommen. So bleibt an einigen
Stellen der Wunsch nach ausführlicheren Meßdaten offen. Erhalten
geblieben ist die gute Zusammenarbeit mit den Wissenschaftlern aus
Troitzk (Moskau) und St.~Petersburg, siehe dazu auch die Abschnitte
\ref{sec:kristallspektrometer}, \ref{sec:mlm} und \cite{bobashev:97}.
\par
Band 1 der kleinen Reihe ist die Arbeit von Mälzig über die \glqq
Rönt"-gen"-dia"-gnos"-tik an Pinchplasmen hoher
Energiedichte\grqq\ \cite{maelzig:phd}. Vor dieser Arbeit wurde
der Plasmafokus SPEED~2 überwiegend als Neutronen- und
Protonenquelle untersucht \cite{calker:phd}. Damit das Pinchplasma
als intensive Quelle weicher Röntgenstrahlung dienen konnte, mußte
das reine Deuteriumplasma durch ein Plasma mit Hoch-Z-Ionen
ersetzt werden. Dieses ist nicht problemlos, weil die Zündung der
Entladung nicht von dem Hoch-Z-Gas gestört werden darf. Die
Röntgenstrahlung von Deuteriumplasmen mit Beimischungen von Xenon,
Krypton, Argon, Neon, Stickstoff, Kohlenstoff oder Silizium wurde
mit verschiedenen Diagnostiken untersucht. Bei Xenon, Krypton oder
Argon als Beimischung wurden mehrere (typisch 10) kleine
Strahlungsquellen ($\mu$m-Bereich) in dem Pinchplasma gefunden.
Diese Mikropinche erreichen eine Elektronendichte von bis zu
\fwert{29}{m$^{-3}$} und eine Temperatur bis zu \wert{1.5}{keV}.
Bei Neon oder Stickstoff als Beimischung wurden keine Mikropinche
beobachtet, sondern eine mehr als \wert{10}{mm} lange Plasmasäule.
\par
Im Anschluß an die Dissertation von Mälzig \cite{maelzig:phd}
wurde konzentriert an der Untersuchung der Mikropinche gearbeitet.
Die 10- bis 20-fache Leistungsdichte der Mikropinche machten diese
gegenüber der großen Säu"-len"-struk"-tur besonders interessant.
Ein Teil dieser Forschungen war die Arbeit von Röwekamp mit der
\glqq Charakterisierung von Mikropinchen mittels SXR- und
XUV-Diagnostik am Plasmafokus SPEED~2\grqq\ \cite{roewe:phd}. Es
zeigte sich, daß Mikropinche aus den Instabilitäten der
Plasmasäule entstehen. Messungen konnten die Entstehung von
Mikropinchen im Zentrum von Einschnürungen der Plasmasäule
nachweisen. Innerhalb von wenigen \wert{100}{ps} wird das
Hoch-Z-Plasma lokal komprimiert und bis zum helium- oder
wasserstoffähnlichen Zustand ionisiert. Nach \wert{0.3 - 3}{ns}
zerfällt der Mikropinch schnell. Die Dichte, die Temperatur und
der Radius der Mikropinche konnten mit dem Strahlungskollapsmodell
erklärt werden.
\par
Alle wesentlichen experimentellen Untersuchungen an den
Mikropinchen waren damit abgeschlossen. Die theoretische
Beschreibung der Entstehung und Entwicklung war auch in ihren
Grundzügen behandelt. Trotz zahlreicher Versuche, konnte keine
Methode, die Positionen der Mikropinche auf die Achse zu
konzentrieren, gefunden werden. Das räumlich und zeitlich
zufällige Auftreten der Mikropinche aus Instabilitäten der
Plasmasäule macht eine Anwendung als Strahlungsquelle schwierig.
Folglich verschob sich das Interesse auf die bereits beobachtete
Säu"-len"-struk"-tur.
\par
Erste Meßreihen an Neonplasmen zeigten die ausgezeichnete
Stabilität (bis zu \wert{100}{ns} konnte jetzt gezeigt werden,
vgl. Kapitel \vref{sec:untersuchung}) und die sehr gute
Reproduzierbarkeit (vgl. Kapitel \vref{sec:betriebsparameter})
dieser Säu"-len"-struk"-tur. Auch bei anderen Maschinen wurden
säulenähnliche Strukturen beobachtet (vgl. Abschnitt
\vref{sec:vergleich}), die aber nicht die Homogenität und
Stabilität der Säulen am SPEED~2-Experiment zeigten. Zur
Abgrenzung wird daher von dem stabilen Säu"-len"-mo"-dus (SCM)
gesprochen.
\par
Dieses war die Ausgangssituation für die vorliegende Arbeit zum SCM.
Untersucht wurde zuerst der Bereich der Betriebsparameter, in dem der
SCM erreichbar ist. Dann wurde der SCM mit aufwendigeren Diagnostiken
wellenlängen- und zeitaufgelöst untersucht. Ein möglicher Mechanismus
für die Stabilisierung \cite{kies:99} wird vorgeschlagen.
\par
Parallel zu den Messungen am großen Plasmafokus SPEED~2 laufen Arbeiten
an kleineren Experimenten der SPEED-Reihe, den Modellen 3 und 4
\cite{raacke:phd}. Dabei handelt es sich um z-Pinchentladungen mit
U$_{\rm max}=$\wert{100}{kV}, E$_{\rm max}=$\wert{12.5}{kJ} bzw.
U$_{\rm max}=$\wert{30}{kV}, E$_{\rm max}=$\wert{1.13}{kJ}. Studien zu
den möglichen Entladungsformen bei diesen Experimenten sind noch nicht
abgeschlossen.
\par
Langfristig könnten Plasmafoki- und z-Pinch-Anlagen im stabilen
Säu"-len"-mo"-dus als gepulste Röntgenstrahlungsquellen genutzt
werden. Die kurzen (typisch \wert{100}{ns}) und energiereichen
(typisch einige 10 Joule) Pulse weicher Röntgenstrahlung
(\wert{\lambda = 10 - 0.1}{nm}) könnten z.B. in der
Röntgenmikroskopie oder der Röntgenlithographie Anwendung finden.
Unmittelbare Anwendungen in der Industrie von Plasmafoki der Größe
von SPEED~2 sind sehr wahrscheinlich nicht zu erwarten. Bevor der
SCM genutzt werden kann, muß er auch in kleineren Anlagen
(Table-top Systemen) realisiert werden.
