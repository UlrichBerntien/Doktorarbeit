%
%  Kurzfassung, eine Seite
%
  \setcounter{page}{5}
  \pagestyle{freeheadings}
  \markboth{\noexpand\roman{page} \hfill}{\hfill \noexpand\roman{page}}
%
\begin{center}
  {\sc Kurzfassung\\
       Charakterisierung des stabilen Säulenmodus\\
       am Plasmafokus SPEED~2}\\
\end{center}
%
\par
Der Plasmafokus SPEED~2 ist eine schnelle (\wert{\tau/4 =
400}{ns}) und stromstarke (\wert{I_m = 2.5}{MA}) Plasmafokusanlage
mit einer Bankenergie von maximal \wert{187}{kJ} bei einer
maximalen Ladespannung von \wert{300}{kV}. Dieser leistungsstarke
Plasmafokus zeigt zwei grundlegend verschiedene Entladungsformen,
den Mikropinchmodus und den stabilen Säu"-len"-mo"-dus. Im
Mikropinchmodus führen Instabilitäten zu stark lokalisierten
Einschnürungen der Plasmasäule. Diese Ein"-schnü"-rung"-en treten
beim stabilen Säulenmodus nicht auf.
\par
Der Mikropinchmodus zeigt auf zeitintegrierten
Röntgenpinholebildern ($\lambda <$ \wert{1}{nm}) die typischen
punktförmigen ($r <$ \wert{0.1}{mm}) Emissionsquellen, die
zeitlich und räumlich zufällig innerhalb des Pinches auftreten.
Der stabile Säulenmodus dagegen zeigt auf diesen Bildern eine gut
reproduzierbare säulenförmige Strahlungsquelle ($r \approx$
\wert{1}{mm}, $l \approx$ \wert{30}{mm}).
\par
Bei anderen Fokus-Anlagen werden auch verschiedene Entladungsmodi
beobachtet, aber keine dieser Entladungsformen erreicht die Stabilität
und Reproduzierbarkeit des stabilen Säulenmodus.
\par
Der Entladungsmodus kann bei SPEED~2 über die Betriebsparameter
(Injektionsgas, Bankenergie, u.a.) eingestellt werden. Nur in
einem Übergangsbereich für die Parameter ist der Modus von den
Schuß-zu-Schuß Schwankungen ab"-hängig. Entladungen beider Modi
können zuverlässig sowohl in D$_2$/Ne- als auch D$_2$/Ar-Gas
produziert werden. Der Wechsel vom Mikropinchmodus in dem stabilen
Säulenmodus erfolgt bei Vergrößerung der Hoch-Z-Gasmenge.
\par
Die Bildung der Säule ist auf Multi"-layer"-mirror-Bil"-dern
zeit"-auf"-ge"-löst ($\Delta t =$ \wert{10}{ns}) und
wellenlängenselektiv ($\lambda =$ \wert{0.85}{nm},
\wert{\lambda/\Delta \lambda = 50-100}{}) sichtbar. Streakbilder
im Röntgenbereich ($\lambda <$ \wert{2}{nm}) zeigen eine
Lebensdauer von typisch \wert{90}{ns} für die stabile Säule. Eine
Elektronendichte von $n_e =$ \ewert{2.5}{26}{m$^{-3}$} und eine
Elektronentemperatur von $kT_e =$ \wert{300}{eV} wurde
\wert{20}{ns} vor der maximalen Kompression röntgenspektroskopisch
bestimmt. Bei der weiteren Kompression ist \ewert{n_e \approx
1}{27}{m$^{-3}$} und \wert{kT_e \approx 1}{keV} zu erwarten. Die
Vergrößerung der Hoch-Z-Gasmenge führt zu einer Reduzierung der
Neutronenproduktion und zum Wechsel in den Mikropinchmodus. Aus
den Spannungs- und Stromsignalen, ist die Tendenz zur geringerer
Kompression bei Entladungen im stabilen Säulenmodus erkennbar.
\par
Erklärt werden kann die Stabilisierung der Plasmasäule durch den
\glqq gyro-reflexion acceleration mechanism\grqq . Bei der
schnellen $v_s \approx$ (\ewert{1}{5}{m/s}) Kompression der
Plasmaschicht durch den magnetischen Kolben werden Ionen zwischen
der kollabierenden Schicht reflektiert. Bei jeder Reflektion
nehmen die Ionen kinetische Energie auf. Der Druck dieser
schnellen Ionen wirkt Instabilitäten entgegen, behindert aber auch
die Kompression der Plasmasäule. Werden die Ionen zwischen den
Reflexionen in der Plasmaschicht zu stark abgebremst, kann der
Mechanismus nicht wirksam einsetzen und Mikropinche entstehen.
%
\hfill
