%
%  Theoretische Überlegungen zum stabilen Säulenmodus
%
\beginsection{Möglicher Stabilisierungsmechanismus}
\label{sec:mechanismus}
%
Ein theoretisches Modell der Fokusentladung liefert im optimalen Fall
aus den Betriebsparametern die Daten der Entladung, insbesondere die
Entscheidung zwischen MPM und SCM. Die Entwicklung eines solchen
Modells ist wünschenswert, kann aber nicht in dieser experimentell
ausgerichteten Arbeit erfolgen.
\par
Diese Arbeit beschränkt sich auf die Vorstellung eines möglichen
Stabilisierungsmechanismus, der von Kies \cite{kies:99} für den
SCM vorgeschlagen wurde. Die Basis für diesen Mechanismus wurde in
der Arbeit von Deutsch \cite{deutsch:87} gelegt. Dort wurde die
Neutronenproduktion und die Stabilisierung des Pinches am SPEED~2
Experiment mit Deuteriumgas behandelt. Die ausgeführten Ideen
mußten auf eine Entladung mit Deuterium/Hoch-Z-Gasgemisch
über"-tra"-gen werden.
%
\par
\begin{figure}[H]
  \center
  \fbox{\importimage{C7-01}}
  \caption{Bei der theoretischen Behandlung der Entladung wird vereinfacht
    die Kompressionsphase der Pinchentladung betrachtet. Die
    Plasmaschicht, getrieben vom Magnetfeld, läuft in das Gas und bildet
    einen z-Pinch. Nach \cite{deutsch:87}; vgl. Abbildung \vref{fig:fokusschema}.}
  \label{fig:modell}
\end{figure}
%
\par
Die Wirkung des Stabilisierungsmechanismus ist nur in dem Bereich des
Fokus (3-4) interessant. Die Zündung (1) und die Laufphase (2) der
Entladung wird bei diesem Modell nicht beschrieben. Die Abbildung
\vref{fig:modell} zeigt die Vereinfachung auf die Modellsituation. Die
Krümmung der Plasmaschicht wird vernachlässigt, der Strom fließt über
eine plane Kathode ab. Damit entsteht ein Modell, das einem z-Pinch
entspricht.
\par
Der Strom fließt in einer Plasmaschicht. Diese wird getrieben von einem
magnetischen Kolben (Zone I.). Vor der Plasmaschicht läuft eine
Schockfront in das Gas (Zone II.). In der Schockfront werden die
Deuteriummoleküle dissoziiert und ionisiert, ebenso werden die Neon-
oder Argon-Atome ionisiert. Mit der radialen Schichtgeschwindigkeit
$v_s$ (typisch \ewert{1}{5}{m/s}) komprimiert sich der Plasmazylinder
auf die Achse. Dabei werden die Ionen und Elektronen in der
Plasmaschicht gesammelt und mitgeführt, so daß kein Gas zurückbleibt
(Zone II.). Dieses ist das sogenannte Schneepflugmodell. Die
Gleichungen für Massen-, Impuls- und Energieerhaltung führen zur
mathematischen Darstellung des Modells \cite{deutsch:86}. Es liefert
brauchbare Abschätzungen für die Dichte (typisch \ewert{n_e =
1}{24}{m$^{-3}$}) und Temperatur (typisch \wert{kT_e = 10}{eV}) der
Plasmaschicht, aber keinen Stabilisierungsmechanismus.
\par
Genauere Untersuchungen \cite{deutsch:87} zeigen, daß nicht alle
Ionen in der Plas"-ma"-schicht bleiben. Das Verhalten einzelner
Deuteronen im Deu"-te"-rium-Pinch wurde mit einer
Monte--Carlo--Simulation betrachtet. Dabei wurde (pro
Simulationslauf) nur die Bewegung eines einzelnen Ions betrachtet,
alle anderen Ionen wurden als Hintergrund über die Ionendichte
berücksichtigt. Die Abbildung \ref{fig:deuteronbahnen} zeigt
Beispiele für die berechneten Bahnen, dabei wurden Stöße
vernachlässigt und der Plasmazylinder durch zwei ebene
Plasmaschichten ersetzt.
%
\par
\begin{figure}[H]
  \center
  \fbox{\importimage{C7-02}}
  \caption{Bahnen von Deuteronen im ebenen Modell mit \wert{B=5}{T},
  \ewert{v_s=1}{5}{m/s} und $v_{th}=0.1 \cdot v_s$. Startposition der
  Schicht \wert{0.02}{m}, Startpositionen der Deuteronen (1)
  \wert{0.027}{m}, (2) \wert{0.02}{m}, (3) \wert{0.0197}{m},
  (4) \wert{0.015}{m}. Aus \cite{deutsch:87}.}
  \label{fig:deuteronbahnen}
\end{figure}
%
\par
Auffällig ist das Verhalten der Deuteronen (3) und (4), sie werden von
dem magnetischen Kolben reflektiert. Bei jeder Reflektion im Magnetfeld
erhöht sich ihre Geschwindigkeit um $2v_s$, dazu kommt noch ein
Beschleunigungsanteil aus dem elektrischen Feld in der Plasmaschicht
senkrecht zur Bewegung der Schicht. Dieser Mechanismus wird \glqq
gyro-reflection acceleration mechanism\grqq\ (GRAM) genannt. Die
Abbildung \vref{fig:deuteronenergie} zeigt die Zunahme der kinetischen
Energie dieser Deuteronen.
%
\par
\begin{figure}[H]
  \center
  \fbox{\importimage{C7-03}}
  \caption{Kinetische Energie der Ionen aufgetragen gegen die Zeit.
  Energie normiert auf $E_0=\frac{1}{2}mv_s^2$. Die Nummern entsprechen
  denen aus Abbildung \ref{fig:deuteronbahnen}.
  Aus \cite{deutsch:87}.}
  \label{fig:deuteronenergie}
\end{figure}
%
\par
Näherungsweise verhält sich die kinetische Energie wie
\par
$$ E_{\rm kin}\propto \frac{1}{r^2}. $$
\par
Stöße mit Elektronen, Ionen und Atomen im Plasma und im Füllgas führen
zu einer Bremsung dieser schnellen Ionen. Eine ausreichende freie
Weglänge ist daher eine Bedingung für diesen Mechanismus.
\par
Haben sich die schnellen Deuteronen gebildet, so stabilisieren sie die
Plasmasäule. Führt eine kleine Störung zu einer Verringerung des Radius
r, so erhöht sich der Magnetfelddruck mit $1/r^2$. Der Plasmadruck
erhöht sich aufgrund der vergrößerten Teilchendichte mit $1/r^\alpha,
\alpha > 0$ und aufgrund des GRAMs mit $1/r^2$, also zusammen
proportional zu $1/r^{(2+\alpha)}$. Die Erhöhung des Plasmadrucks kann
die Störung ausgleichen, weil der Plasmadruck stärker als der
Magnetfelddruck mit abnehmendem Radius $r$ ansteigt.
\par
Wenn der GRAM bei Entladungen im SCM auftritt, dann kann dieser
Mechanismus die Stabilisierung der Plasmasäule erklären. Damit der GRAM
auftreten kann, muß ein kritischer Strom $I_C$ erreicht werden, müssen
die Ionen das Plasma und das Gas durchdringen können.
\par
Der kritische Strom $I_C$ ergibt sich aus einer Bedingung für den
Krüm"-mungs"-ra"-dius $\rho$ der Ionenbahn bei der Reflexion am
magnetischen Kolben. Dieser Radius $\rho$ muß kleiner sein als der
Krümmungsradius $R^\ast$ der Schicht. Näherungsweise ist der
Krümmungsradius $\rho$ gleich dem Lamorradius $r_L$. Mit den
Beziehungen
\par
$$ r_L = \frac{mv}{qB} \quad,\quad B = \frac{\mu_0 I}{2 \pi R^\ast} $$
\par
($m$ = Ionenmasse, $q$ = Ionenladung, $v$ = Ionengeschwindigkeit)
ergibt sich der kritischen Strom $I_C$ aus
\par
$$ I > I_C = \frac{6\pi}{\mu_0} \cdot \frac{m}{q} \cdot v_s .$$
\par
Dabei wurde für die Teilchengeschwindigkeit bei der ersten Reflexion $v
= 3 \cdot v_s$ angenommen.
\par
Die Schichtgeschwindigkeit $v_s$ liegt zwischen \ewert{1}{5}{m/s}
im Deuterium-Hoch-Z-Gasgemisch und \ewert{3}{5}{m/s} in reinem
Deuterium-Gas. Neben den Deuteronen kommen auch die Neon-Ionen
dreifach geladen und die Argon-Ionen fünffach geladen für den GRAM
in Betracht.
\par
Die Tabelle \vref{tab:kritischerStrom} gibt die kritischen
Stromstärken in den verschiedenen Situationen an.
%
\par
\begin{table}[H]
  \center
  \begin{tabular}{|c|c|c||c|c|}
    \hline
    Ion & Masse & Ladung & \ewert{v_s=1}{5}{m/s}  & \ewert{v_s=3}{5}{m/s} \\
    \hline
    D$^+$      &  2u & +1e & \wert{I_C = \ 31}{kA}  &  \wert{I_C = \ 93}{kA} \\
    Ne$^{3+}$  & 20u & +3e & \wert{I_C = 104}{kA} & \wert{I_C = 312}{kA} \\
    Ar$^{5+}$  & 40u & +5e & \wert{I_C = 125}{kA} & \wert{I_C = 375}{kA} \\
  \hline
  \end{tabular}
  \caption{Kritischer Strom $I_C$ für das Einsetzen des GRAMs bei
  verschiedenen Ionen und verschiedenen Schichtgeschwindigkeiten $v_s$.}
  \label{tab:kritischerStrom}
\end{table}
%
\par
Beim Plasmafokus SPEED~2 beträgt der Strom im Maximum \wert{\ge 1}{MA}.
Das Strommaximum war bei den hier vorgestellten Experimenten ca.
\wert{150}{ns} vor der maximalen Kompression. Daher wurde der kritische
Strom $I_C$ immer deutlich vor der Bildung der Pinchsäule
überschritten. Diese Bedingung für den GRAM ist bei allen untersuchten
Betriebsparametern erfüllt worden.
\par
Die Reflexion der Ionen am magnetischen Kolben ist also möglich. Die
schnellen Ionen müssen nach einer Reflexion die Plasmaschicht, das Gas
im Zentrum (Zone I. in Abbildung \vref{fig:modell}) und nochmals die
Plasmaschicht durchdringen bis die nächste Reflexion erfolgt. Der
Energieverlust auf diesem Weg muß deutlich kleiner sein als der
Energiegewinn beim Reflexionsvorgang, damit der GRAM einsetzen kann.
\par
Während die Dicke der Plasmaschicht konstant typisch \wert{0.1}{mm}
beträgt, ändert sich die Weglänge durch das Gas erheblich. Damit die
Energie der schnellen Ionen die Pinchsäule stabilisieren kann, muß der
Durchmesser der Plasmazylinder $d^\ast$ beim Einsetzen des GRAMs um den
Faktor 3 größer sein als der Pinchdurchmesser $d_p$. Bei einem
maximalen $d_p$ von \wert{2}{mm} ergibt sich \wert{d^\ast = 6}{mm}. Die
Ionen müssen also mindestens eine \wert{6}{mm} dicke Gasschicht
durchdringen.
\par
Der Einfluß des Füll- und Injektionsgases auf den Strahl schneller
Ionen kann mit Hilfe eines frei verfügbaren Programms abgeschätzt
werden. Das Programm
TRIM\footnote{\url{http://www.research.ibm.com/ionbeams}} führt
Monte-Carlo-Simulationen durch, bei denen die Reichweite von
Ionenstrahlen in Gasen und Fest"-kör"-pern bestimmt werden.
Grundlage des Programms ist die abgeschirmte
Coulomb-Wechselwirkung der Atome mit den schnellen Ionen
\cite{ziegler:96}.
%
\par
\begin{table}[H]
  \center
  \begin{tabular}{|c|c|c|c|}
    \hline
    Gas      & Reichweite von              & Reichweite von             & Reichweite von \\
             & D-Ionen mit                 & Ne-Ionen mit               & Ar-Ionen mit \\
             & \wert{E_{\rm kin}=1.3}{keV} & \wert{E_{\rm kin}=13}{keV} & \wert{E_{\rm kin}=25}{keV} \\
    \hline
    H        & \wert{72}{mm}       & \wert{> 100}{mm}  & \wert{> 100}{mm} \\
    H + Ne   & \wert{17}{mm}       & \wert{\ \ \ \ 15}{mm}     & \wert{\ \ \ \ 16}{mm} \\
    H + Ar   & \wert{11}{mm}       & \wert{\ \ \ \ 12}{mm}     & \wert{\ \ \ \ 13}{mm} \\
  \hline
  \end{tabular}
  \caption{Reichweiten von schnellen Ionen in Gasmischungen mit einer
  Teilchendichte von \ewert{n = 1}{23}{m$^{-1}$}.
  Die Ionen sind vollständig ionisiert angenommen.}
  \label{tab:IonenReichweiten}
\end{table}
%
Mit dem TRIM Programm ermittelte Reichweiten für die interessierenden
Ionen sind in der Tabelle \vref{tab:IonenReichweiten} zusammengefaßt.
Deuterium ist in dem Programm nicht vorgesehen, daher wurde auf
Wasserstoff ausgewichen. Dieses ist ohne Bedeutung für die Ergebnisse,
weil die Hoch-Z-Gase die Abbremsung dominieren. Die Geschwindigkeit der
Ionen wurde auf \ewert{1}{5}{m/s} gesetzt, die untere Grenze der zu
erwartenden Geschwindigkeiten. Der Fülldruck des Wasserstoffs wurde auf
\wert{4}{hPa} bei \wert{20}{$^\circ$C} gesetzt. Die Teilchendichte von
Neon bzw. Argon wurde gleich der Teilchendichte des Wasserstoffs
gesetzt. Diese Dichte tritt bei ca. \ewert{5}{5}{Pa} Injektionsdruck
und \wert{5}{ms} effektive Injektionszeit beim Experiment auf.
\par
Die Bremswirkung von Wasserstoff bzw. Deuterium ist so gering, daß der
Einfluß des Gases auf die schnellen Ionen vernachlässigt werden kann,
wie es auch bei den Monte-Carlo-Simulationen \cite{deutsch:87} gemacht
wurde. Der Einfluß des Injektionsgases auf die schnellen Deuteronen ist
nicht mehr in allen Fällen vernachlässigbar. Insbesondere ist bei Argon
und hohen Injektionsdrücken eine Hemmung des GRAMs für die Deuteronen
möglich. Für schnelle Neon- bzw. Argon-Ionen ist wahrscheinlich auch
der Einfluß des Hoch-Z-Gases vernachlässigbar, weil nicht vollständig
ionisierte Atome, sondern nur 3- bis 5-fach ionisierten Atome zu
erwarten sind. Für diese ist die Coulomb-Wechselwirkung mit den
Gas-Atomen deutlich geringer.
\par
Damit ist der Einfluß des D$_2$-Gases und des Injektionsgases auf die
schnellen Ionen abgeschätzt. Es bleibt der Einfluß der Plasmaschichten,
die von den Ionen zweimal pro Reflexion durchlaufen werden. Formeln zur
Berechnung des Bremsvermögens eines vollständig ionisierten
Wasserstoffplasmas werden in \cite{peter:91}\footnote{Der Artikel
benutzt weitgehend dimensionslose Größen. In der
Transformationsvorschrift (8) hat sich ein Schreibfehler
eingeschlichen: $k_D$ ist als $\lambda_D$ zu lesen.} angegeben. Im Fall
von einem Krypton-Ionenstrahl wurde eine gute Übereinstimmung der
Formeln mit den experimentellen Werten gefunden \cite{jacoby:95}.
\par
Für die Plasmaschicht gibt es Schätzwerte, aber speziell für die
Endphase der Kompression auch Meßwerte. Spektrometrisch wurde ca.
\wert{20}{ns} vor der maximalen Kompression die Elektronentemperatur
\wert{kT_e = 300}{eV} und die Elektronendichte \ewert{n_e =
2.5}{26}{m$^{-3}$} bestimmt (siehe Seite \pageref{fig:spektrum}).
Aufgrund der Bedingung $d^\ast / d_p > 3$ für die Kompression (s.o.)
müssen spätestens zu diesem Zeitpunkt Ionen mit dem GRAM beschleunigt
werden.
\par
Aus den Meßwerten folgen die Kenngrößen:
  Plasmafrequenz $\omega_p =$ \ewert{9.4}{9}{s$^{-1}$},
  mittlere thermische Geschwindigkeit der Elektronen $v_{th} =$ \ewert{7.3}{6}{m/s},
  mittlere thermische Geschwindigkeit der Deuteronen $v_{th,i} =$ \ewert{1.2}{5}{m/s}
  (Schätzung über $T_i = T_e$),
  Debye-Länge $\lambda_D =$ \ewert{8.1}{-9}{m} und
  Anzahl der Elektronen in der Debye-Kugel $N_D =$ \ewert{5.6}{2}{}.
\par
Die Geschwindigkeit der Ionen wurde auf \ewert{v_p = 5}{5}{m/s}
gesetzt, weil zu diesem Zeitpunkt schon erste Reflexionen stattgefunden
haben können. Die Geschwindigkeit der schnellen Ionen liegt damit
deutlich unter der thermischen Geschwindigkeit der Elektronen $v_p \ll
v_{th}$. Benutzt wurde daher die angegebene Formel für geringe
Geschwindigkeiten und mittlere Dichten:
\par
$$ - \frac{dE}{dx} = \frac{kT_e}{\lambda_D}
\frac{Z^2N_D}{12\pi\sqrt{2\pi}} \left[ (\ln K^2 -1)\left(
\frac{v_p}{v_{th}} \right) - \left( \frac{3}{10} \ln K^2 -
\frac{8}{5} + \frac{\pi}{20} \right) \left( \frac{v_p}{v_{th}}
\right)^3 \right] $$
\par
mit $$ Z = \frac{4\pi}{3} \frac{Z_{\rm eff}}{N_D} \quad {\rm
und}\quad K = \frac{8\pi}{Z}. $$
\par
Die berechneten Werte sind in der Tabelle \vref{tab:ElektronenBremsung}
aufgeführt.
%
\par
\begin{table}[H]
  \center
  \begin{tabular}{|r|c|c|c|}
    \hline
       & D$^+$ & Ne$^{3+}$ & Ar$^{5+}$ \\
    \hline
    Bremsvermögen: $-\frac{dE}{dx}=$ & \wert{17}{$\frac{\rm keV}{\rm mm}$} & \wert{98}{$\frac{\rm keV}{\rm mm}$} &  \wert{890}{$\frac{\rm keV}{\rm mm}$} \\
    kin. Energie: $E_{\rm kin}=$ & \wert{6.5}{keV} & \wert{65}{keV} & \wert{130}{keV} \\
    Bremsweg auf $\frac{1}{2} E_{\rm kin}$: $l_{1/2}=$ & \wert{0.19}{mm} & \wert{0.33}{mm} & \wert{0.07}{mm} \\
  \hline
  \end{tabular}
  \caption{Bremsung der schnellen Ionen durch die Elektronen in der Plasmaschicht}
  \label{tab:ElektronenBremsung}
\end{table}
%
Das Verhältnis von Ladung zur Ionenmasse ($q^2/m$) ist für die
Neon-Ionen am günstigsten, sie haben daher die größte Reichweite. Die
Energieverluste für die schnellen Ionen durch die Plasmaelektronen sind
hoch, aber diese Elektronendichte wird erst zum Ende der
Kompressionsphase erreicht. Ionen, die bereits eine höhere
Geschwindigkeit erreicht haben, werden relativ zu ihrer kinetischen
Energie schwächer gebremst ($ E_{\rm kin} \propto v_p^2$\
,$-\frac{dE}{dx} \propto v_p$ für $v_p \ll v_{th}$ ). Das Bremsvermögen
$-\frac{dE}{dx}$ der Plasmaelektronen ist maximal bei $v_p \approx 2
\cdot v_{th}$. Die schnellen Ionen können bis in diesen Bereich
beschleunigt werden, wenn sie nicht in der Anfangsphase der
Beschleunigung thermalisiert wurden (Runaway-Effekt).
\par
Neben der Bremsung durch die Elektronen erfahren die schnellen Ionen
auch eine Bremsung durch die Deuteronen des Plasmas. Die thermische
Geschwindigkeit der Deuteronen liegt mit \ewert{v_{th,i} = 1}{5}{m/s}
im Bereich der Geschwindigkeit der schnellen Ionen zu Anfang ihrer
Beschleunigung. Daher ist das Bremsvermögen durch die Deuteronen
besonders hoch, wie die Zahlen in der Tabelle
\vref{tab:DeuteronenBremsung} zeigen.
\par
Das Bremsvermögen durch die Ionen des Plasmas für $v_{th,i} \ll v_p \ll
v_{th}$ wird angegeben mit
\par
$$ - \frac{dE}{dx} = \frac{kT_e}{\lambda_D} \left[
\frac{Z^2N_D}{12\pi\sqrt{2\pi}} \left( \frac{v_p}{v_{th}} \right)
\ln( \frac{64\pi^2}{Z^2} ) + \frac{Z^2N_D}{4\pi M} \left(
\frac{v_{th}}{v_p} \right)^2 \ln( M \frac{8\pi}{Z} ) \right] $$
wobei $M$ die Ionenmasse in $m_e$-Einheiten ist.
%
\par
\begin{table}[H]
  \center
  \begin{tabular}{|r|c|c|c|}
    \hline
       & D$^+$ & Ne$^{3+}$ & Ar$^{5+}$ \\
    \hline
    Bremsvermögen: $-\frac{dE}{dx}=$ & \wert{98}{$\frac{\rm keV}{\rm mm}$} & \wert{0.83}{$\frac{\rm MeV}{\rm mm}$} &  \wert{2.2}{$\frac{\rm MeV}{\rm mm}$} \\
    kin. Energie: $E_{\rm kin}=$ & \wert{6.5}{keV} & \wert{65}{keV} & \wert{130}{keV} \\
    Bremsweg auf $\frac{1}{2} E_{\rm kin}$: $l_{1/2}=$ & \wert{0.03}{mm} & \wert{0.04}{mm} & \wert{0.03}{mm} \\
  \hline
  \end{tabular}
  \caption{Bremsung der schnellen Ionen durch die Deuteronen in der Plasmaschicht}
  \label{tab:DeuteronenBremsung}
\end{table}
%
Wieder sind die schnellen Neon-Ionen durch ihr $q^2/m$-Verhältnis
bevorzugt. Die Bremsung durch die Deuteronen stellt eine erhebliche
Behinderung des GRAMs dar, aber nur bei den ersten Reflexionen. Haben
die Ionen dieses Hindernis überwunden, nimmt das Bremsvermögen der
Deuteronen mit zunehmender Geschwindigkeit $v_p$ schnell ab.
\par
Damit der GRAM wirksam werden kann muß er also früher, bei geringeren
Deuteronendichten, einsetzten ($-\frac{dE}{dx} \propto n_i$). Da der
kritische Strom $I_C$ schon früh erreicht wird, kann auch der GRAM früh
einsetzen.
\par
Im Plasma der Schicht wird nicht nur Deuteriumgas, sondern auch das
Injektionsgas gesammelt. Am Anfang der Kompression sind nur Deuteronen
in dem Plasma, aber zum Ende (Durchmesser Injektionsgasstrom \wert{<
2}{cm}) kommen auch Hoch-Z-Ionen dazu. Der Anteil der Hoch-Z-Ionen
bleibt gering, $\frac{n_D}{n_X} \approx (\frac{10cm}{2cm})^2 = 25$.
Ihre höhere Ladung kompensiert die geringe Dichte nicht vollständig.
Das Bremsvermögen ($-\frac{dE}{dx}$) wird also am Ende der Kompression
unterhalb des Bremsvermögens der Deuteronen liegen. (Die Arbeit von
Peter \cite{peter:91} beschränkt sich auf Wasserstoffplasmen.) Die
geringe thermische Geschwindigkeit der schweren Ionen führt zu einem
gegen die Deuteronen verschobenem Maximum des Bremsvermögens bei $v_p
\approx 2 \cdot v_{th,i}$.
%
\par
Eine ausführliche theoretische Behandlung kann in dieser Arbeit, wie
schon anfangs erwähnt, nicht gegeben werden. Es fehlt die Betrachtung
der Ionisation und Rekombination der schnellen Ionen im Plasma; es
fehlt die Behandlung des azimutalen Runaways. Wesentlichen Effekte sind
vermutlich erläutert worden. Weitergehende Betrachtungen müßten in
Richtung einer Erweiterung der Simulationsrechnungen für den GRAM in
Deuterium/Hoch-Z-Gemischen geführt werden. Diese Simulationen könnten
auf den hier vorgestellten Effekten basieren, sie müßten den Einfluß
der Neon- bzw. Argon-Teilchendichte auf die Effizienz des GRAMs zeigen.
\par
Anhand der Abschätzungen ist sichtbar geworden, daß der GRAM in dem
hier untersuchten Arbeitsbereich des Plasmafokus SPEED~2 auftreten
kann. Der kritische Strom $I_C$ wird für die Ionen, D$^+$, Ne$^{3+}$,
Ar$^{5+}$ weit überschritten. Das Einsetzen des GRAMs kann durch das
Bremsvermögen der Ionen in der Plasmaschicht verhindert werden. Auch
können die Elektronen in der Plasmaschicht (und die Injektionsgasatome
bzgl. D$^+$) die Effizienz dieses Mechanismus reduzieren. Die
Abschätzungen zeigen die Möglichkeit der Grenze zwischen wirksamen GRAM
und kein wirksamer GRAM im Bereich des typischen Arbeitsbereiches.
Dieses deckt sich mit den experimentellen Ergebnissen in Kapitel
\ref{sec:betriebsparameter}, die immer die Grenze zwischen SCM und MPM
im Arbeitsbereich zeigen.
\par
Das größte Hindernis für den GRAM ist nach den Abschätzungen die
Abbremsung in der Plasmaschicht. Damit wird die starke
Ab"-hän"-gig"-keit vom Isolatorzustand plausibel, weil die Zündung
der Entladung über der Isolatoroberfläche einen großen Einfluß auf
die Schichtdicke hat.
\par
Unter bestimmten Umständen, die aufgrund der Abschätzungen plausibel
sind, aber hier nicht exakt berechnet werden können, kann der GRAM zur
Bildung einer stabilen Säule geführt haben. Dann aber können die
schnellen Ionen durch die hohe Dichte der Plasmasäule abgebremst und
thermalisiert werden. Dies führt dazu, daß das Magnetfeld die
Plasmasäule weiter komprimiert und der beobachtete zweite
Strahlungsausbruch im Röntgenbereich erfolgt (siehe Abschnitt
\vref{sec:zweiteKompression}).
\par
Die schnellen Deuteronen zeigten sich bei Entladungen in reinem
Deuterium auch durch ein deutlich früheres Einsetzen der
Neutronenproduktion (Beam-Beam- und Beam-Target-Reaktionen)
\cite{deutsch:87}. Dieses konnte bei Entladungen im SCM  nicht
eindeutig beobachtet werden. Im Mittel setzte die Neutronenproduktion
nur \wert{14}{ns} früher als beim MPM ein. Die Ursache dafür kann der
Anteil der Neon- bzw. Argon-Ionen an der schnellen Ionenkomponente
sein.
