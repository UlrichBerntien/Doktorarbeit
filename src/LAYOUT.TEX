%
%  Layout und andere Definitionen fuer die Doktorarbeit
%

%
%  In der Datei temp.tex wird \dviDriver definiert.
%  So kann aus der Batch der Treiber eingestellt werden.
%  Moeglich sind:
%
%   \newcommand{\dviDriver}{dvips}
%   \newcommand{\dviDriver}{dvipdfm}
%   \newcommand{\dviDriver}{emtex}
%
\input{temp}

\documentclass[\dviDriver,a4paper,10pt,oneside,final,german]{ub_artcl}

\usepackage[cp850]{inputenc}
\usepackage{babel}
\usepackage{varioref}
\usepackage{float}
\usepackage{hyperref}
\usepackage{graphics}

\bibliographystyle{alphadin}

\pagestyle{empty}

\textwidth150mm
\textheight235mm
\oddsidemargin0mm
\evensidemargin0mm
\topmargin-5mm

\nonfrenchspacing
\sloppy
\flushbottom

%
%  Anfang eines Kapitels
%
\newcommand{\beginsection}[1]{
    \clearpage
    \thispagestyle{onlypagenumbers}
    \cleardoublepage
    \vspace*{0pt plus 100pt minus 20pt}
    \section{#1}
    \thispagestyle{onlypagenumbers}
    \pagestyle{freeheadings}
    \markboth{\noexpand\thepage \hfill Kapitel \thesection}{#1 \hfill \noexpand\thepage}
    }

%
%  Anfang eines Abschnitts
%
\newcommand{\beginsubsection}[1]{
    \par
    \vspace*{0pt plus 30pt minus 10pt}
    \subsection{#1}
    }

%
%  Einfügen einer Abbildung
%  alle Abbildungen liegen im eps-Format für dvips
%  und im jpg-Format für dvipdfm vor
%
\DeclareOption{dvips}{\newcommand{\importimage}[1]{\includegraphics{#1.eps}}}
\DeclareOption{dvipdfm}{\newcommand{\importimage}[1]{\includegraphics{#1.jpg}}}
\DeclareOption{emtex}{\newcommand{\importimage}[1]{Datei: #1}}
\ProcessOptions

%
%  Kürzel für Isolatormaterial Al2O3
%
\newcommand{\AlO}{Al$_2$O$_3$}

%
%  Kürzel für effektive Injektionszeit
%
\newcommand{\teff}{$\tau _{\rm eff}$ }

%
%  Eine Zahl mit Einheit z.B:  10 kg
%
\newcommand{\wert}[2]{\mbox{$#1$\ #2}}

%
%  Eine Zahl mit Exponenten und Einheit z.B. 1*10^4 g
%
\newcommand{\ewert}[3]{\mbox{$#1 \cdot 10^{#2}$\ #3}}

%
%  Eine Zahl nur aus Exponenten und Einheit z.B. 1*10^4 g
%
\newcommand{\fwert}[2]{\mbox{$10^{#1}$\ #2}}

%
%  Besondere Trennungsregeln
%
\hyphenation{ließe}
