%
%  Anhang
%
\beginsection{Anhang}
%
%
\beginsubsection{Isolatoren von 1988 bis 1999}
\label{iso:liste}
%
Für das Auftreten des SCM hat sich der Zustand des Isolators als
wichtiger Parameter gezeigt (Abschnitt \vref{iso:problem}). Die
Problematik ist schon in der allgemeinen Formulierung \glqq
Zustand\grqq\ abzulesen. Eine genaue Untersuchung dieser Abhängigkeit
ist aufgrund der am Experiment gemachten Erfahrungen als sehr schwierig
einzustufen. Daher beschränkt sich diese Arbeit auf einen kurzen Blick
zurück, auf die lange Reihe der Versuche mit verschiedenen Isolatoren
in den letzten 10 Jahren.
\par
Die Liste gibt einen kleinen Eindruck von den auftretenden
Schwierigkeiten. Selbst Isolatoren aus einer Lieferung haben erhebliche
Unterschiede im Verhalten gezeigt. Untersuchungen der
Isolatoroberflächen gaben auch keine Hilfe für die Lösung des
Isolatorproblems \cite{kies:86}.
%
\newcommand{\fm}[5]{
    (a) \>#1 \>(b) #2 \>(c) #3 hPa\\*
    \>(d) #4\\*
    \>(e) #5\\[0pt plus10pt]
  }
%
\begin{tabbing}
(a) \=XX.XX.XXXX \=(b) XXXXXX--XXXXXX \=(c) \kill
    (a) \> Einbau am \>(b) Entladungen \>(c) p(D$_2$) Arbeitsbereich\\*
    \>(d) Typ des Isolators\\*
    \>(e) Ursache für den Ausbau\\[0pt plus 10pt]
\fm{    22.02.1988}
    {3146--3147}
    {5}
    {\AlO-Isolator}
    {Spannung hat den Isolator zerstört}
\fm{    29.06.1988}
    {3193--3383}
    {3--7.5}
    {\AlO-Isolator}
    {---}
\fm{    26.10.1988}
    {3390--3455}
    {2.5--4}
    {\AlO beschichteter Isolator}
    {---}
\fm{    15.11.1988}
    {3456--3556}
    {2.5--4.5}
    {bereits benutzter Glasisolator}
    {Umbau auf neuen Typ}
\fm{    26.01.1989}
    {3559--3563}
    {2.8--3.5}
    {\AlO-Isolator}
    {Spannung hat den Isolator zerstört}
\fm{    01.02.1989}
    {3564--3574}
    {3}
    {\AlO-Isolator}
    {Spannung hat den Isolator zerstört}
\fm{    13.02.1989}
    {3575--3583}
    {3--4}
    {\AlO-Isolator mit zusätzlichen Isolationsfolien}
    {Spannung hat den Isolator zerstört}
\fm{    22.02.1989}
    {3585--3589}
    {5}
    {\AlO-Isolator mit kunstharzgetränkten Isolationsfolien}
    {Spannung hat den Isolator zerstört}
\fm{    27.02.1989}
    {3590--3612}
    {2.5--3.5}
    {Glasisolator}
    {---}
\fm{    22.03.1989}
    {3613--3629}
    {---}
    {Glasisolator}
    {keine effektive Pinchbildung}
\fm{    05.06.1989}
    {3631--3668}
    {3.5--5}
    {\AlO beschichteter Quarzglas-Isolator}
    {Beschichtung ist abgesplittert}
\fm{    12.06.1989}
    {3669--3716}
    {3.5--5}
    {\AlO beschichteter Quarzglas-Isolator}
    {vermutlich ein mechanischer Defekt}
\fm{    31.07.1989}
    {3717--3795}
    {4--5}
    {\AlO beschichteter Quarzglas-Isolator}
    {---}
\fm{    25.09.1989}
    {3796--3831}
    {3--4.5}
    {\AlO beschichteter Quarzglas-Isolator}
    {Beschichtung bei den Entladungen abgesplittert}
\fm{    03.10.1989}
    {3832--3963}
    {3.5--5}
    {\AlO beschichteter Quarzglas-Isolator}
    {Isolator zerbrochen}
\fm{    26.10.1989}
    {3964--4031}
    {4}
    {\AlO beschichteter Quarzglas-Isolator}
    {Isolator zerbrochen}
\fm{    29.10.1989}
    {4032--4400}
    {4}
    {Duranglas-Isolator}
    {zu geringe Neutronenproduktion}
\fm{    02.04.1990}
    {4401--5030}
    {3--5}
    {Duranglas-Isolator}
    {---}
\fm{    17.10.1990}
    {5031--5156}
    {3.4--5}
    {Duranglas-Isolator}
    {---}
\fm{    12.12.1990}
    {5158--5176}
    {3.5--4}
    {\AlO-Isolator, 10fache Dichte bzgl. 3146}
    {Spannung hat den Isolator zerstört}
\fm{    20.12.1990}
    {5177--5591}
    {4--5}
    {\AlO-Isolator wie 5158 mit Isolationsfolien}
    {mechanische Zerstörung}
\fm{    25.06.1991}
    {5592--5605}
    {4}
    {\AlO-Isolator mit Isolationsfolien wie 5177}
    {Spannung hat den Isolator zerstört}
\fm{    27.06.1991}
    {5606--5630}
    {2.5--4}
    {Duranglas-Isolator}
    {Umbau auf anderen Isolator}
\fm{    02.07.1991}
    {5635--5648}
    {3.5}
    {\AlO-Isolator mit Isolationsfolien wie 5177}
    {Folien verrutscht}
\fm{    04.07.1991}
    {5649--5691}
    {4}
    {\AlO-Isolator mit Isolationsfolien wie 5177}
    {Spannung hat den Isolator zerstört}
\fm{    10.07.1991}
    {5695--5722}
    {4--5}
    {\AlO-Isolator mit Isolationsfolien wie 5177}
    {Spannung hat den Isolator zerstört}
\fm{    17.07.1991}
    {5725--5769}
    {4--5}
    {\AlO-Isolator mit Isolationsfolien wie 5177}
    {---}
\fm{    24.07.1991}
    {5770--5849}
    {3.5--4}
    {\AlO-Isolator mit Isolationsfolien wie 5177}
    {Umbau auf neuen Isolator}
\fm{    04.11.1991}
    {5850--5855}
    {---}
    {Duranglas-Zylinder auf Delrin-Kern mit Silikonkleber}
    {keine effizienten Pinche}
\fm{    25.11.1991}
    {5856--5885}
    {3}
    {Duranglas-Zylinder auf Delrin-Kern mit Uhu endfest300}
    {mechanisch zerstört}
\fm{    05.12.1991}
    {5888--5925}
    {4--5}
    {\AlO-Zylinder auf Delrin-Kern mit Uhu endfest300}
    {mechanisch zerstört}
\fm{    23.12.1991}
    {5826--5969}
    {4}
    {\AlO-Isolator von Degussa}
    {gebrochen}
\fm{    24.02.1992}
    {5870--5998}
    {4--4.5}
    {\AlO-Isolator von Degussa, Stirnfläche mit Kupferring}
    {gebrochen}
\fm{    12.03.1992}
    {5999--6199}
    {4}
    {\AlO-Isolator von Degussa wie 5870}
    {Umbau auf neuen Isolator}
\fm{    24.03.1992}
    {6200--6236}
    {---}
    {\AlO-Staub beschichteter Duranglas-Isolator}
    {keine effizienten Entladungen}
\fm{    27.03.1992}
    {6237--6252}
    {---}
    {dichter mit \AlO-Staub beschichteter Duranglas-Isolator}
    {keine effizienten Entladungen}
\fm{    01.04.1992}
    {6253}
    {---}
    {Quarzglas-Isolator mit \AlO-Folie beklebt}
    {zerstört}
\fm{    06.04.1992}
    {6254--6270}
    {---}
    {Quarzglas-Isolator}
    {keine effizienten Entladungen}
\fm{    01.07.1992}
    {6271--6272}
    {---}
    {Quarzglas mit \AlO beschichtet von VAW}
    {Gleitentladung über beschichteten Isolatorfuß}
\fm{    08.07.1992}
    {6273--6861}
    {3.5--4}
    {wie 6271, Fußbereich abgeschliffen}
    {Oberfläche verschmutzt}
\fm{    06.05.1993}
    {6862--7047}
    {4.0--5.0}
    {wie 6273}
    {Beschichtung beschädigt}
\fm{    11.06.1993}
    {7048--7257}
    {5.0--8.0}
    {wie 6273, Oberfläche durch schleifen geglättet}
    {Oberfläche beschädigt}
\fm{    13.08.1993}
    {7258--7968}
    {5.0--8.0}
    {wie 7048}
    {keine homogene Zündung auf dem Isolator}
\fm{    17.01.1994}
    {7969--8286}
    {5.4--7.5}
    {wie 7048}
    {absplittern der Beschichtung}
\fm{    02.03.1994}
    {8281--9495}
    {4.6--8.6}
    {wie 7048}
    {Umbau auf neuen Isolator}
\fm{    03.04.1994}
    {9496--11064}
    {4.5--5.4}
    {Quarzglas \AlO beschichtet (LWK), durch Schliff geglättet}
    {Beschichtung abgesplittert}
\fm{    19.03.1997}
    {11065--11573}
    {4.8--5.8}
    {wie 9496}
    {Spannung hat den Isolator zerstört}
\fm{    15.10.1997}
    {11574--11598}
    {4.8--8.2}
    {wie 9496}
    {Durchschlag bei sehr hoher Pinchspannung}
\fm{    22.10.1997}
    {11599-11724}
    {4.8--5.4}
    {wie 9496, Oberfläche durch Längsschliff geglättet}
    {nur jede 2. Entladung war effektiv}
\fm{    19.01.1998}
    {11725--11785}
    {4.8}
    {wie 9496, Oberfläche durch Radialschliff geglättet}
    {mechanische Zerstörung}
\fm{    09.02.1998}
    {11787--11987}
    {9.7--10.2}
    {Isolator von 9495 wiederverwendet}
    {Beschichtung ist abgesplittert}
\fm{    06.04.1998}
    {11988--12318}
    {8.8--10.2}
    {wie 9496}
    {nach Abplatzen der Beschichtung keine effektiven Pinche}
\fm{    14.10.1998}
    {12319--12417}
    {9.0--9.4}
    {wie 9496 mit abgerundeter Stirnfläche}
    {langsames Absplittern von der Stirnfläche}
\fm{    28.10.1998}
    {11418--12485}
    {8.8--10.2}
    {wie 12319}
    {Teil der Beschichtung der Mantelfläche abgesplittert}
\fm{    10.11.1998}
    {12486--12661}
    {7.5--8.4}
    {wie 12319}
    {Teil der Beschichtung (ca. \wert{20}{mm$^2$}) abgesplittert}
\fm{    10.03.1999}
    {12662--12739}
    {9.9--10.9}
    {wie 12319, Oberfläche etwas stärker geglättet}
    {erst Beschichtung abgesplittert, dann Glas zertrümmert}
\fm{    08.04.1999}
    {12740--12771}
    {8.2--10.2}
    {wie 12662}
    {Glas zertrümmert, durch eine heftige Speiche?}
\fm{    15.04.1999}
    {12772--12809}
    {6.1--7.5}
    {einen benutzten Isolator von LWK neu mit \AlO beschichtet}
    {erst Beschichtung abgesplittert, dann Glas zertrümmert}
\fm{    23.04.1999}
    {12810--12840}
    {7.5}
    {wie 7048}
    {erst Beschichtung abgesplittert, dann Glas zertrümmert}
\end{tabbing}
%
%  für die Literaturliste, auch ohne expliziten Verweis
%
\nocite{kopka:latex}
